\documentclass[ngerman,final,fontsize=12pt,paper=a4,twoside,bibliography=totocnumbered,BCOR=8mm,draft=false]{scrartcl}

\usepackage[T1]{fontenc}
\usepackage[ngerman]{babel}
\usepackage[utf8]{inputenx}
\usepackage[sort&compress,square]{natbib}
\usepackage[babel]{csquotes}
\usepackage[hyphens]{url}
\usepackage[draft=false,final,plainpages=false,pdftex]{hyperref}
\usepackage{eso-pic}
\usepackage{graphicx}
\usepackage{xcolor}
\usepackage{pdflscape}
\usepackage{longtable}
\usepackage{framed}
\usepackage{textcomp}

\usepackage[charter,sfscaled]{mathdesign}

%\usepackage[spacing=true,tracking=true,kerning=true,babel]{microtype}
\usepackage[spacing=true,kerning=true,babel]{microtype}

%\setparsizes{1em}{.5\baselineskip}{0pt plus 1fil}

\author{VroniPlag} 

\title{Gemeinschaftlicher Bericht}
\subtitle{Dokumentation von Plagiaten in der Dissertation „Test“ von Max Mustermann, Wakaluba, 2012}
\publishers{\url{http://de.vroniplag.wikia.com/wiki/Lm/Bericht/Entwurf}}

\hypersetup{%
        pdfauthor={VroniPlag},%
	pdftitle={Gemeinschaftlicher Bericht -- Dokumentation von Plagiaten in der Dissertation „Test“ von Max Mustermann, Wakaluba, 2012},%
        pdflang={en},%
        pdfduplex={DuplexFlipLongEdge},%
        pdfprintscaling={None},%
	linktoc=all,%
	colorlinks,%
	linkcolor=black,%
	citecolor=green!50!black,%
	filecolor=blue,%
	urlcolor=blue,%
	linkbordercolor={1 0 0},%
	citebordercolor={0 0.5 0},%
	filebordercolor={0 0 1},%
	urlbordercolor={0 0 1},%
}

\definecolor{shadecolor}{rgb}{0.95,0.95,0.95} 

\newenvironment{fragment}
	{\begin{snugshade}}
	{\end{snugshade}
	 \penalty-200
	 \vskip 0pt plus 10mm minus 5mm}
\newenvironment{fragmentpart}[1]
	{\noindent\textbf{#1}\par\penalty500}
	{\par}
\newcommand{\BackgroundPic}
	{\put(0,0){\parbox[b][\paperheight]{\paperwidth}{%
		\vfill%
		\centering%
		\includegraphics[width=\paperwidth,height=\paperheight,%
			keepaspectratio]{background.png}%
		\vfill%
	}}}

%\setkomafont{chapter}{\Large}
\setkomafont{section}{\large}
\addtokomafont{disposition}{\normalfont\boldmath\bfseries}
\urlstyle{rm}

\begin{document}


%\AddToShipoutPicture*{\BackgroundPic}
\maketitle\thispagestyle{empty}
%\ClearShipoutPicture

\tableofcontents
\newpage

\section{Einleitung}

Die ist ein Testbericht



\appendix
\section{Textnachweise}

%Ignoriere Lm/Fragment 002 21: Kategorie in Blacklist: Kategorie:Verdächtig
%Ignoriere Lm/Fragment 011 06: Kategorie Kategorie:Geprüft nicht gesetzt
%XXX: Lm/Fragment 021 01: Warnung, Diskrepanz zwischen Fragment und Kategorisierung! (BauernOpfer != Verschleierung)
%XXX: Lm/Fragment 049 115: Warnung, Diskrepanz zwischen Fragment und Kategorisierung! (BauernOpfer != Verschleierung)
%Ignoriere Lm/Fragment 058 10: Kategorie Kategorie:Geprüft nicht gesetzt
%Ignoriere Lm/Fragment 060 01: Kategorie in Blacklist: Kategorie:Verdächtig
%Ignoriere Lm/Fragment 069 03: Kategorie Kategorie:Geprüft nicht gesetzt
%Ignoriere Lm/Fragment 168 13: Kategorie Kategorie:Geprüft nicht gesetzt
%XXX: Lm/Fragment 184 28: Warnung, Diskrepanz zwischen Fragment und Kategorisierung! (Verschleierung != BauernOpfer)
%XXX: Lm/Fragment 203 09: Warnung, Diskrepanz zwischen Fragment und Kategorisierung! ( != BauernOpfer)
%Ignoriere Lm/Fragment 205 01: Kategorie Kategorie:Geprüft nicht gesetzt
\phantomsection{}
\belowpdfbookmark{Fragment 1 5--12}{Lm-Fragment-001-05}
\hypertarget{Lm-Fragment-001-05}{}
\begin{fragment}
\begin{fragmentpart}{Dissertation S.~1 Z.~5--12 (Verschleierung)}
\enquote{Die Bedeutung des internationalen Privatrechts ist im Laufe des 20. Jahrhunderts um ein Vielfaches gestiegen. Die \textsl{Eigenart} des Kollisionsrechts ergibt sich aus seiner Funktion als Weisungsrecht \textsl{über} Rechtsordnungen mit stark limitierter Weisungsbefugnis. Die hier und jetzt positiv geltende heimische Rechtsordnung wird relativiert selbst durch die Betrachtung anderer, inhaltlich abweichender Normen, seien diese von der Rechtsphilosophie theoretisch entworfen werden, oder von der Rechtsgeschichte und der Rechtsvergleichung empirisch in anderen Ländern, Zeiten und Räumen vorgefunden.}
\end{fragmentpart}
\begin{fragmentpart}{Original \cite[S.~2 Z.~5--6,~10--11,~18--22]{Kropholler-1997}}
\enquote{Die \textsl{Bedeutung} des IPR ist im Laufe des 20. Jahrhunderts um ein Vielfaches gestiegen. $[$...$]$

Die \textsl{Eigenart} des IPR ergibt sich aus seiner Funktion als Recht „über“ Rechtsordnungen. $[$...$]$

Die hier und jetzt positiv geltende heimische Rechtsordnung wird relativiert durch die Betrachtung anderer, inhaltlich abweichender Normen, seien diese nach den Grundsätzen der Rechtsphilosophie theoretisch entworfen oder von der Rechtsgeschichte und der Rechtsvergleichung empirisch in anderen Zeiten und Räumen vorgefunden.}
\end{fragmentpart}
\begin{fragmentpart}{Anmerkung}
Ein Paukenschlag zum Auftakt. Die ersten Sätze der Einleitung stammen aus dem Lehrbuch von Kropholler, dem Herausgeber der Serie, in der die Arbeit des Verfassers erschienen ist.
\end{fragmentpart}
\end{fragment}
\phantomsection{}
\belowpdfbookmark{Fragment 1 101--108}{Lm-Fragment-001-101}
\hypertarget{Lm-Fragment-001-101}{}
\begin{fragment}
\begin{fragmentpart}{Dissertation S.~1 Z.~101--108 (Verschleierung)}
\enquote{$[$FN 1$]$ So in deutscher, französischer (qualification), italienischer (qualificazione), portugiesischer (qualifica\&ccedil;\&atilde;o), spanischer (calificación), niederländischer (qualificatie oder kwalificatie), schwedischer (kvalifikation) Sprache und in allen slawischen Sprachen. Auf griechisch heißt es \&nu;\&omicron;\&mu;\&kappa;ó\&sigmaf; \&chi;\&alpha;\&rho;\&alpha;\&kappa;\&tau;\&eta;\&rho;\&iota;\&sigma;\&mu;ó\&sigmaf; (juristische Charakterisierung). Dagegen setzte sich im Englischen kein einheitlicher Begriff durch. Man spricht dort von \textsl{characterization} und
\textsl{classification}. Zu dem Wort „Qualifikation“ s. HELMUT WEBER, Die Theorie der Qualifikation. Franz Kahn --- Etienne Bartin und die Entwickung ihrer Lehre 1890-1945, Tübingen 1986, S. 197-202.}
\end{fragmentpart}
\begin{fragmentpart}{Original \cite[S.~4 Z.~117--121]{Weber-1986}}
\enquote{$[$FN 14$]$ So u.a. auf deutsch, französisch (qualification), italienisch (qualificazione), portugiesisch (qualifica\&ccedil;\&atilde;o), spanisch (calificación), niederländisch (qualificatie oder kwalificatie), schwedisch (kvalifikation); s. dazu auch Kapitel 12. Dagegen setzte sich im Englischen kein einheitlicher Begriff durch. Man spricht dort von ,characterization‘ und ,classification‘; s. dazu S. 167 u. Kapitel 14.}
\end{fragmentpart}
\begin{fragmentpart}{Anmerkung}
Die von Lm angegebenen Seiten 197-202 umfassen genau Kapitel 12 von Weber (1986). Obwohl diese Quelle angegeben wird, ist überhaupt nicht deutlich, dass auch die vorliegende Stelle offensichtlich schon Weber entspringt (und zwar Kapitel 1 und nicht dem Kapitel 12: der Verweis auf Kapitel 12 stammt auch aus dem Text in der Quelle).

Man beachte, dass die Reihenfolge der angegebenen Sprachen identisch ist und mit dem Englischen abschließt, von dem beide Autoren in identischer Formulierung feststellen, dass dort kein \textquotedbl{}einheitlicher Begriff\textquotedbl{} existiert. 

Kaum überraschend: neu bei dem gebürtigen Griechen Lm ist die griechische Übersetzung des Begriffs.
\end{fragmentpart}
\end{fragment}
\phantomsection{}
\belowpdfbookmark{Fragment 2 101--106}{Lm-Fragment-002-101}
\hypertarget{Lm-Fragment-002-101}{}
\begin{fragment}
\begin{fragmentpart}{Dissertation S.~2 Z.~101--106 (BauernOpfer)}
\enquote{$[$FN 2$]$ DÖLLE illustrierte dies in einem Festvortrag, seiner bekannter $[$sic!$]$ Rede vor dem 42. DJT in Düsseldorf, an Hand einer Reihe von Beispielen aus der Rechtsentwicklung dieses und des vergangenen Jahrhunderts. Als letztes, aber besonders treffendes Beispiel nannte er die Qualifikation im IPR; cf. HANS DÖLLE, Juristische Entdeckungen, in Verhandlungen des 42. DJT, Bd. II, Tübingen 1959, S. BI-B22 (B3, B19); vgl. auch Weber, a.a.O. (Fn. 1), S. 3-6 m.w.N.}
\end{fragmentpart}
\begin{fragmentpart}{Original \cite[S.~3 Z.~3--4,~15--18]{Weber-1986}}
\enquote{In seiner bekannten Rede vor dem 42. Deutschen Juristentag$[$FN 1$]$ in Düsseldorf hat Hans DÖLLE $[$...$]$

DÖLLE illustrierte dies in seinem Festvortrag an Hand einer Reihe von Beispielen aus der Rechtsentwicklung dieses und des vergangenen Jahrhunderts$[$FN 3$]$. Als letztes, aber besonders treffendes Beispiel nannte er die \textsl{Qualifikation} im Internationalen Privatrecht$[$FN 4$]$.

$[$FN 1$]$ DÖLLE, Entdeckungen (1957).

$[$FN 4$]$ S. B 19ff.}
\end{fragmentpart}
\begin{fragmentpart}{Anmerkung}
Eine Fußnote wird aus Originalformulierungen und Literaturverweisen aus Weber (1986) zusammengebaut. Art und Umfang der Übernahme bleibt trotz Nennung der Quelle ungeklärt.
\end{fragmentpart}
\end{fragment}
\phantomsection{}
\belowpdfbookmark{Fragment 7 107--112}{Lm-Fragment-007-107}
\hypertarget{Lm-Fragment-007-107}{}
\begin{fragment}
\begin{fragmentpart}{Dissertation S.~7 Z.~107--112 (BauernOpfer)}
\enquote{$[$FN 15$]$In der Tat ist es in der Rechtswissenschaft bekanntlich ein gängiges Verfahren, eine Theorie mit ihren „Ergebnissen“ zu konfrontieren und sie wegen deren „Unhaltbarkeit“ zu verwerfen. Diese Vorgehensweise hat eine offenkundige Ähnlichkeit mit dem Falsifikationismus, wie er vor allem von POPPER entwickelt worden ist (POPPER, Logik der Forschung, 10. Aufl., 1994, S. 8, 52-59, 73, 83-85, 87-89, 199-208). Das Grundschema POPPERS läßt sich nämlich \textsl{mutatis mutandis} auch in der Jurisprudenz verwenden.}
\end{fragmentpart}
\begin{fragmentpart}{Original \cite[S.~386 Z.~7--13,~28--29]{Canaris-1993}}
\enquote{In der Tat ist es in der Rechtswissenschaft bekanntlich ein gängiges Verfahren, eine Theorie mit ihren „Ergebnissen“ zu konfrontieren und sie wegen deren „Unhaltbarkeit“ zu verwerfen.

a) Diese Vorgehensweise hat eine offenkundige Ähnlichkeit mit dem \textsl{Falsifikationismus}, wie er vor allem von \textsl{Popper} entwickelt worden ist. $[$...$]$

Das Grundschema \textsl{Poppers} läßt sich nämlich mutatis mutandis auch in der Jurisprudenz verwenden.}
\end{fragmentpart}
\begin{fragmentpart}{Anmerkung}
Eine wortgetreue Übernahme. Canaris wird in derselben Fußnote für einen weiterführenden Gedanken mit einem als solchem gekennzeichneten und nahezu korrekt wiedergegebenen wörtlichen Zitat angeführt.
\end{fragmentpart}
\end{fragment}
\phantomsection{}
\belowpdfbookmark{Fragment 8 1--3}{Lm-Fragment-008-01}
\hypertarget{Lm-Fragment-008-01}{}
\begin{fragment}
\begin{fragmentpart}{Dissertation S.~8 Z.~1--3 (BauernOpfer)}
\enquote{$[$Die Falsifikation bedeutet ja nur,$]$ daß die Theorie \textsl{in ihrer derzeitigen Formulierung} nicht völlig richtig sein kann. Folglich ist im Grundsatz nicht ausgeschlossen, sich um ihre Verbesserung bzw. Präzisierung zu bemühen.}
\end{fragmentpart}
\begin{fragmentpart}{Original \cite[S.~388 Z.~14--17]{Canaris-1993}}
\enquote{Denn Falsifikation bedeutet ja nur, daß die Theorie \textsl{in ihrer derzeitigen Formulierung} nicht richtig sein kann. Folglich steht grundsätzlich
nichts im Wege, sich um ihre Verbesserung zu bemühen.}
\end{fragmentpart}
\begin{fragmentpart}{Anmerkung}
Canaris wird im Laufe der vorausgehenden Fußnote 15 als Beleg für einen weiterführenden Gedanken genannt, siehe \hyperlink{Lm-Fragment-007-107}{Fragment\_007\_107}. Alternativ könnte man das als Verschleierung werten.
\end{fragmentpart}
\end{fragment}
\phantomsection{}
\belowpdfbookmark{Fragment 17 8--15,~109--115}{Lm-Fragment-017-08}
\hypertarget{Lm-Fragment-017-08}{}
\begin{fragment}
\begin{fragmentpart}{Dissertation S.~17 Z.~8--15,~109--115 (Verschleierung)}
\enquote{Allgemeine Kritik am internationalen Privatrecht$[$40$]$ wird durch ein bestimmtes Bild vom IPR geprägt, dessen Ausgangspunkt die zweiseitigen Kollisionsnormen sind. Das überkommene IPR-System$[$41$]$ besteht aus den klassischen Kollisionsnormen, die jeweils für bestimmte Rechtsverhältnisse oder auch Lebenssachverhalte aufgrund vorher festgelegter abstrakter Anknüpfungspunkte (Staatsangehörigkeit, Belegenheit von Sachen, Ort des Vertragsabschlusses, Parteiautonomie u.a.m.) pauschal eine staatliche (nationale) Rechtsordnung berufen.

$[$41$]$ Cf. etwa CAVERS, A Critique of the Choice-of-Law Problem, Harv. L. Rev. 47 (1933), S. 173-208; EHRENZWEIG, Private International Law, Leyden 1974, S. 75f.; P. M. GUTZWILLER, Von Ziel und Methode des IPR, SchwJbIR XXV (1968), S. 161-194 (169); HANCOCK, Three Approaches to the Choice-of-Law Problem; The Classificatory, the Functional and the Result-Selective, FS fur H. Yntema, 1961, S. 365-375 (365f., 379); JOERGES, Zum Funktionswandel des Kollisionsrechts, Berlin 1971, S. 4; JUENGER, Zum Wandel des internationalen Privatrechts, Karlsruhe 1976, S. 6-14.}
\end{fragmentpart}
\begin{fragmentpart}{Original \cite[S.~15 Z.~5--6,~11--15]{Schurig-1981}}
\enquote{Die Kritik \textsl{am} internationalen Privatrecht ist bedingt durch ein bestimmtes Bild \textsl{vom} Internationalen Privatrecht. $[$...$]$

Internationales Privatrecht besteht so gesehen$[$1$]$ aus den Normen, die jeweils für bestimmte Rechtsverhältnisse oder auch Lebenssachverhalte aufgrund vorher festgelegter Anknüpfungsmomente (Wohnsitz, Staatsangehörigkeit, Geschäftsort, Lageort und anderes mehr) pauschal eine staatliche Rechtsordnung berufen.

$[$1$]$ Vgl. etwa \textsl{Cavers}, Crit. 173-182; \textsl{Ehrenzweig}, P.I.L. 75 f.; \textsl{Gutzwiller}, Ziel 169; \textsl{Hancock}, Appr. 365f., 379; \textsl{Juenger}, Wandel 6-14; \textsl{Rehbinder}, Polit. 151; \textsl{Joerges}, Funktionswandel 4; \textsl{Kelly}, Confl. 32; ''Jessurun d'Oliveira\textsl{, Ruïne 5 f.; }de Boer\textsl{, Tekort; }Leflar'', Confl. 173 f., 198.}
\end{fragmentpart}
\begin{fragmentpart}{Anmerkung}
Der übernommene Text samt Fußnote wird bearbeitet, der Gedankengang übernommen, elf zusammenhängende Wörter ebenfalls, und Schurig wird hier nicht erwähnt (Schurig wird erstmalig in FN 42 am Ende der Seite erwähnt). Im ersten Satz werden stilprägende Elemente entfernt.
\end{fragmentpart}
\end{fragment}
\phantomsection{}
\belowpdfbookmark{Fragment 18 8--12}{Lm-Fragment-018-08}
\hypertarget{Lm-Fragment-018-08}{}
\begin{fragment}
\begin{fragmentpart}{Dissertation S.~18 Z.~8--12 (BauernOpfer)}
\enquote{SAVIGNY erwartete, muß man wohl sagen, einen weltweiten Konsens darüber, welche Anknüpfung jeden Richter in der Welt in jedem Kollisionsfall zur Anwendung desselben Rechts fuhren würde. Er dachte sich ein einheitliches Kollisionsrecht, das sich ohne Eingriffe nationaler Gesetzgeber in Wissenschaft und Praxis entwickeln sollte.$[$FN 46$]$
----

$[$FN 46$]$ ZWEIGERT, Zur Armut des internationalen Privatrechts an sozialen Werten, RabelsZ 37 (1973), S. 436; vgl. Flessner, Fakultatives Kollisionsrecht, RabelsZ 34 (1970), S. 557f.; SAKURADA, Wirkungsbereich und Funktion des Kollisionsrechts --- Einige Gedanken über Savignys IPR, in Holl \& Klinke (Hrsg.), Internationales Privatrecht --- Internationales Wirtschaftsrecht, Köln usw. 1985, S. 127-144 (129-141). Kritisch in England ist BECKETT, What Is Private International Law, B.Yb.I.L. 7 (1926), S. 73-96.}
\end{fragmentpart}
\begin{fragmentpart}{Original \cite[S.~436 Z.~17--21]{Zweigert-1973}}
\enquote{\textsl{Savigny} erwartete einen weltweiten Konsens darüber, welche Anknüpfung jeden Richter in der Welt in jedem Kollisionsfall zur Anwendung desselben Rechts führen würde. Er dachte sich ein einheitliches Kollisionsrecht, das sich ohne Eingriffe nationaler Gesetzgeber in Wissenschaft und Praxis entwickeln sollte$[$FN 6$]$.
----
$[$FN 6$]$ Obwohl er nicht mehr — wie Jahre vorher (\textsl{Savigny}, Vom Beruf unsrer Zeit für Gesetzgebung und Rechtswissenschaft $[$1814$]$) — vor nationaler Gesetzgebung warnte, befürwortete er dennoch nur den Abschluß internationaler Abkommen: \textsl{Savigny} a.a.O. 30—32.}
\end{fragmentpart}
\begin{fragmentpart}{Anmerkung}
Zweigert wird in Fußnote 46 erwähnt, die in für diese Arbeit typischer Weise mit weiteren Belegen angereichert wird.
\end{fragmentpart}
\end{fragment}
\phantomsection{}
\belowpdfbookmark{Fragment 20 5--10}{Lm-Fragment-020-05}
\hypertarget{Lm-Fragment-020-05}{}
\begin{fragment}
\begin{fragmentpart}{Dissertation S.~20 Z.~5--10 (KomplettPlagiat)}
\enquote{Selbstverständlich ist, daß internationales Verfahrensrecht und das Prozeßrecht der lex fori$[$50$]$ die Zuständigkeit, die Verfahrensvoraussetzungen und den organisatorischen Ablauf des Prozesses bestimmen.$[$51$]$ Unklar ist, wieweit inhaltliche Entscheidungsregeln heimischen oder fremden Verfahrensrechts „mit-“ oder „hineinwirken“ in die Entscheidung zur Sache.$[$52$]$

$[$50$]$ $[$...\}

$[$51$]$ Schack, Internationales Zivilverfahrensrecht, 2. Aufl. München 1996, Rn 1-29, 34-52, S. 1-10, 11-18 m.w.N. Über die Komplikationen, die sich aus einem genauen Zugriff auf einzelne Entscheidungsaufgaben wie z.B. die Zulässigkeit eines Festellungsantrags ergeben können vgl. Stoll, a.a.O. (Fn. 35): „... Das Kollisionsrecht ist geradezu der Prüfstein für die prozeßrechtliche oder materiellrechtliche Natur eines Rechtsinstituts (S. 365).

$[$52$]$ Diese Frage ist öfters positiv geantwortet $[$sic$]$: Niederländer, a.a.O. (Fn. 35)...}
\end{fragmentpart}
\begin{fragmentpart}{Original \cite[S.~609 Z.~5--10]{Hartwieg-1993}}
\enquote{Selbstverständlich ist, daß Internationales Verfahrensrecht und das Prozeßrecht der lex fori die Zuständigkeit, die Verfahrensvoraussetzungen und den organisatorischen Ablauf des Prozesses bestimmen$[$2$]$. Unklar ist, wieweit inhaltliche Entscheidungsregeln heimischen oder fremden Verfahrensrechts »mit-« oder »hineinwirken« in die Entscheidung zur Sache.

$[$2$]$ Haimo Schack, Internationales Zivilverfahrensrecht (München 1991) § 11—III, § 2 III—V (S. 1-10 $[$Rz. 1-29$]$, S. 12-19 $[$Rz. 34-52$]$) mit Nachweisen. Über die Komplikationen, die sich aus einem genauen Zugriff auf einzelne Entscheidungsaufgaben wie z. B. die Zulässigkeit eines Feststellungsantrags ergeben können, vgl. Hans Stoll, Typen der Feststellungsklage aus der Sicht des bürgerlichen Rechts, in: FS Eduard Bötticher (Berlin 1969) 341-368 (365-367).}
\end{fragmentpart}
\begin{fragmentpart}{Anmerkung}
Lm übernimmt in seiner Einleitung 10 zusammenhängende Zeilen (Text und dazugehörige Fußnote) wörtlich von seinem Doktorvater, ohne ihn in diesem Zusammenhang zu erwähnen. Eine Erwähnung von Hartwieg findet sich erst in \hyperlink{Lm-Fragment-021-01}{Fragment\_021\_01} eine Seite weiter unten.
\end{fragmentpart}
\end{fragment}
\phantomsection{}
\belowpdfbookmark{Fragment 20 15--20,~101--104}{Lm-Fragment-020-15}
\hypertarget{Lm-Fragment-020-15}{}
\begin{fragment}
\begin{fragmentpart}{Dissertation S.~20 Z.~15--20,~101--104 (BauernOpfer)}
\enquote{Im deutschen Zivilverfahrensrecht besteht ein Unterschied zwischen der \textquotedbl{}freiwilligen\textquotedbl{} und der \textquotedbl{}streitigen\textquotedbl{} Gerichtsbarkeit mit verschiedenen Verfahrensgesetzen: in der freiwilligen Gerichtsbarkeit (Familien- und Erbrecht) sind nach § 12 FGG von Amts wegen \textsl{alle} Tatsachen zu erforschen und bei der Entscheidung zu berücksichtigen. Hier hat im Tatsächlichen wie im Rechtlichen die „umfassende“ kollisionsrechtliche Würdigung, d.h. die Suche nach $[$dem „Sitz des Rechtsverhältnisses“, ihren Platz.$[$FN 53$]$$]$

$[$FN 53$]$ Cf. SAVIGNY, a.a.O. (Fn. 42) und V. BAR, IPRI, Rn 506-513, S. 432-438; FIRSCHING/V. HOFFMANN, IPR5, S. 165-197; KEGEL, IPR7, S. 227-238; KOCH / MAGNUS / WINKLER V. MOHRENFELS, IPR2, S. 8-19; KROPHOLLER, IPR3, §3; LÜDERITZ, IPR2, S. 13-20; RAAPE / STURM, IPR I, S. 98-102.}
\end{fragmentpart}
\begin{fragmentpart}{Original \cite[S.~611 Z.~4--10]{Hartwieg-1993}}
\enquote{Das deutsche Zivilverfahrensrecht demonstriert mit verschiedenen Verfahrensgesetzen den Unterschied: In der »freiwilligen Gerichtsbarkeit« (Familien- und Erbrecht) sind nach § 12 FGG von Amts wegen \textsl{alle} Tatsachen zu erforschen und bei der Entscheidung zu berücksichtigen. Hier hat im Tatsächlichen wie im Rechtlichen die »umfassende« kollisionsrechtliche Würdigung (Artt. 13-26 EGBGB), d. h. die Suche nach dem »Sitz des Rechtsverhältnisses«, ihren Platz$[$FN 5$]$.

$[$FN 5$]$ \textsl{v. Bar}, IPR I S. 432-438 (Rz. 506-513); \textsl{Kegel} S. 185-196 (§ 6); \textsl{Koch/Magnus/Winkler v. Mohrenfels} S. 7-15 (§ 1 II); \textsl{Kropholler} S. 14-21 (§3); \textsl{Leo Raape/Fritz Sturm}, Internationales Privatrecht I: Allgemeine Lehren (München 1977) S. 98-102 (§6).}
\end{fragmentpart}
\begin{fragmentpart}{Anmerkung}
Identisch in Formulierung, Hervorhebungen und Literaturverweisen, aber nicht als Zitat gekennzeichnet.

Fortsetzung in \hyperlink{Lm-Fragment-021-01}{Fragment\_021\_01}. Dort wird Hartwieg genannt.
\end{fragmentpart}
\end{fragment}
\phantomsection{}
\belowpdfbookmark{Fragment 21 1--16}{Lm-Fragment-021-01}
\hypertarget{Lm-Fragment-021-01}{}
\begin{fragment}
\begin{fragmentpart}{Dissertation S.~21 Z.~1--16 (Verschleierung)}
\enquote{$[$\{\{highlight|linen|in der freiwilligen Gerichtsbarkeit (Familien- und Erbrecht) sind nach § 12 FGG von Amts wegen alle Tatsachen zu erforschen und bei der Entscheidung zu berücksichtigen. Hier hat im Tatsächlichen wie im Rechtlichen die „umfassende“ kollisionsrechtliche Würdigung, d.h. die Suche nach\}\}$]$ 

\{\{highlight|linen|dem „Sitz des Rechtsverhältnisses“, ihren Platz.\}\}$[$FN 53$]$ \{\{highlight|linen|Anders im Vermögensrecht (Schuld- und Sachenrecht gemäß Artt. 27-38 EGBGB) des streitigen Zivilprozesses. Hier ist ein deutsches Gericht in den Rechtsfolgen nach § 308 Abs. 1 ZPO an die wechselseitigen Anträge der Parteien gebunden und kann dank der Verhandlungsmaximen (vgl. §§ 128, 138, 288, 330 I 1 ZPO) von unstreitigen Tatsachenvorgaben im Tatbestandsbereich des materiellen Rechts ausgehen.\}\}$[$FN 54$]$ 

\{\{highlight|linen|Das Rechtsverständnis des common law betont die Parteiherrschaft noch deutlicher im „adversialen“ Verfahren. Die Parteien definieren in ihrem Sachvortrag alle entscheidungswichtigen Elemente des Streits (Tatsachen, Rechtswertungen\}\} und Ansprüche). Das sind die sog. \textsl{pleading}.$[$FN 55$]$ 

\{\{highlight|linen|Die kollisionsrechtliche Diskussion\}\}, im allgemeinen, und insbesondere die über die Qualifikation, \{\{highlight|linen|hat die verfahrensrechtlichen Aspekte selten beachtet.\}\} Die vorliegende Arbeit versucht falsifizierte Qualifikationstheorien in dieser Hinsicht zu verdeutlichen. Außerdem \{\{highlight|linen|wird der Betrachter auf die im common und civil law gleichermaßen vorzufindende Aktionenstruktur des Zivilrechts verwiesen. Hier\}\} liege also \{\{highlight|linen|eine reale Basis zur vieldiskutierten$[$FN 56$]$ europäischen $[$Rechtsvereinheitlichung.\}\}$]$
----

$[$FN 53$]$ Cf. SAVIGNY, a.a.O. (Fn. 42) und \{\{highlight|lightcyan|V. BAR\}\}, IPR I, Rn 506-513, S. 432-438; FIRSCHING/v. HOFFMANN, IPR5, S. 165-197; \{\{highlight|lightcyan|KEGEL\}\}, IPR7, S. 227-238; \{\{highlight|lightcyan|KOCH / MAGNUS / WINKLER V. MOHRENFELS\}\}, IPR2, S. 8-19; \{\{highlight|lightcyan|KROPHOLLER\}\}, IPR3, §3; LÜDERITZ, IPR2, S. 13-20; \{\{highlight|lightcyan|RAAPE / STURM\}\}, IPR I, S. 98-102.

$[$FN 54$]$ Cf. HARTWIEG, a.a.O. (Fn. 9), RabelsZ 1993, S. 611 m.w.N. (Fn 6).

$[$FN 55$]$ Näher zum englischen pleading und rechtvergleichend HARTWIEG, a.a.O. (Fn. 9), Die Kunst des Sachvortrags im Zivilprozeß.

$[$FN 56$]$ \{\{highlight|linen|Die neue Diskussion geht auf den Aspekt der Aktionenstruktur kaum ein\}\}: GORDLEY, Common law und civil law: eine überholte Unterscheidung, ZEuP 1993, S. 498-518; HARTLIEF, Towards a European Private Law? A Review Essay, MJ 1 (1994), S. 166-178; JAYME, Entwurf eines EG-Familien- und Erbrechtsübereinkommens, IPRax 1994, S. 67-69; DERS., Ein internationales Zivilverfahrensrecht für Gesamteuropa, Heidelberg 1992; JOERGES (Ed.), The Europeanisation of Private Law as a Rationalisation Process and as a Contest of Disciplines, Journal of European Private Law 1995 (special issue); \{\{highlight|lightcyan|KÖTZ\}\}, Was erwartet die Rechtsvergleichung von der Rechtsgeschichte, JZ 1992, S. 20-22; Müller-Graf (Hrsg.), Gemeinsames Privatrecht in der Europäischen Gemeinschaft, Baden-Baden 1993; REIMANN, American Private Law and European Legal Unification --- Can the United States be a Model?, MJ 3 (1996), S. 217-234; REMIEN, Illusion und Realität eines europäischen Privatrechts, JZ 1992, S. 277-284; DERS., Über den Stil des europäischen Privatrechts --- Versuch einer Analyse und Prognose, RabelsZ 60 (1996), S. 1-39; P. \{\{highlight|lightcyan|ULMER\}\}, Vom deutschen zum europäischen Privatrecht? JZ 1992, S. 1-8; \{\{highlight|lightcyan|ZIMMERMANN\}\}, Das römisch-kanonische ius commune als Grundlage europäischer Rechtseinheit; JZ 1992, S. 8-20; DERS., Der europäische Charakter des englischen Rechts --- Historische Verbindungen zwischen civil law und common law, $[$ZEuP 1993, S. 4-51; ders., Konturen eines Europäischen Vertragsrechts, JZ 50 (1995), S. 477-491.$]$}
\end{fragmentpart}
\begin{fragmentpart}{Original \cite[S.~611,~612 Z.~5--20,~104--123,~1--2,~8--10,~114--117]{Hartwieg-1993}}
\enquote{$[$\{\{highlight|linen|In der »freiwilligen Gerichtsbarkeit« (Familien- und Erbrecht) sind nach § 12 FGG von Amts wegen \textsl{alle} Tatsachen zu erforschen und bei der Entscheidung zu berücksichtigen. Hier hat im Tatsächlichen wie im Rechtlichen die »umfassende« kollisionsrechtliche Würdigung \}\} (Artt. 13-26 EGBGB), \{\{highlight|linen|d. h. die Suche nach dem »Sitz des Rechtsverhältnisses«, ihren Platz\}\}$[$FN 5$]$. Zu qualifizieren ist das \textsl{ganze} Rechtsverhältnis.

\{\{highlight|linen|Anders im Vermögensrecht (Schuld- und Sachenrecht gemäß Artt. 27-38 EGBGB) des streitigen Zivilprozesses. Hier ist ein deutsches Gericht in den Rechtsfolgen nach §308 ZPO an die wechselseitigen Anträge der Parteien gebunden und kann dank der Verhandlungsmaxime (vgl. §§ 128, 138, 288, 330 I 1 ZPO) von unstreitigen Tatsachenvorgaben im Tatbestandsbereich des materiellen Rechts ausgehen\}\}$[$FN 6$]$. \{\{highlight|linen|Das Rechtsverständnis des Common Law betont die Parteiherrschaft noch deutlicher im »adversialen« Verfahren. Die Parteien definieren in ihrem\}\} wechselseitigen \{\{highlight|linen|Sachvortrag\}\} (pleading) \{\{highlight|linen|die entscheidungsbedürftigen Elemente des Streits (Tatsachen und Rechtswertungen)\}\}. $[$...$]$ \{\{highlight|linen|Die\}\}

$[$S. 612$]$

\{\{highlight|linen|kollisionsrechtliche Diskussion hat die verfahrensrechtlichen Aspekte selten beachtet.\}\} $[$...$]$

So \{\{highlight|linen|wird der Betrachter auf die im Common Law und Civil Law gleichermaßen vorzufindende Aktionenstruktur des Zivilrechts verwiesen. Hier\}\} liegt \{\{highlight|linen|eine reale Basis zur vieldiskutierten europäischen Rechtsvereinheitlichung\}\}$[$FN 10$]$.
 
$[$FN 5$]$ \{\{highlight|lightcyan|v. Bar\}\}, IPR I S. 432-438 (Rz. 506-513); \{\{highlight|lightcyan|Kegel\}\} S. 185-196 (§ 6); \{\{highlight|lightcyan|Koch/Magnus/Winkler v. Mohrenfels\}\} S. 7-15 (§ 1 II); \{\{highlight|lightcyan|Kropholler\}\} S. 14-21 (§3); \{\{highlight|lightcyan|Leo Raape/Fritz Sturm\}\}, Internationales Privatrecht I: Allgemeine Lehren (München 1977) S. 98-102 (§6). 

$[$FN 6$]$ Der liberale Grundsatz ist vom modernen deutschen Verfahrensrecht modifiziert worden; vgl. einerseits § 139 ZPO und andererseits §§ 273 und 278 ZPO. $[$...$]$

$[$FN 7$]$ Rechtsvergleichend Oskar Hartwieg, Die Kunst des Sachvortrags im Zivilprozeß, Eine rechtsvergleichende Studie zur Arbeitsweise des englischen Pleading-Systems (Heidelberg 1988) 9-U, 25-36, 205-233 (Motive, Texte, Materialien, 48).

$[$...$]$

$[$FN 10$]$ \{\{highlight|linen|Die neue Diskussion geht auf diesen Aspekt kaum ein\}\}; Peter \{\{highlight|lightcyan|Ulmer\}\}, Vom deutschen zum europäischen Privatrecht?: JZ 1992, 1-8; Reinhard \{\{highlight|lightcyan|Zimmermann\}\}, Das römisch-kanonische ius commune als Grundlage europäischer Rechtseinheit, ebd. 8-20; Hein \{\{highlight|lightcyan|Kötz\}\}, Was erwartet die Rechtsvergleichung von der Rechtsgeschichte?, ebd. 20-22.}
\end{fragmentpart}
\begin{fragmentpart}{Anmerkung}
Lm zitiert Hartwieg (Doktorvater) in Fn 54 und 55 als Quelle, lässt aber nicht erkennen, dass er seinen Text in geraffter Form, aber weitgehend wörtlich übernommen hat.

Lm füllt die Fußnote 10 von Hartwieg, derzufolge die neue Diskussion auf den Aspekt der Aktionenstruktur kaum eingehe, von drei Belegen auf vierzehn auf. Die müsste man nachlesen, um festzustellen, welche davon wie weit auf den Aspekt der Aktionenstruktur eingehen. Auf den ersten Blick sieht es so aus, als ob der Verfasser mit den vierzehn Belegen diese von Hartwieg übernommene Aussage stark in Frage stellt.

Fortsetzung von \hyperlink{Lm-Fragment-020-15}{Fragment\_020\_15}
\end{fragmentpart}
\end{fragment}
\phantomsection{}
\belowpdfbookmark{Fragment 24 7--9,~101--102}{Lm-Fragment-024-07}
\hypertarget{Lm-Fragment-024-07}{}
\begin{fragment}
\begin{fragmentpart}{Dissertation S.~24 Z.~7--9,~101--102 (Verschleierung)}
\enquote{Der Jurist, der über einen Rechtsstreit mit Auslandsberührung zu befinden hat, beginnt regelmäßig mit der Suche nach gerichtlicher Zuständigkeit und -
einhergehend mit diesem Ergebnis --- nach dem anwendbaren Kollisionsrecht.$[$FN 1$]$

$[$FN 1$]$ Anderes gilt nach h.M. nur, wenn internationales, materielles Einheitsrecht einschlägig ist. ...}
\end{fragmentpart}
\begin{fragmentpart}{Original \cite[S.~17 Z.~11--12]{Heyn-1986}}
\enquote{Der Jurist, der über eine Angelegenheit mit Auslandsberührung zu befinden hat,
beginnt regelmäßig mit der Suche nach der einschlägigen Kollisionsnorm. $[$FN 1$]$

$[$FN 1$]$ Anderes gilt nur, wenn internationales, materielles Einheitsrecht einschlägig ist, --- s. dazu ...}
\end{fragmentpart}
\begin{fragmentpart}{Anmerkung}
Beide Formulierungen von Heyn werden übernommen. Im Haupttext werden einige Wörter dazwischengeschoben. Heyn wird in diesem Zusammenhang nicht erwähnt. Die bei Heyn direkt anschließenden drei Sätze werden im \hyperlink{Lm-Fragment-025-12}{Fragment\_025\_12} übernommen, zwei davon wörtlich.
\end{fragmentpart}
\end{fragment}
\phantomsection{}
\belowpdfbookmark{Fragment 25 12--16}{Lm-Fragment-025-12}
\hypertarget{Lm-Fragment-025-12}{}
\begin{fragment}
\begin{fragmentpart}{Dissertation S.~25 Z.~12--16 (VerschärftesBauernopfer)}
\enquote{Folglich muß der Jurist den vorgelegten Fall mit Hilfe der Systembegriffe der Verweisungsnormen (ein- oder allseitige Kollisionsnormen) seines IPR erfassen und bewältigen. Dieser Schritt stellt nichts anderes als eine Subsumtion dar.$[$FN 3$]$ Umstritten ist aber, ob diese Subsumtion bereits als Qualifikation bezeichnet werden soll.$[$FN 4$]$

\{$[$FN 4$]$ Die Terminologie verwirrt sich schon hier. Einen kurzen Überblick bietet HEYN, a.a.O. (Fn. 5), S. 17, Fn. 3.\}}
\end{fragmentpart}
\begin{fragmentpart}{Original \cite[S.~17 Z.~12--16]{Heyn-1986}}
\enquote{Dabei muß er den ihm vorgelegten Fall den Systembegriffen der Verweisungsnormen seines IPR zuordnen. 

Dieser Schritt stellt nichts anderes als eine Subsumtion dar.$[$FN 2$]$ Umstritten ist, ob diese Subsumtion bereits als Qualifikation bezeichnet werden soll.$[$FN 3$]$}
\end{fragmentpart}
\begin{fragmentpart}{Anmerkung}
Die Stelle bei Heyn schließt dort unmittelbar an die im \hyperlink{Lm-Fragment-024-07}{Fragment 024 07} übernommene an.
\end{fragmentpart}
\end{fragment}
\phantomsection{}
\belowpdfbookmark{Fragment 25 17--24}{Lm-Fragment-025-17}
\hypertarget{Lm-Fragment-025-17}{}
\begin{fragment}
\begin{fragmentpart}{Dissertation S.~25 Z.~17--24 (VerschärftesBauernopfer)}
\enquote{Nicht nur in der Praxis, sondern auch im Schrifttum, wird die Meinung vertreten, der Umfang der theoretischen Diskussion über die Qualifikation stehe möglicherweise in umgekehrter Relation zu ihrer praktischen Bedeutung.$[$FN 5$]$ Tatsächlich wird bei der Bearbeitung von Fällen mit Auslandsberührung dauernd qualifiziert, häufig unbewußt und deshalb ohne daß davon gesprochen wird, während es der Theorie gerade darauf ankommt, den Schleier des Unbewußten zu lüften und durch allgemeine theoretische Erörterung zu bewirken, daß die Vorgänge richtig behandelt werden.$[$FN 6$]$ Allerdings hat die sprachliche $[$Ungenauigkeit der Begriffe wesentlich dazu beigetragen, daß die Meinungen über das Wesen der Qualifikation bis heute auseinander laufen.$[$FN 7$]$$]$

$[$FN 5$]$ FRAGISTAS, Zur Testamentsform im internationalen Privatrecht, RabelsZ 4 (1930), S. 930-936 (934) hat schon daraufhingewiesen, daß die Lösung von Probleme des IPR manchmal unnötig erschwert wird, weil man unrichtigerweise ein Qualifikationsproblem daraus macht; vgl. auch statt anderen MünchKomm-SONNENBERGER, a.a.O., Rn. 339, Fn. 755.

$[$FN 6$]$ KEGEL, IPR7, § 7 III 3d, S. 259.}
\end{fragmentpart}
\begin{fragmentpart}{Original \cite[S.~144 Z.~339]{Sonnenberger-1990}}
\enquote{Nicht nur in der Praxis, auch im Schrifttum kann man die Meinung finden, die Theorie der Qualifikation stehe möglicherweise in ihrem Umfang in umgekehrter Relation zur praktischen Bedeutung.$[$FN 755$]$ Tatsächlich wird bei der Bearbeitung von IPR-Fällen dauernd qualifiziert, nur häufig unbewußt und deshalb ohne daß davon gesprochen wird, während es der Theorie gerade darauf ankommt, den Schleier des Unbewußten zu lüften und durch allgemeine theoretische Erörterung zu bewirken, daß die Vorgänge richtig behandelt werden.$[$FN 756$]$ Es muß allerdings festgestellt werden, daß die sprachliche Ungenauigkeit des Begriffes leider wesentlich dazu beigetragen hat, daß die Meinungen über das Wesen der Qualifikation bis heute auseinandergehen.

$[$FN 756$]$ So zutreffend KEGEL IPR § 7 III 3 d.}
\end{fragmentpart}
\begin{fragmentpart}{Anmerkung}
Sonnenberger wird in Fußnote 5 an zweiter Stelle und als \textquotedbl{}vgl. auch statt anderen\textquotedbl{} erwähnt. Sonnenbergers Fußnote 756 wird übernommen, aber um die Seitenzahl ergänzt, die Wertung \textquotedbl{}so zutreffend\textquotedbl{} wird weggelassen.
\end{fragmentpart}
\end{fragment}
\phantomsection{}
\belowpdfbookmark{Fragment 26 4--11}{Lm-Fragment-026-04}
\hypertarget{Lm-Fragment-026-04}{}
\begin{fragment}
\begin{fragmentpart}{Dissertation S.~26 Z.~4--11 (BauernOpfer)}
\enquote{Angelehnt an den französischen Sprachgebrauch meint der Begriff „Qualifikation“ die Feststellung der Qualität, Beschaffenheit oder Eigenschaft eines Gegenstandes (was auf deutsch eher „Qualifizierung“ genannt wird oder im naturwissenschaftlichen Sprachgebrauch „Bestimmung der Art“).$[$FN 9$]$ Die \textsl{common law} Juristen sprechen dagegen treffend von \textsl{classification} oder \textsl{characterization}, während \textsl{qualification} bei ihnen meistens eine Einschränkung bezeichnet.$[$FN 10$]$ Gemeint ist also soviel wie Kennzeichnung, Einordnung, Beurteilung oder Subsumtion.
----
$[$FN 9$]$ KROPHOLLER, IPR3, § 14 I 1 mit Nachweisen, insbesondere auf KAHN.

$[$FN 10$]$ Siehe dazu auch MENDELSSOHN BARTHOLDY $[$...$]$}
\end{fragmentpart}
\begin{fragmentpart}{Original \cite[S.~98 Z.~20--27]{Kropholler-1997}}
\enquote{Entsprechend dem französischen Sprachgebrauch meint der Begriff die Feststellung der Qualität, Beschaffenheit oder Eigenschaft eines Gegenstandes (die wir eher \textquotedbl{}Qualifizierung\textquotedbl{} nennen oder im naturwissenschaftlichen Sprachgebrauch \textquotedbl{}Bestimmung der Art\textquotedbl{}). Die anglo-amerikanischen Juristen sprechen treffend von \textquotedbl{}classification\textquotedbl{} oder \textquotedbl{}characterization\textquotedbl{} (während \textquotedbl{}qualification\textquotedbl{} bei ihnen meistens eine Einschränkung bezeichnet). Gemeint ist also soviel wie Kennzeichnung, Beurteilung, Einordnung oder Subsumtion.}
\end{fragmentpart}
\begin{fragmentpart}{Anmerkung}
Die Fußnote verweist auf Kropholler, seltsamerweise auch auf die dort zu findenden Nachweise zu Kahn, die in der untersuchten Arbeit viel zahlreicher und an dieser Stelle auch nicht erforderlich sind, weil der Begriff --- wie Kropholler im Satz zuvor erläutert --- nicht von Kahn stammt, sondern von Bartin.
\end{fragmentpart}
\end{fragment}
\phantomsection{}
\belowpdfbookmark{Fragment 27 6--13}{Lm-Fragment-027-06}
\hypertarget{Lm-Fragment-027-06}{}
\begin{fragment}
\begin{fragmentpart}{Dissertation S.~27 Z.~6--13 (BauernOpfer)}
\enquote{BECKETT wählt den Begriff \textsl{classification}, weil für ihn die Einordnung von Tatsachen, Normen oder Rechten in die kollisionsrechtlichen Kategorien im Vordergrund steht. Da genau dies ein Teil des internationalprivatrechtlichen Fallösungsprozesses darstellt, folgt für ihn, daß Klassifikation in jedem IPR-Fall notwendig und ein fundamentales Problem des IPR ist.$[$FN 12$]$

Der Kanadier FALCONBRIDGE führte den Begriff \textsl{characterization} ein. Im Gegensatz zu BECKETT hat er sich nicht mit einem Aufsatz zur Darlegung seiner Ansicht begnügt, sondern mit einer ganzen Serie von Artikeln.$[$FN 13$]$ 

$[$FN 12$]$ Ibid., S. 46; cf. noch WEBER, a.a.O., S. 167f.

$[$FN 13$]$ Cf. a.a.O. (Fn 11) und noch FALCONBRIDGE, Renvoi, Characterization and Acquired Rights, Can.Bar.Rev. 17 (1939), S. 369-398; DERS., Renvoi and the Law of Domicile § 6 Postscript: Renvoi and Characterization (Besprechung von: CORMACK, Renvoi, Characterization, Localisation and Preliminary Question in the Conflict of Laws), Can.Bar.Rev 19 (1941), S. 334-341; DERS., Conflict Rule and Characterization of Question, Can. Bar.Rev. 30 (1952), S. 118.}
\end{fragmentpart}
\begin{fragmentpart}{Original \cite[S.~167--169 Z.~19--21,~1--4,~25--26,~1--6]{Weber-1986}}
\enquote{$[$S. 167$]$

BECKETT wählt den Begriff \textsl{classification}$[$FN 222$]$, weil für ihn die Einordnung von Tatsachen, Normen oder Rechten (näher auf den Qualifikationsgegenstand geht er nicht ein) in die kollisionsrechtlichen Kategorien im Vorder-

$[$S. 168, Z. 1-4$]$

grund steht.$[$FN 223$]$ Da genau dies ein Teil des internationalprivatrechtlichen Fallösungsprozesses darstellt --- letztlich die Auswahl der Kollisionsnorm -, folgt für BECKETT, daß Klassifikation in jedem IPR-Fall notwendig und ein fundamentales Problem des IPR sei$[$224$]$.

$[$...$]$
 
$[$S. 168, Z. 25-26$]$

Der zweite, wegen der Einführung des Begriffes \textsl{characterization}$[$FN 230$]$ eben schon genannte Autor, der in den dreißiger Jahren in der englischsprachigen 

$[$S. 169$]$ 
Welt das Thema Qualifikation aufgriff, vertieft behandelte und durchsetzen half, war der Kanadier John FALCONBRIDGE$[$FN 231$]$. Im Gegensatz zu BECKETT hat er sich auch nicht mit \textsl{einem} Aufsatz zur Darlegung seiner Ansicht begnügt, sondern konfrontierte sein Publikum mit einer ganzen Salve von Artikeln, in denen er sich der Qualifikation unter immer wieder neuen Aspekten annäherte, für eine immer wieder andere Leserschaft.


$[$FN 222$]$ Ihm schließt sich sogleich an FOSTER, defects, S. 102.

$[$FN 223$]$ BECKETT, Question, S. 46. 

$[$FN 224$]$ S. 46, 81.

$[$...$]$

$[$FN 230$]$ Ich bediene mich der von FALCONBRIDGE gewählten amerikanischen Schreibweise. Manche britischen Autoren (z.B. MORRIS, Conflict, 10. Aufl., S. 31 ff), jedoch nicht alle (vgl. ANTON, S. 43ff), ziehen die Schreibweise \textsl{characterisation} vor.

$[$FN 231$]$ Zu seiner Person s. $[$CULP$]$, S. 892.

$[$FN 232$]$ Neben den im Text und oben S. 201 Genannten: zur Qualifikation noch in FALCONBRIDGE, contratti; renvoi; disorder, S. 786-800; Bespr. Nußbaum, principles; contract, S. 662-666; Bespr. Cormack, renvoi; sowie sein Lehrbuch zum Bank- und Wechselrecht (Banking and Bills of Exchange, 5. Aufl., 1935, S. 869, zitiert nach FALCONBRIDGE, Characterization, S. 239).}
\end{fragmentpart}
\begin{fragmentpart}{Anmerkung}
Weber wird zu BECKETT als \textquotedbl{}cf. noch\textquotedbl{} zitiert, zu FALCONBRIDGE nachfolgend als \textquotedbl{}vgl. auch\textquotedbl{}.
\end{fragmentpart}
\end{fragment}
\phantomsection{}
\belowpdfbookmark{Fragment 28 4--17}{Lm-Fragment-028-04}
\hypertarget{Lm-Fragment-028-04}{}
\begin{fragment}
\begin{fragmentpart}{Dissertation S.~28 Z.~4--17 (BauernOpfer)}
\enquote{Ihm genügen die verschiedenen Stufen der juristischen Arbeit unter einem gemeinsamen Oberbegriff \textquotedbl{}Qualifikation\textquotedbl{} (als \textsl{primary / secondary characterization / classification}, etc.) nicht; statt dessen unterscheidet er schon terminologisch die Verschiedenheit dieser Stufen durch je eigene Benennungen. Dies ist um so interessanter als er für seine Benennungen sowohl die \textsl{classification} als auch die \textsl{characterization} einsetzt, jeweils für eine andere Stufe.$[$FN 18$]$ Er bringt damit zum Ausdruck, daß diese beiden Begriffe keine Synonyme darstellen, sondern beide in ihren von einander abweichenden Bedeutungen verschiedene Rollen in der internationalprivatrechtlichen Fallösung spielen.$[$FN 19$]$

So bezeichnet \textsl{characterization} für Unger die Subsumtion von Anknüpfungsgegenständen unter die in einer Kollisionsnorm genannten Begriffe. \textsl{Classification} dient dagegen als Abgrenzung der aus dem berufenen Recht anzuwendenden Teile seiner Rechtsnormen insgesamt.$[$FN 20$]$ 
----
$[$FN 18$]$ UNGER, a.a.O. (Fn. 16), S. 16f.

$[$FN 19$]$ Ibid, S. 21.

$[$FN 20$]$ Cf. aus deutscher Sicht, wenn auch nicht sehr deutlich: WEBER, a.a.O., S. 209ff. Die Unterscheidung zwischen \textquotedbl{}characterization\textquotedbl{} und \textquotedbl{}classification aufgrund unterschiedlicher Funktionen ist zum ersten Mal von DESPAGNET (Des conflits de lois relatifs à la qualification des rapports juridiques, Clunet 25 (1898), S. 253-273) im Kontext der Unterscheidung von Mobilien und Immobilien aufgegriffen worden: \textquotedbl{}Mais y a-t-il vraiment lieu de rapprocher cette question de celle qui nous occupé $[$sic$]$? La \textsl{classification} des biens, la détermination de leur caractère mobilier ou immobilier, même leur répartition en biens dans le commerce ou hors de commerce, ne sont-elles pas des questions d'ordre public territorial $[$...$]$. D' ailleurs, ces questions sont-elles bien relatives à la \textsl{qualification} des rapports juridiques, et ne se réfèrent-elles pas plutôt à la qualification des choses?\textquotedbl{} (S. 255f.)}
\end{fragmentpart}
\begin{fragmentpart}{Original \cite[S.~171 Z.~7--21]{Weber-1986}}
\enquote{Neu daran, und von den übrigen englischsprachigen Autoren seiner Zeit abweichend, ist, daß er nun nicht wieder mehrere dieser Stufen unter einen gemeinsamen Oberbegriff 'Qualifikation' stellt (als \textsl{primary/secondary characterization/classification}, etc.), sondern daß er schon terminologisch die Verschiedenheit dieser Stufen durch je eigene Benennungen zum Ausdruck bringt. Dies ist um so interessanter als er für die Benennungen
sowohl das Wort \textsl{characterization} als auch das Wort \textsl{classification} verwendet, jeweils für eine andere Stufe$[$FN 252$]$. Er bringt damit klar zum Ausdruck, daß diese beiden Begriffe keine Synonyme sind und daß dennoch beide in ihren von einander abweichenden Bedeutungen eine Rolle im internationalprivatrechtlichen Fallösungsprozeß spielen$[$FN 253$]$.

\textsl{Characterization} ist für UNGER die Subsumtion unter die die Anknüpfungsgegenstände der Kollisionsnorm bezeichnenden Begriffe, \textsl{Classification} (begrifflich weniger überzeugend) die Abgrenzung der aus dem berufenen Recht anzuwendenden Teile seiner Rechtsnormen$[$FN 254$]$.
----
$[$FN 252$]$ UNGER, Place, S. 16f.

$[$FN 253$]$ S. 21.

$[$FN 254$]$ Ausführlich dazu unten S. 209ff.}
\end{fragmentpart}
\begin{fragmentpart}{Anmerkung}
Weber wird in der vorausgehenden Fußnote 17 dafür erwähnt, dass er Unger \textquotedbl{}kurz analysiert\textquotedbl{}.
\end{fragmentpart}
\end{fragment}
\phantomsection{}
\belowpdfbookmark{Fragment 30 5--16}{Lm-Fragment-030-05}
\hypertarget{Lm-Fragment-030-05}{}
\begin{fragment}
\begin{fragmentpart}{Dissertation S.~30 Z.~5--16 (BauernOpfer)}
\enquote{Unter Qualifikation im internationalen Privatrecht versteht man im allgemeinen die Subsumtion unter den Tatbestand einer geschriebenen oder gewohnheitsrechtlich geltenden Kollisionsnorm. Hierbei kommt es auf den „Charakter“ des zu subsumierenden Gegenstandes an. Es geht um die Fragen des sachlichen Anwendungsbereichs und der Auslegung der Kollisionsnormen. Dieses spezifische Problem der Qualifikation (Subsumtion oder Auslegung) ergibt sich daraus, daß das IPR alle für die einschlägigen staatlichen Sachrechte in Betracht kommenden Sachverhalte erfassen und dabei handhabbar und übersichtlich eine Rechtsanwendungsbestimmung (Rechtsfolge) definieren muß. Der durch diese Aufgabe bedingte \textsl{hohe Abstraktionsgrad des IPR} erfordert die Verwendung weiter Sammelbegriffe in den Kollisionsnormen. Größere Schwierigkeiten als das Sachrecht bereitet die Subsumtion.$[$FN 29$]$
----
$[$FN 29$]$ So mit vielen Beispielen MAREDAKIS, IPR2 Bd. I, Athen 1967, S. 209, 211, 213, 215 (gr.), KROPHOLLER, IPR3, § 14 I 2.}
\end{fragmentpart}
\begin{fragmentpart}{Original \cite[S.~99 Z.~1--13]{Kropholler-1997}}
\enquote{Im IPR versteht man unter Qualifikation die Subsumtion unter den Tatbestand einer (geschriebenen oder gewohnheitsrechtlich geltenden) Kollisionsnorm. Hierbei kommt es auf den „Charakter“ des zu subsumierenden Gegenstandes an. Schlicht gesagt geht es um die Frage des sachlichen Anwendungsbereichs der Kollisionsnormen.

2. Das \textsl{spezifische Problem} der Qualifikation ergibt sich daraus, daß das IPR alle in den verschiedenen staatlichen Sachrechten widergespiegelten Sachverhalte erfassen und dabei doch handhabbar und übersichtlich bleiben muß. Der durch diese Aufgabe bedingte hohe Abstraktionsgrad des IPR erfordert die Verwendung weiter Sammelbegriffe in den Kollisionsnormen, wie z. B. „Geschäftsfähigkeit“ (Art. 7 EGBGB), „Form von Rechtsgeschäften“ (Art. 11), „Voraussetzungen der Eheschließung“ (Art. 13). Entsprechend größere Schwierigkeiten als im Sachrecht bereitet die Subsumtion.}
\end{fragmentpart}
\begin{fragmentpart}{Anmerkung}
Alternativ könnte dieses Fragment als \textquotedbl{}Verschärftes Bauernopfer\textquotedbl{} gewertet werden, weil Kropholler in Fußnote 29 erst an zweiter Stelle genannt wird.
\end{fragmentpart}
\end{fragment}
\phantomsection{}
\belowpdfbookmark{Fragment 30--31 18--20,~1--2}{Lm-Fragment-030-18}
\hypertarget{Lm-Fragment-030-18}{}
\begin{fragment}
\begin{fragmentpart}{Dissertation S.~30--31 Z.~18--20,~1--2 (BauernOpfer)}
\enquote{Die \textsl{Systembegriffe}, die wir in den Kollisionsnormen des IPR antreffen, sind nach den Zielen auszulegen, die die Kollisionsnormen verfolgen. Die Interessen, denen die Kollisionsnormen dienen, sind zu ermitteln. Das $[$internationale Privatrecht beruht auf internationalprivatrechtlicher Gerechtigkeit,die sich nach internationalprivatrechtlichen Interessen bestimmt.$[$FN 30$]$
----
$[$FN 30$]$ Cf. KEGEL, IPR7, § 7 III 3 a, S. 254 mit Verweis auf § 2, S. 106-119; KROPHOLLER, IPR3, § 4 I.}
\end{fragmentpart}
\begin{fragmentpart}{Original \cite[S.~254 Z.~3--7]{Kegel-1995}}
\enquote{Die Systembegriffe, die wir in den Kollisionsnormen des IPR antreffen, sind auszulegen nach den Zielen, die die Kollisionsnormen verfolgen. Die Interessen sind zu ermitteln, denen die Kollisionsnormen dienen. Das IPR beruht auf \textsl{internationalprivatrechtlicher Gerechtigkeit}, die zwischen einer Reihe \textsl{internationalprivatrechtlicher Interessen} wählt (oben § 2).}
\end{fragmentpart}
\begin{fragmentpart}{Anmerkung}
Die Fußnote 30 verweist an erster Stelle auf Kegel, lässt aber nicht erkennen, dass diese Formulierungen von ihm stammen. Die Sätze sind identisch und unterscheiden sich lediglich in der Reihenfolge der Satzteile.
\end{fragmentpart}
\end{fragment}
\phantomsection{}
\belowpdfbookmark{Fragment 31 4--7,~108--109}{Lm-Fragment-031-04}
\hypertarget{Lm-Fragment-031-04}{}
\begin{fragment}
\begin{fragmentpart}{Dissertation S.~31 Z.~4--7,~108--109 (VerschärftesBauernopfer)}
\enquote{Wenn RABEL die Parole from \textsl{characterization to interpretation}$[$FN 32$]$ ausgab, so wollte er damit wohl nicht diesen Zusammenhang leugnen, sondern von einer nur logischen Einordnung zugunsten einer teleologischen Betrachtung abraten.

$[$FN 32$]$ RABEL, The Conflict of Laws. A Comparative Study, Bd. I, Ann Arbor 1958, S. 50, bemerkt, man sollte den Begriff der Qualifikation durch den Begriff der Auslegung ersetzen („Emphasis should be shifted from 'characterization' to 'interpretation'...“). von BAR, IPR I, und KROPHOLLER, IPR3, nehmen diese Meinung an, während RAAPE/ STURM, IPR I, S. 25 8f., 275ff, die Zusammengehörigkeit beider Aspekte verdunkelt.}
\end{fragmentpart}
\begin{fragmentpart}{Original \cite[S.~99 Z.~101--107]{Kropholler-1997}}
\enquote{$[$FN 4$]$ Wenn \textsl{Rabel}, The Conflict of Laws I (1945) 44 (= 2. Aufl. $[$1958$]$ 50), die Parole „from characterization to interprétation“ ausgab, so wollte er damit wohl nicht diesen Zusammenhang leugnen, sondern nur von einer rein logischen Einordnung abraten zugunsten einer mehr teleologischen Betrachtung. Die Unterscheidung von \textsl{Raape/Sturm} I 258f., 275ff. zwischen Abgrenzung und Qualifikation $[$...$]$ verdunkelt die Zusammengehörigkeit beider Aspekte.}
\end{fragmentpart}
\begin{fragmentpart}{Anmerkung}
Fußnote 32 verweist auf Krophollers Buch (ohne Angabe der Seitenzahl) an dritter von vier Stellen, und auch nur dafür, dass er Rabels Meinung angenommen habe. Alternativ ist eine Einordnung als \textquotedbl{}Verschleierung\textquotedbl{} möglich.
\end{fragmentpart}
\end{fragment}
\phantomsection{}
\belowpdfbookmark{Fragment 37--38 1--3}{Lm-Fragment-038-01}
\hypertarget{Lm-Fragment-038-01}{}
\begin{fragment}
\begin{fragmentpart}{Dissertation S.~37--38 Z.~1--3 (BauernOpfer)}
\enquote{$[$Die umfangreiche$]$ 143-seitige Arbeit besteht aus drei Hauptteilen, von denen jeder einer der drei von KAHN erstmals definierten Klassen von Gesetzeskollisionen gewidmet ist, ferner aus je einer kurzen Einleitung und Schlußbetrachtung.}
\end{fragmentpart}
\begin{fragmentpart}{Original \cite[S.~24 Z.~5--8]{Weber-1986}}
\enquote{Die umfangreiche Arbeit – in Jherings Jahrbüchern 143 Seiten – besteht aus drei Hauptteilen$[$5$]$, von denen jeder einer der drei von KAHN ausgemachten Klassen von Gesetzeskollisionen gewidmet ist$[$6$]$, ferner aus je einer kurzen Einleitung und Schlußbetrachtung.}
\end{fragmentpart}
\begin{fragmentpart}{Anmerkung}
Fortsetzung von \hyperlink{Lm-Fragment-037-08}{Lm/Fragment\_037\_08}
\end{fragmentpart}
\end{fragment}
\phantomsection{}
\belowpdfbookmark{Fragment 37 7--9}{Lm-Fragment-037-08}
\hypertarget{Lm-Fragment-037-08}{}
\begin{fragment}
\begin{fragmentpart}{Dissertation S.~37 Z.~7--9 (BauernOpfer)}
\enquote{a.  FRANZ KAHN$[$FN 23$]$

Im Jahre 1891 entstand Franz Michael KAHNs erste – und vielleicht wichtigste – große Abhandlung zum IPR, die \textsl{Gesetzeskollisionen}.$[$FN 24$]$ Die umfangreiche $[$143-seitige Arbeit besteht aus drei Hauptteilen, von denen jeder einer der drei von KAHN erstmals definierten Klassen von Gesetzeskollisionen gewidmet ist, ferner aus je einer kurzen Einleitung und Schlußbetrachtung.$]$

$[$FN 23$]$ Über KAHNs Leben, Werk und Wirkung, cf. WEBER, a.a.O., S. 14-23; EVRIGENIS, Les classiques de droit international privé: \textsl{Franz Kahn}. A propos du cinquantenaire de sa mort (1904-1954), Internationales recht und Diplomatie, Hamburg 1957, S. 301-314 = EEAN 72/73 (1954, griechisch), S. 93 ; TSOUCA, Anazitontas tis rizes tou synchronou idiotikou diethnous dikaiou. O diachronikos charaktiras tou ergou tou Franz Kahn (Auf der Suche nach den Wurzeln des modernen IPR. Der überzeitliche Charakter des Werks Kahns), Armenopoulos- Jahrbuch der Anwaltskammer von Thessaloniki 13 (1992), S. 169-183.

$[$FN 24$]$ Vollständiger Titel: Gesetzeskollisionen. Ein Beitrag zur Lehre des internationalen Privatrechts, Jherings Jb 30 (1891), S. 1-143. Vgl. VON BAR, IPR I, Rn. 210, S. 488; BARA$[$ZETTI, B esprechung von: Kahn, Gesetzeskollisionen, ZIR 1 (1891), S. 424-428, 726-731; HEYN, a.a.O., S. 18-22; MARIDAKIS, IPR2, S. 232f. (Fn. 64), MORRIS, Conflict of Laws4,, S. 416f; NEUNER, Der Sinn …, S.11-17; NIEDERER, Die Frage der Qualifikation, S 21-24; RABEL, Conflict of Laws I, S. 53-66; DERS., RabelsZ 1931, 243ff.; ROBERTSON, Characterization, S. 26f., 33, 60, 95; VON STEIGER, a.a.O., s. 26-36; VRELLIS, IPR, S. 46f.$]$}
\end{fragmentpart}
\begin{fragmentpart}{Original \cite[S.~24 Z.~3--9]{Weber-1986}}
\enquote{In den Jahren 1890 und 1891$[$FN 2$]$ entstand also KAHNS erste – und im Rückblick vielleicht wichtigste$[$FN 3$]$ – große Abhandlung zum Internationalen Privatrecht, die \textsl{Gesetzeskollisionen}$[$FN 4$]$.

Die umfangreiche Arbeit – in Jherings Jahrbüchern 143 Seiten – besteht aus drei Hauptteilen$[$FN 5$]$, von denen jeder einer der drei von KAHN ausgemachten Klassen von Gesetzeskollisionen gewidmet ist$[$FN 6$]$, ferner aus je einer kurzen Einleitung und Schlußbetrachtung.

$[$FN 4$]$ Vollständiger Titel: „Gesetzeskollisionen. Ein Beitrag zur Lehre des internationalen Privatrechts“.}
\end{fragmentpart}
\begin{fragmentpart}{Anmerkung}
Die Quelle wird eingangs in einem weiterführenden Verweis zum Werk und Wirkung von Kahn genannt.
\end{fragmentpart}
\end{fragment}
\phantomsection{}
\belowpdfbookmark{Fragment 37--38 8--9}{Lm-Fragment-038-08}
\hypertarget{Lm-Fragment-038-08}{}
\begin{fragment}
\begin{fragmentpart}{Dissertation S.~37--38 Z.~8--9 (Verschleierung)}
\enquote{Die Schlußfolgerung KAHNs stimmt in verblüffender Weise mit dem überein, was BARTIN sechs Jahre später sogar als Titel seiner Abhandlung wählte:}
\end{fragmentpart}
\begin{fragmentpart}{Original \cite[S.~31 Z.~16--17]{Weber-1986}}
\enquote{Die \textsl{conclusio} KAHNs stimmt in verblüffender Weise mit dem überein, was BARTIN einige Jahre später sogar als Titel seiner Abhandlung wählte:}
\end{fragmentpart}
\begin{fragmentpart}{Anmerkung}
Fortsetzung (unterbrochen durch ein Zitat von Bartin) in \hyperlink{Lm-Fragment-039-08}{Fragment\_039\_08}.

Weber wird zuletzt in Fußnote 25 genannt, die sich auf den letzten Abschnitt bezieht. Das Fragment könnte alternativ als Bauernopfer eingeordnet werden.
\end{fragmentpart}
\end{fragment}
\phantomsection{}
\belowpdfbookmark{Fragment 38 112--115}{Lm-Fragment-038-112}
\hypertarget{Lm-Fragment-038-112}{}
\begin{fragment}
\begin{fragmentpart}{Dissertation S.~38 Z.~112--115 (Verschleierung)}
\enquote{$[$FN 26$]$ KAHN, ibid., S. 56-106. Er versteht darunter die möglichen Konfliktsfälle, die entstehen können, wenn Anknüpfungspunkte von Kollisionsnormen ihren Erwerbs- und Verlustvoraussetzungen bzw. ihrem Inhalt nach, nicht übernational einheitlich feststehen, sondern ihrerseits wieder Unterschiede aufweisen. $[$...$]$ Cf. ferner WEBER, a.a.O., S. 26-28.}
\end{fragmentpart}
\begin{fragmentpart}{Original \cite[S.~26--27 Z.~16--18,~1--2]{Weber-1986}}
\enquote{$[$Seite 26$]$

Er versteht darunter die möglichen Konfliktfälle, die daraus entstehen können, daß die Anknüpfungspunkte$[$FN 23$]$ der Kollisionsnormen $[$...$]$ ihren Erwerbs- und
Verlustvoraussetzungen, bzw. ihrem Inhalt nach, nicht übernational

$[$Seite 27$]$

einheitlich feststehen, sondern ihrerseits wieder nationale Unterschiede aufweisen.}
\end{fragmentpart}
\begin{fragmentpart}{Anmerkung}
Wortwörtliche Übereinstimmung, ohne dass dies durch die abschließende Quellenangabe gekennzeichnet wurde.
\end{fragmentpart}
\end{fragment}
\phantomsection{}
\belowpdfbookmark{Fragment 39 8--14}{Lm-Fragment-039-08}
\hypertarget{Lm-Fragment-039-08}{}
\begin{fragment}
\begin{fragmentpart}{Dissertation S.~39 Z.~8--14 (Verschleierung)}
\enquote{Oder, aus heutiger Sicht zusammengefaßt: Die vollständige Beseitigung von Gesetzeskollisionen scheint unmöglich zu sein.$[$32$]$ Solange es kein universell übergreifendes Sachrecht gibt, das jedes IPR ohnehin funktionslos machen würde, wird es immer Qualifikationskonflikte geben.

Welche Lösung bietet KAHN an? KAHN, als Hauptvertreter der positivistischen nationalistischen Richtung, spricht sich dafür aus, nach der lex fori zu qualifizieren.

$[$32$]$ So auch der Aufsatz von BARTIN: De l'impossibilité d'arriver à la suppression définitive des conflits des lois, Clunet 4 ( 1897), S. 225-255, 466-495, 720-738.}
\end{fragmentpart}
\begin{fragmentpart}{Original \cite[S.~31 Z.~16--23]{Weber-1986}}
\enquote{Die conclusio KAHNs stimmt in verblüffender Weise mit dem überein, was BARTIN einige Jahre später sogar als Titel seiner Abhandlung wählte: Die Unmöglichkeit der vollständigen Beseitigung von Gesetzeskollisionen$[$61$]$. Solange es kein Welt(sach)recht gebe, das ohnehin zugleich jedes IPR funktionslos machen würde, werde es immer Qualifikationskonflikte geben$[$62$]$. 

Wie also sind diese zu lösen? KAHN spricht sich --- für einen Hauptvertreter der positivistisch/nationalistischen Richtung nicht verwunderlich --- dafür aus (wie man es später nennen sollte), nach der lex fori zu qualifizieren:

$[$61$]$ BARTIN: De l’impossibilité d’arriver à la suppression définitive des conflits de lois.

$[$62$]$ KAHN: „$[$Die dritte Konfliktskategorie$]$ ist \textsl{notwendig} bedingt durch die Existenz tiefer Verschiedenheiten innerhalb der einzelnen Privatrechtsordnungen.“ S. J 142f = A 122 (meine Hervorhebung).}
\end{fragmentpart}
\begin{fragmentpart}{Anmerkung}
Anschluss an \hyperlink{Lm-Fragment-038-08}{Fragment\_038\_08}, unterbrochen durch ein Zitat aus dem Werk von Bartin. Weber wird zuletzt in Fußnote 25 auf der Vorseite erwähnt; eine alternative Einordnung als Bauernopfer ist denkbar.
\end{fragmentpart}
\end{fragment}
\phantomsection{}
\belowpdfbookmark{Fragment 40 2--15}{Lm-Fragment-040-02}
\hypertarget{Lm-Fragment-040-02}{}
\begin{fragment}
\begin{fragmentpart}{Dissertation S.~40 Z.~2--15 (Verschleierung)}
\enquote{Um noch einmal KAHN zu Wort kommen zu lassen:
:\textquotedbl{}Nicht ob die fremden Rechtsregeln herrschen wollen haben wir zu untersuchen, sondern ob sie herrschen \textsl{sollen} $[$...$]$ Das fremde Recht \textsl{soll} angewendet werden, wenn \textsl{unser Gesetzgeber will}, daß es angewendet werde. Unsere Kollisionsnorm entscheidet also über die Anwendung des fremden Rechts. Maßgebliches Kriterium für diese Anwendung ist die Identität der eigenen mit den fremden Rechtsinstituten eigentlich das Erfordernis der \textsl{funktionalen} Identität der in Betracht kommenden Rechtsinstitute.\textquotedbl{}$[$FN 35$]$
Seine Erklärung zur Anwendung ausländischen Rechts geht noch weiter. Bei fremden Rechtsinstituten zieht er die Grenze zunächst eng: Die ausländischen Rechtsnormen sollen in solchen Fällen ohne weiteres unanwendbar und nur im Bereich von Vorfragen beachtlich sein.$[$FN 36$]$
----
$[$FN 35$]$ KAHN, Jherings Jb. 1891, S. 129f.

$[$FN 36$]$ Ibid, S. 133-136, 141-143.}
\end{fragmentpart}
\begin{fragmentpart}{Original \cite[S.~31,~32 Z.~24--29,~1--7]{Weber-1986}}
\enquote{„Nicht ob die fremden Rechtsregeln herrschen \textsl{wollen} haben wir zu untersuchen, sondern ob sie herrschen \textsl{sollen} $[$. . .$]$ Das fremde Recht \textsl{soll} angewendet werden, wenn \textsl{unser Gesetzgeber will}, daß es angewendet werde.“$[$FN 63$]$ Unsere Kollisionsnorm entscheidet also über die Anwendung des fremden Rechts$[$FN 64$]$, aber wie tut sie das? In einer Formulierung, die bereits weit über das hinausging, was in den Jahren nach KAHN die streng nationalistische Schule kennzeichnete, betont er das Erfordernis der \textsl{funktionalen} Identität der eigenen mit den fremden Rechtsinstituten als maßgebliches Kriterium$[$Fn 65$]$.

Die Möglichkeiten dieses Ansatzes hat KAHN 1890 allerdings noch nicht ausgeschöpft. Bei den fremden Instituten$[$66$]$ zieht er die Grenze zunächst noch$[$FN 67$]$ vorsichtig und eng: die fremden Rechtsregeln sollen in solchen Fällen
ohne weiteres unanwendbar$[$FN 68$]$ und nur im Bereich von Vorfragen beachtlich sein$[$69$]$.
----
$[$FN 63$]$ S. J 129f = A 111 (Hervorhebungen im Original).

$[$FN 64$]$ S.J 131-133 = A 113f.

$[$FN 65$]$ S. J 115 = A 99.

$[$FN 66$]$ Diese Bezeichnung sei als Kürzel für die o. S. 31 genannten Fälle (Polygamie, etc.) im weiteren verwandt. KAHN spricht von „Rechtsregeln, welche mit den unsrigen inkommensurabel sind“, S. J 142 = A 122.

$[$FN 67$]$ Zu KAHNs späterer Ansicht u. S. 72, zum Ganzen ferner u. S. 235 ff.

$[$FN 68$]$ S. J 133-136 = A 114-117.

$[$FN 69$]$ S.J 137f = A 117f.}
\end{fragmentpart}
\begin{fragmentpart}{Anmerkung}
Die dem Zitat aus KAHN nachfolgenden Erläuterungen in der Quelle Weber werden leicht verändert, aber dennoch als Teil des Zitates aus KAHN dargestellt. Dem Verfasser fällt dabei nicht auf, dass die \textquotedbl{}funktionale Identität\textquotedbl{} von Rechtsnormen erst Jahrzehnte später thematisiert wird.  Die Quelle Weber wird in diesem Kontext nicht erwähnt.
\end{fragmentpart}
\end{fragment}
\phantomsection{}
\belowpdfbookmark{Fragment 42 17--20}{Lm-Fragment-042-17}
\hypertarget{Lm-Fragment-042-17}{}
\begin{fragment}
\begin{fragmentpart}{Dissertation S.~42 Z.~17--20 (Verschleierung)}
\enquote{KAHNS Aufsatz blieb in Frankreich ohne Echo. So konnte Etienne BARTIN das
Qualifikationsproblem sechs Jahre später noch einmal entdecken. In einem später
vieldiskutierten Aufsatz im Clunet --- Journal du Droit International$[$FN 45$]$ hat
BARTIN --- unabhängig von KAHN und ohne dessen Gesetzeskollisionen zu ken$[$nen,$[$FN 46$]$ aber oft mit verblüffender Ähnlichkeit in Gedankenfolge und vorgeschlagenen Lösungswegen --- das Problem erneut abgehandelt und ihm auch seinen eigentlichen Namen gegeben.$]$

$[$FN 45$]$ BARTIN, a.a.O., Kapitel I, Fn. 4. Cf. auch von BAR, IPR I, Rn. 210, S. 487; BATIFFOL / LAGARDE, DIP8, Rn. 241, S. 402-404; BRIERLY, Besprechung von Bartin, La doctrine des qualifications, B.Yb.I.L. 13 (1932), S. 195; KEGEL, IPR7, S. 155, 247; MARIDAKIS, IPR I2, S.
232-234, 238f.; MORRIS, Conflict of Laws44, S. 416f.; NEUNER, Der Sinn ..., S. 10-19; NIDERER, Die Frage der Qualifikation, S. 21-23, 26-28; RABEL, Conflict of Laws I, S. 53-66; DERS., RabelsZ 1931, S. 46-54; ROBERTSON, Characterization, S. 25f, 33, 35-38, 41, 59f., 69-72, 115f., 133, 136, 158-168, 235-238; VON STEIGER, a.a.O., S. 26-36; VRELLIS, IPR, S. 47-49; WEBER, a.a.O., S. 38-77.

$[$$[$FN 46$]$ Bartin hat in einer Fußnote zum Nachdruck seines Aufsatzes diese Unabhängigkeit betont und damit zugleich die Priorität Kahns anerkannt. Bartin, Études de droit international privé, Paris 1899, S. 1-82, (S. 2, Anm. 2)$]$}
\end{fragmentpart}
\begin{fragmentpart}{Original \cite[S.~4 Z.~4--11]{Weber-1986}}
\enquote{KAHNS Aufsatz blieb in Frankreich zunächst ohne Echo$[$FN 9$]$ und so konnte noch sechs Jahre später Etienne BARTIN das Qualifikationsproblem neu entdecken:

In einem der späterhin meistdiskutierten Aufsätze$[$FN 10$]$ in Clunets Journal$[$FN 11$]$ hat BARTIN --- unabhängig von KAHN und ohne dessen \textsl{Gesetzeskollisionen} zu kennen$[$FN 12$]$, aber mit oft verblüffender Ähnlichkeit der Gedankenfolge und
des vorgeschlagenen Lösungsweges$[$FN 13$]$ --- das Problem abgehandelt und ihm auch seinen Namen gegeben: \textsl{Qualifikation}$[$FN 14$]$.

$[$FN 12$]$ Diese Unabhängigkeit hat Bartin später in einer Fußnote zum Nachdruck seines Aufsatzes
betont und damit zugleich die Priorität Kahns anerkannt: Bartin, Qualifications, S. E 2 = W 346; s. a. u. S. 41.}
\end{fragmentpart}
\begin{fragmentpart}{Anmerkung}
Beginn von \hyperlink{Lm-Fragment-043-01}{Lm/Fragment\_043\_01}.

Die Quelle wir in FN 45 zwar genannt, aber nur als eine von vielen Quellen. Außerdem wird \textquotedbl{}S. 38-77\textquotedbl{} angegeben und nicht S 4, von wo die fast wörtliche Übernahme tatsächlich stammt. Auch FN 46 ist sehr ähnlich in der Quelle zu finden.
\end{fragmentpart}
\end{fragment}
\phantomsection{}
\belowpdfbookmark{Fragment 43 1--3,~101--103}{Lm-Fragment-043-01}
\hypertarget{Lm-Fragment-043-01}{}
\begin{fragment}
\begin{fragmentpart}{Dissertation S.~43 Z.~1--3,~101--103 (Verschleierung)}
\enquote{$[$In einem später vieldiskutierten Aufsatz im Clunet --- Journal du Droit International$[$FN 45$]$ hat BARTIN --- unabhängig von KAHN und ohne dessen Gesetzeskollisionen zu ken$]$nen,$[$FN 46$]$ aber oft mit verblüffender Ähnlichkeit in Gedankenfolge und vorgeschlagenen Lösungswegen --- das Problem erneut abgehandelt und ihm auch seinen eigentlichen Namen gegeben. Die Bezeichnung „\textsl{Qualifikationsproblem}“ bzw. „\textsl{Qualifikationstheorie}“ $[$...$]$

$[$FN 45$]$ BARTIN, a.a.O., Kapitel I, Fn. 4. Cf. auch von BAR, IPR I, Rn. 210, S. 487; BATIFFOL / LAGARDE, DIP8, Rn. 241, S. 402-404; BRIERLY, Besprechung von Bartin, La doctrine des qualifications, B.Yb.I.L. 13 (1932), S. 195; KEGEL, IPR7, S. 155, 247; MARIDAKIS, IPR I2, S.
232-234, 238f.; MORRIS, Conflict of Laws4, S. 416f.; NEUNER, Der Sinn ..., S. 10-19; NIEDERER, Die Frage der Qualifikation, S. 21-23, 26-28; RABEL, Conflict of Laws I, S. 53-66; DERS., RabelsZ 1931, S. 46-54; ROBERTSON, Characterization, S. 25f, 33, 35-38, 41, 59f., 69-72, 115f., 133, 136, 158-168, 235-238; VON STEIGER, a.a.O., S. 26-36; VRELLIS, IPR, S.
47-49; WEBER, a.a.O., S. 38-77.

$[$FN 46$]$ BARTIN hat in einer Fußnote zum Nachdruck seines Aufsatzes diese Unabhängigkeit betont und damit zugleich die Priorität KAHNS anerkannt. BARTIN, Études de droit international privé, Paris 1899, S. 1-82, (S. 2, Anm. 2)}
\end{fragmentpart}
\begin{fragmentpart}{Original \cite[S.~4 Z.~7--11,~113--115]{Weber-1986}}
\enquote{In einem der späterhin meistdiskutierten Aufsätze$[$FN 10$]$ in Clunets Journal$[$FN 11$]$ hat BARTIN --- unabhängig von KAHN und ohne dessen \textsl{Gesetzeskollisionen} zu
kennen$[$FN 12$]$, aber mit oft verblüffender Ähnlichkeit der Gedankenfolge und
des vorgeschlagenen Lösungsweges$[$FN 13$]$ --- das Problem abgehandelt und ihm
auch seinen Namen gegeben: \textsl{Qualifikation}$[$FN 14$]$.

$[$FN 12$]$ Diese Unabhängigkeit hat BARTIN später in einer Fußnote zum Nachdruck seines Aufsatzes betont und damit zugleich die Priorität KAHNS anerkannt: BARTIN, Qualifications, S. E 2 = W 346; s. a. u. S. 41.}
\end{fragmentpart}
\begin{fragmentpart}{Anmerkung}
Haupttext und Fußnote 46 stimmen fast wörtlich mit Weber (1986) überein. Die vom Verfasser in der vorangegangenen Fußnote angegebene Stelle bei Weber hat offensichtlich nichts mit der hier übernommenen Passage zu tun und erfolgt in einer Reihe mit den grundlegenden Primärquellen zum Thema.
\end{fragmentpart}
\end{fragment}
\phantomsection{}
\belowpdfbookmark{Fragment 43 7--14}{Lm-Fragment-043-07}
\hypertarget{Lm-Fragment-043-07}{}
\begin{fragment}
\begin{fragmentpart}{Dissertation S.~43 Z.~7--14 (Verschleierung)}
\enquote{BARTINs 1897 in drei Teilen erschienene Abhandlung zur Qualifikation umfaßt 80 Seiten. Diese Arbeit ist 1899 zusammen mit anderen Schriften des gleichen Autors unter dem Titel \textsl{„La théorie des qualifications en droit international privé“ erschienen.} Sie ist damit kürzer als KAHNs \textsl{Gesetzeskollisionen} (143 Seiten) und befaßt sich fast ausschließlich mit latenten Kollisionen. Schon die Überschrift enthält das Programm und die zentrale Aussage der Abhandlung: „Von der Unmöglichkeit, die Gesetzeskollisionen jemals endgültig beseitigen zu können“.$[$FN 47$]$

$[$FN 47$]$ BARTIN, De l’impossibilité d’arriver à la suppression définitive des conflits de lois, Clunet 24 (1897), S. 225-255, 466-495, 720-738 = La théorie des qualifications en droit international privé, in BARTIN, Études de droit international privé, Paris 1899, S. 1-82 = (Teilabdruck) in $[$PICONE/WENGLER$]$, Internationales Privatrecht, 1974, S. 345-374 ; vgl. DERS., La doctrine des qualifications et ses rapports avec le caractère national des règles du conflit des lois, RCADI 32 (1930-I), S. 561-621.}
\end{fragmentpart}
\begin{fragmentpart}{Original \cite[S.~40 Z.~9--16]{Weber-1986}}
\enquote{BARTINS 1897 in drei Teilen im Journal Clunet erschienene Abhandlung zur Qualifikation umfaßt 80 Seiten$[$FN 16$]$. Sie ist damit kürzer als KAHNs \textsl{Gesetzeskollisionen}. Da sie sich jedoch fast ausschließlich mit den – in KAHNSCHER Terminologie – \textsl{latenten Kollisionen} befaßt, handelt sie diese ausführlicher und detaillierter ab, als KAHN es tat.

Schon die Überschrift (die im Nachdruck von 1899 geändert ist$[$FN 17$]$) enthält bereits das Programm, die zentrale Aussage der Abhandlung: die Unmöglichkeit, die Gesetzeskollisionen jemals endgültig beseitigen zu können. 

$[$FN 16$]$ S. C 225-255, 466-495, 720-738.

$[$FN 17$]$ Études de droit international privé, Teil 1 : „La théorie des qualifications en droit international privé“.}
\end{fragmentpart}
\begin{fragmentpart}{Anmerkung}
Die Quelle wird in der vorausgegangenen Seite am Ende einer langen Fußnote erwähnt, allerdings ohne erkennbaren Bezug auf die übernommene Textstelle.
\end{fragmentpart}
\end{fragment}
\phantomsection{}
\belowpdfbookmark{Fragment 44 1--4,~5--12,~101--103,~130--132}{Lm-Fragment-044-01}
\hypertarget{Lm-Fragment-044-01}{}
\begin{fragment}
\begin{fragmentpart}{Dissertation S.~44 Z.~1--4,~5--12,~101--103,~130--132 (BauernOpfer)}
\enquote{BARTIN beginnt seine Argumentation mit der Klärung der Problemlage: Er verdeutlicht, um welche \textsl{conflits de lois} es ihm geht.$[$FN 50$]$ Er tut dies anhand einiger Fallbeispiele, darunter dem der \textquotedbl{}maltesischen Witwe\textquotedbl{}.$[$FN 51$]$ Alle Beispiele BARTINs gehören demselben Falltypus an. $[$...$]$ Ein Anspruch besteht, wenn eine bestimmte anspruchsbegründende fremde Sachnorm anwendbar ist. Fraglich ist, welcher von mehreren möglicherweise einschlägigen, aber in den Rechtsfolgen unterschiedlichen eigenen Kollisionsnormen diese fremde Sachnorm zuzuordnen ist. Je nachdem kommt letztere zur Anwendung oder nicht, d.h., ist das Begehren des Anspruchstellers erfolgreich oder nicht.$[$FN 52$]$ Im Zusammenhang mit der Darstellung seiner Beispiele fällt nun erstmals das Wort \textsl{qualification}.$[$FN 53$]$

$[$FN 50$]$ Vgl. BARTIN, a.a.O. (Fn. 4), S. 226-230; WEBER, a.a.O. (Fn. 1), S.41-43.

$[$FN 51$]$ Der bekannte Fall \textsl{Bartholo}, Cour d'appel d'Alger vom 24.12.1889, Clunet 1891, S. 1171-1175 $[$...$]$

$[$FN 52$]$ Der Falltypus formelhaft in WEBER, a.a.O., S. 42; VRELLIS, IPR, S. 48.

$[$FN 53$]$ BARTIN, Clunet 1897, S. 227. Das Problem hatte damit seinen Namen gefunden, der sofort aufgegriffen wurde.}
\end{fragmentpart}
\begin{fragmentpart}{Original \cite[S.~41--42 Z.~24--26,~111--113,~42--4,~7--11,~15--17,~120--122]{Weber-1986}}
\enquote{$[$Seite 41$]$ 

BARTIN beginnt seine Argumentation mit der Klärung der Problemlage: Er verdeutlicht, um welche \textsl{conflits de lois} es ihm geht.$[$FN 25$]$ Er tut dies an Hand einiger Fallbeispiele, darunter dem der \textquotedbl{}maltesischen Witwe\textquotedbl{}.$[$FN 26$]$ --- $[$...$]$

$[$FN 25$]$ S. C 226-230 = E 2-8 = W 347-352

$[$FN 26$]$ Entscheidung d. Cour d'alger v. 24.12.1889, Clunet (o. S. 34 FN 27) 1891, 1171-1175 (Fall Anton-Bartholo): $[$...$]$

$[$Seite 42$]$

$[$...$]$ 

Alle Beispiele BARTINS gehören demselben Falltypus an: Ein Anspruch besteht, wenn eine bestimmte anspruchsbegründende fremde Sachnorm anwendbar $[$...$]$ ist. Fraglich ist, welcher von mehreren möglicherweise einschlägigen, aber in den Rechtsfolgen unterschiedlichen eigenen Kollisionsnormen diese fremde Sachnorm zuzuordnen ist. Je nachdem kommt letztere zur Anwendung oder nicht, das heißt, je nachdem ist das Begehren des Anspruchstellers erfolgreich oder nicht. $[$...$]$

In Zusammenhang mit der Darstellung seiner Beispiele fällt nun erstmals$[$FN 30$]$ das Wort \textsl{qualification}.$[$FN 31$]$ Das Problem hatte damit seinen Namen gefunden, der sofort aufgegriffen wurde$[$FN 32$]$.


$[$FN 30$]$ Aber s. u. S. 200ff.

$[$FN 31$]$ S. C 227 = E4 = W349

$[$FN 32$]$ S. u. S. 53, 57.}
\end{fragmentpart}
\begin{fragmentpart}{Anmerkung}
Fortsetzung in \hyperlink{Lm-Fragment-045-01}{Fragment\_045\_01}
\end{fragmentpart}
\end{fragment}
\phantomsection{}
\belowpdfbookmark{Fragment 45 1--16}{Lm-Fragment-045-01}
\hypertarget{Lm-Fragment-045-01}{}
\begin{fragment}
\begin{fragmentpart}{Dissertation S.~45 Z.~1--16 (BauernOpfer)}
\enquote{Im eigentlichen Hauptteil seiner Arbeit geht es BARTIN um die Beantwortung der eingangs gestellten Frage nach dem Qualifikations\textsl{statut}. Er greift SAVIGNYs zentralen Satz auf: Die Anwendung dieses oder jenes (Sach-) Rechts auf eine Rechtsbeziehung hängt von seiner Natur ab. Welches Recht aber bestimmt diese Natur? Für BARTIN ist dies die \textsl{lex fori}; sie soll die Natur des Rechtsverhältnisses bestimmen.$[$54$]$ Der Zentralbegriff in BARTINs Begründung für diese These ist die Souveränität des Staates, die er als absolut ansieht. Eine Motivation für den Staat, seine Souveränität mit der Anwendung ausländischen Rechts einzuschränken, liegt für BARTIN nicht etwa in der alten Lehre von der \textsl{comitas gentium}.$[$55$]$ Ein Staat erlaube die Anwendung ausländischen Rechts allenfalls, weil und wenn er dies für gerecht erachte.

Nach der Behandlung des Qualifikationsstatuts (lex fori$[$56$]$) wendet sich BARTIN der Qualifikations\textsl{methode} zu: Maßgeblichkeit der lex fori bedeute, daß für die Bestimmung der Natur eines Rechtsverhältnisses die Vorstellung und Kategorien des eigenen Rechts von eben diesem Rechtsverhältnis zu ermitteln seien. Diese Vorstellung solle im Ergebnis den Ausschlag geben.$[$57$]$

$[$54$]$ BARTIN, ibid., S. 235-246; vgl. auch WEBER, a.a.O., S. 43-45.

$[$55$]$ Zu der comitas-Lehre rechtsvergleichend und -historisch vgl. den ausführlichen Aufsatz von PAUL, Comity in International Law (Private International Law), HarvIntLJ 1991, S. 1-79.

$[$56$]$ Zur Herstellung des doktrinär-historischen Zusammenhanges mag die Feststellung genügen, daß die Theorie der lex fori die Qualifikation der kollisionsrechtlichen Begriffe im Rahmen und unter Zuhilfenahme der Begriffsordnung des Rechtssystems des Richters (besser: des Beurteilers) postuliert, der einen konkreten internationalprivatrechtlichen Fall zu lösen hat. Cf. NIEDERER, Die Frage der Qualifikation, S. 23, 70-80.

$[$57$]$ BARTIN, RCADI 1930-I, S. 568, nennt den Qualifikationskonflikt \textquotedbl{}l'obstacle décisif au succès de la methode universelle, --- radicalement contraire à la conception international de droit international privé $[$...$]$ par sa répercussion sur toutes les théories fondamentales du droit international privé\textquotedbl{}.}
\end{fragmentpart}
\begin{fragmentpart}{Original \cite[S.~43--44 Z.~21--24,~1--4,~6--9,~21--30]{Weber-1986}}
\enquote{$[$Seite 43$]$

Im eigentlichen Hauptteil$[$42$]$ seiner Arbeit geht es BARTIN um die Beantwortung der eingangs gestellten Fragen: Wie sind die Beispielsfälle zu lösen? Allgemein: Nach welchem Recht wird qualifiziert? BARTIN formuliert noch ausführlicher und greift damit unausgesprochen$[$43$]$ SAVIGNYS zentralen Satz$[$44$]$ 

$[$Seite 44$]$ 

auf: Wenn die Anwendung dieses oder jenes (Sach)Rechts auf eine Rechtsbeziehung von der Natur dieser Beziehung abhängt --- welches Recht bestimmt diese Natur$[$45$]$?

Die Antwort auf diese Frage ist für BARTIN nicht zweifelhaft $[$...$]$: es
ist die \textsl{lex fori}, die die Natur des Rechtsverhältnisses bestimmen soll.

Der Zentralbegriff in BARTINs Begründung für diese These ist der der
Souveränität des Staates, die er als absolute sieht. $[$...$]$

Die Motivation für den Staat, seine Souveränität überhaupt in dieser Weise einzuschränken, liegt für BARTIN nicht in der alten Lehre von der \textsl{comitas gentium}. Vielmehr tue der Staat dies aus im engeren Sinne juristischen Erwägungen: der Staat erlaube die Anwendung fremden Rechts, weil und wenn er dies für gerecht$[$48$]$ erachte.

Nach Behandlung des Qualifikationsstatus$[$49$]$ (\textsl{lex fori}) wendet sich BARTIN der Qualifikationsmethode$[$49$]$ zu: Maßgeblichkeit der \textsl{lex fori} bedeute, daß für die Bestimmung der Natur eines Rechtsverhältnisses die Vorstellung des eigenen Rechts von eben diesem Rechtsverhältnis zu ermitteln sei und daß diese Vorstellung im Ergebnis den Ausschlag gebe$[$50$]$.

$[$42$]$ S. C 235-246 = E 13-24 = W 357-369 (Teil III).

$[$43$]$ S. aber u. S. 47.

$[$44$]$ S. o. S. 9.

$[$45$]$ S. C 235 = E 13 = W 357.

$[$46$]$ S. C 236 = E 14 = W 308 mit Paraphrasierungen auf den folgenden Seiten.

$[$47$]$ „il ne serait plus maître chez soi“ (S. C 239 = E 17 = W 361).

$[$48$]$ „$[$. . .$]$ parce qu’il les croit justes“ (S. C 237 = E 15 = W 359).

$[$49$]$ Diese Begriffe verwendet BARTIN noch nicht. Sie tauchen erst spät auf.

$[$50$]$ S. C 238 = E 16 = W 360.}
\end{fragmentpart}
\begin{fragmentpart}{Anmerkung}
Fortsetzung von \hyperlink{Lm-Fragment-044-01}{Fragment\_044\_01}. Dort wird Weber in Fußnote 50 erwähnt, hier in Fußnote 54 als \textquotedbl{}vgl. auch\textquotedbl{}.
\end{fragmentpart}
\end{fragment}
\phantomsection{}
\belowpdfbookmark{Fragment 46 9--12}{Lm-Fragment-046-09}
\hypertarget{Lm-Fragment-046-09}{}
\begin{fragment}
\begin{fragmentpart}{Dissertation S.~46 Z.~9--12 (Verschleierung)}
\enquote{Er resümiert in wenigen Sätzen: Ein tatsächlich einheitliches IPR setzt ein einheitliches Verständnis seiner Begriffe voraus. Dies wiederum ist letztlich nur bei Einheitlichkeit der Sachrechte überhaupt möglich und darum nur möglich zum Preis des Verschwindens des IPR, es wäre dann ja überflüssig.$[$61$]$

$[$61$]$ BARTIN spricht von einem \textquotedbl{}\textsl{Pyrrhus-Sieg}\textquotedbl{}, ja sogar von einem \textquotedbl{}Selbstmord\textquotedbl{} des IPR; BARTIN, Clunet 1897, S. 735.}
\end{fragmentpart}
\begin{fragmentpart}{Original \cite[S.~48 Z.~2--6]{Weber-1986}}
\enquote{In wenigen Sätzen resümiert er: ein tatsächlich einheitliches IPR setzt ein einheitliches Verständnis seiner Begriffe voraus. Dies wiederum ist letztlich nur möglich bei Einheitlichkeit der Sachrechte überhaupt --- und darum nur möglich zum Preis des Verschwindens des IPR, es wäre dann ja überflüssig$[$79$]$.

$[$79$]$ BARTIN spricht von einem „Pyrrhus-Sieg“, ja sogar von „Selbstmord“ (S. C 734 = E 72f).}
\end{fragmentpart}
\begin{fragmentpart}{Anmerkung}
Hier übernimmt Lm auch die Fußnote von Weber (allerdings mit geänderter Seitenzahl). Der letzte Verweis auf Weber war in Fußnote 54 am Anfang der Vorseite.
\end{fragmentpart}
\end{fragment}
\phantomsection{}
\belowpdfbookmark{Fragment 48 3--16}{Lm-Fragment-048-03}
\hypertarget{Lm-Fragment-048-03}{}
\begin{fragment}
\begin{fragmentpart}{Dissertation S.~48 Z.~3--16 (BauernOpfer)}
\enquote{Ferner ist zu bemerken, daß BARTIN, ebenso wie schon KAHN in seinen Gesetzeskollisionen,$[$69$]$ seine Studie über den renvoi zunächst mit dem Versuch einer Kategoriebildung der verschiedenen Kollisionsarten beginnt. Neben den „eigentlichen Gesetzeskollisionen“ (conflits de lois proprement dits), er meint damit unterschiedliches Sachrecht in verschiedenen Rechtsordnungen, erkennt er zwei Arten von „IPR-Kollisionen“ (conflits qui s’élèvent entre dispositions législatives de droit international privé):$[$70$]$ renvoi-Konflikte, bei denen verschiedene Rechtsordnungen unterschiedliche, d.h. unterschiedlich anknüpfende, Kollisionsnormen haben und Qualifikationskonflikte, die vorliegen, wenn die Kollisionsnormen verschiedener Rechtsordnungen zwar gleich sind, aber auf dasselbe Rechtsinstitut nicht einheitlich angewandt werden.$[$71$]$ Die KAHNsche Terminologie erlaubt eine Präzisierung: zu unterscheiden ist zwischen „ausdrücklichen“ (renvoi-Konflikten) einerseits und „latenten“ (Qualifikationskonflikten) andererseits.$[$72$]$

$[$70$]$ BARTIN, a.a.O. (Fn. 49), S. 129-187, 272-310 (129f.) = BARTIN, Études de droit international privé, S. 83-187 (83f).

$[$71$]$ Ibid, S. 131f. bzw. S. 85f. Interessanterweise sind nach STEINDORFF, Sachnormen im IPR, S. 54 (Fn. 5), viel später „als Qualifikationskonflikte die Fälle zu bezeichnen, in denen das Qualifikationsverfahren eine eindeutige Rechtswahl nicht ermöglicht“ während VON STEIGER, a.a.
O., S. 21 (Fn. 1) noch 1937, meinte, es gäbe keine solchen Konflikte: stets handele es sich um Qualifikationsfragen.

$[$72$]$ Cf. RIGAUX, DIP, a.a.O. (Fn. 66); WEBER, a.a.O, S. 74.}
\end{fragmentpart}
\begin{fragmentpart}{Original \cite[S.~74 Z.~15--26]{Weber-1986}}
\enquote{Wie schon KAHN seine Gesetzeskollisionen$[$174$]$ beginnt BARTIN (erst) den Aufsatz zum renvoi mit dem Versuch einer Kategoriebildung der verschiedenen Kollisionsarten. Neben den „eigentlichen Gesetzeskollisionen“$[$175$]$ – er meint damit unterschiedliches Sachrecht in verschiedenen Rechtsordnungen – erkennt er zwei Arten von „IPR-Kollisionen“$[$176$]$: renvoi-Konflikte, bei denen verschiedene Rechtsordnungen unterschiedliche, das heißt unterschiedlich anknüpfende Kollisionsnormen haben, und Qualifikationskonflikte, wenn die Kollisionsnormen verschiedener Rechtsordnungen zwar gleich sind, aber auf dasselbe Rechtsinstitut nicht einheitlich angewandt werden. Es handelt sich also, in KAHNSCHER Terminologie$[$178$]$, genau um die Unterscheidung zwischen ‚ausdrücklichen‘ (renvoi-Konflikt) und ‚latenten‘ (Qualifikationskonflikt) Gesetzeskollisionen$[$179$]$.

$[$174$]$ S. o. S. 25ff.

$[$175$]$ „conflits des lois proprement dits“, BARTIN, renvoi, S. R 129f = E83f.

$[$176$]$ „conflits qui s’élèvent entre dispositions législatives de droit international privé\textquotedbl{}, S. R 130 = E84.

$[$177$]$ S. R 131f = E 85f.

$[$178$]$ Die BARTIN bei der Abfassung dieser Arbeit kannte, vgl. oben S. 41.

$[$179$]$ Gesetzeskollisionen, S. A 6, 92 = J 7, 107.}
\end{fragmentpart}
\begin{fragmentpart}{Anmerkung}
Der Verweis auf die Quelle erfolgt in der letzten Fußnote als zweiter Eintrag unter \textquotedbl{}Cf\textquotedbl{}. Die in diesem Fragment vorliegende Übernahme von Inhalt und auch vieler Formulierungen ist damit nicht abgedeckt.

Könnte alternativ als Verschleierung gewertet werden.
\end{fragmentpart}
\end{fragment}
\phantomsection{}
\belowpdfbookmark{Fragment 49 115--122}{Lm-Fragment-049-115}
\hypertarget{Lm-Fragment-049-115}{}
\begin{fragment}
\begin{fragmentpart}{Dissertation S.~49 Z.~115--122 (Verschleierung)}
\enquote{$[$FN 81$]$ NIEMEYER, Zur Methodik des internationalen Privatrechts, 1894, S. 12, 19. Er stellte nicht das ganze System Kahns dar, nahm aber dessen Gedanken auf und staffelte sie. Besonders relevant ist sein Ergebnis: „Die wörtliche Übereinstimmung der in zwei Rechtsgebieten geltenden Kollisionsnormen gibt keineswegs die Sicherheit inhaltlicher Übereinstimmung.” Mit dieser Aussage geht NIEMEYER insofern über KAHN hinaus, als er die wichtige perspektivische Änderung vollzieht, auch im Qualifikationsproblem nicht mehr die Sachnorm, sondern die Kollisionsnorm als Ausgangspunkt der Überlegungen zu wählen. Cf. WEBER, a.a.O., S. 35f.}
\end{fragmentpart}
\begin{fragmentpart}{Original \cite[S.~35--36 Z.~24--26,~28--30,~1--4]{Weber-1986}}
\enquote{$[$Seite 35$]$

NIEMEYER nahm in seiner kurzen, aber einflußreichen Schrift \textsl{Zur Methodik des Internationalen Privatrechts} die KAHNSCHEN Gedanken auf und straffte sie. Er stellte nicht das ganze System KAHNs dar, $[$...$]$ In klarer Diktion $[$...$]$ führte er seine Zuhörer$[$FN 39$]$ zu dem Ergebnis: „Die wörtliche Übereinstimmung der in zwei Rechtsgebieten geltenden Kollisionsnormen giebt keineswegs die Sicherheit inhaltlicher

$[$Seite 36$]$

Übereinstimmung.“$[$FN 40$]$ Mit dieser Aussage geht NIEMEYER insofern über KAHN hinaus, als er die wichtige perspektivische Änderung vollzieht, auch im „Qualifikationsbereich“ nicht mehr die Sachnorm, sondern die Kollisionsnorm als Ausgangspunkt der Überlegungen zu wählen.}
\end{fragmentpart}
\begin{fragmentpart}{Anmerkung}
Etwas gekürzt, aber wortwörtlich übernommen. Art und Umfang der Übernahme bleiben trotz Nennung der Quelle weitgehend im Dunkeln.

Lm \textquotedbl{}korrigiert\textquotedbl{} (versehentlich?) die Schreibweise von \textquotedbl{}giebt\textquotedbl{} im Zitat aus Niemeyer. Auch die angeblich gestaffelte (statt gestraffte) Darstellung könnte einen Übernahmefehler darstellen.
\end{fragmentpart}
\end{fragment}
\phantomsection{}
\belowpdfbookmark{Fragment 50 1--4}{Lm-Fragment-050-01}
\hypertarget{Lm-Fragment-050-01}{}
\begin{fragment}
\begin{fragmentpart}{Dissertation S.~50 Z.~1--4 (Verschleierung)}
\enquote{Mit dem Journal Clunet hatte BARTIN dagegen das richtige Forum gewählt, um seine Überlegungen bei den Internationalprivatrechtlern weithin bekannt zu machen. Binnen weniger Jahre setzten sich eine Reihe von Autoren ausführlich mit seinem Aufsatz und dem Problem der Qualifikation auseinander.}
\end{fragmentpart}
\begin{fragmentpart}{Original \cite[S.~53 Z.~4--9]{Weber-1986}}
\enquote{Mit dem Journal Clunet hatte BARTIN das richtige Forum gewählt, um seine Überlegungen bei den Fachgelehrten bekannt zu machen. Er fand sofort die Resonanz, die KAHN (in bezug auf die Qualifikation) zunächst versagt geblieben war. Binnen weniger Jahre setzte sich eine Reihe von Autoren ausführlich und vertieft mit dem BARTINSCHEN Aufsatz und dem Problem der Qualifikation auseinander.}
\end{fragmentpart}
\begin{fragmentpart}{Anmerkung}
Die Quelle wird auf S. 50 am Ende einer Fußnote, die sich mit einem anderen Autor (Diena) befasst, als \textquotedbl{}cf\textquotedbl{} erwähnt.
\end{fragmentpart}
\end{fragment}
\phantomsection{}
\belowpdfbookmark{Fragment 50 101--107}{Lm-Fragment-050-101}
\hypertarget{Lm-Fragment-050-101}{}
\begin{fragment}
\begin{fragmentpart}{Dissertation S.~50 Z.~101--107 (Verschleierung)}
\enquote{$[$FN 82$]$ SCHNELL, Über die Zuständigkeit zum Erlaß von Gesetzen und über die räumliche Herrschaft der Rechtsnormen, ZIR 5 (1895), S. 337-343 (342f.). Er wandte sich schon vom Grundsatz her gegen KAHN. Für ein wesentliches Argument zugunsten seiner Ansicht hält SCHNELL, unter Bezugnahme gerade auf KAHN und seine Ausführungen zu den latenten Gesetzeskollisionen, „daß die für den Sitz der Rechtsverhältnisse bestimmenden Eigenschaften der Rechtsverhältnisse, da sie ihre Quellen in den einzelnen Territorialrechten haben, in den verschiedenen Staatsgebieten verschieden seien“.}
\end{fragmentpart}
\begin{fragmentpart}{Original \cite[S.~36 Z.~18--23]{Weber-1986}}
\enquote{SCHNELL$[$FN 43$]$ wandte sich schon vom Grundsatz her gegen KAHN. $[$...$]$ $[$FN 45$]$. Für ein wesentliches Argument zugunsten seiner Ansicht hält SCHNELL, unter Bezugnahme gerade auf KAHN und seine Ausführungen zu den latenten Gesetzeskollisionen, „daß die für den Sitz der Rechtsverhältnisse bestimmenden Eigenschaften der Rechtsverhältnisse, da sie ihre Quellen in den einzelnen Territorialrechten haben, in den verschiedenen Staatsgebieten verschieden“ seien$[$FN 46$]$.

$[$FN 43$]$ Zuständigkeit.

$[$FN 45$]$ S. 342; vgl. o. S. 30f.

$[$FN 46$]$ S. 343.}
\end{fragmentpart}
\begin{fragmentpart}{Anmerkung}
Auch abgesehen vom Originalzitat wortwörtlich identisch, ohne dass die Quelle genannt wurde.
\end{fragmentpart}
\end{fragment}
\phantomsection{}
\belowpdfbookmark{Fragment 50 108--112}{Lm-Fragment-050-108}
\hypertarget{Lm-Fragment-050-108}{}
\begin{fragment}
\begin{fragmentpart}{Dissertation S.~50 Z.~108--112 (Verschleierung)}
\enquote{$[$FN 83$]$ ZITELMANN, IPR, Bd. 1, 1897, S. 141-149, 205-212, 238, 294ff. (317); Bd. 2, 1912, S. 6-11; DERS., Nachruf für Franz Kahn, ZIR 15 (1905), S. 1-10. In Zusammenhang mit dem Wesen der Kollisionsnorm und mit der Technik der internationalprivatrechtlichen Fallösung stellte er Erörterungen zu typischen Qualifikationsfragestellungen an, von denen ihn der argumentative Weg allerdings wieder zum völkerrechtlichen Ansatz zurückführte.}
\end{fragmentpart}
\begin{fragmentpart}{Original \cite[S.~37 Z.~4--8]{Weber-1986}}
\enquote{$[$...$]$ ZITELMANN $[$...$]$ Weniger bekannt ist, daß er sich auch ausführlich mit dem Wesen der Kollisionsnorm und mit der Technik der internationalprivatrechtlichen Fallösung beschäftigt hat$[$FN 50$]$. In diesem Zusammenhang stellte er Erörterungen zu typischen Qualifikationsfragestellungen an$[$FN 51$]$, von denen ihn der argumentative Weg allerdings wieder zum völkerrechtlichen Ansatz zurückführte.

$[$FN 50$]$ Insbesondere Bd. 1, S. 141-149, 205-212, 294ff; Bd. 2, S. 6-11.

$[$FN 51$]$ Bd. 2, S. 7-10, 371f, 465 (die beiden letztgenannten Stellen stammen allerdings schon aus dem erst 1903 ausgelieferten Teil des Buches).}
\end{fragmentpart}
\begin{fragmentpart}{Anmerkung}
Wortwörtliche Übereinstimmung. Aufbau der Fußnote erfolgt nach dem auf dieser Seite mehrfach angewandten Prinzip ohne Nennung der Quelle.
\end{fragmentpart}
\end{fragment}
\phantomsection{}
\belowpdfbookmark{Fragment 50 113--118}{Lm-Fragment-050-113}
\hypertarget{Lm-Fragment-050-113}{}
\begin{fragment}
\begin{fragmentpart}{Dissertation S.~50 Z.~113--118 (Verschleierung)}
\enquote{$[$FN 84$]$ PRUDHOMME, La loi territoriale et les traités diplomatiques devant des jurisdictions des états contractants. Paris 1910, S. 118-121, 126f, 328-33, der mit seiner Studie einen Beitrag zur Auslegung und Anwendung diplomatischer Vertäge $[$sic!$]$ vor den Gerichten der vertragschließenden Staaten leistet und damit der Qualifikationsmethode einige Aufmerksamkeit schenkt. Er unterscheidet dabei zwei Teilvorgänge, bei deren ersterem er von \textsl{qualifier}, beim zweiten von \textsl{classer} spricht. (S. 120)}
\end{fragmentpart}
\begin{fragmentpart}{Original \cite[S.~63 Z.~3--8,~111--112]{Weber-1986}}
\enquote{Unter den Franzosen, die sich in den Spuren BARTINS bewegen, ist noch André PRUDHOMME zu nennen, $[$...$]$. Seine \textsl{thèse} zur Auslegung und Anwendung von diplomatischen Verträgen vor den Gerichten der vertragschließenden Staaten$[$FN 86$]$ ist hier insoweit bemerkenswert und über BARTIN hinausgehend, als er der Qualifikationsmethode einige Aufmerksamkeit schenkt und dabei zwei Teilvorgänge unterscheidet, bei deren ersterem er von \textsl{qualifier}, beim zweiten von \textsl{classer} spricht$[$FN 87$]$

$[$FN 86$]$ PRUDHOMME,, traités (1910), S. 118-121, 126f. 328-333.

$[$FN 87$]$ S. 120. $[$...$]$}
\end{fragmentpart}
\begin{fragmentpart}{Anmerkung}
Erneut wird eine umfangreichere Fußnote ohne Nennung der Quelle Weber (1986) aus Originaltext im Originalwortlaut und den vorgefundenen Literaturverweisen zusammengebaut.
\end{fragmentpart}
\end{fragment}
\phantomsection{}
\belowpdfbookmark{Fragment 50 122--127}{Lm-Fragment-050-122}
\hypertarget{Lm-Fragment-050-122}{}
\begin{fragment}
\begin{fragmentpart}{Dissertation S.~50 Z.~122--127 (Verschleierung)}
\enquote{$[$FN 86$]$ ANZILOTTI, Anmerkung zur Entscheidung Corte d'Appello Milano v. 1.7.1914, Riv. dir.int. 8 (1914), S. 610-614 (614). Er behauptet, daß die \textsl{lex fori} zwar für die Auslegung der Kollisionsnorm zuständig sei, danach aber das vermittels dieser Kollisionsnorm gefundene Recht die Herrschaft übernehme, und zwar auch hinsichtlich der Begrifflichkeit und der Einteilungen. Damit beginnt die vor allem in Italien stark aufkommende, später allgemein verbreitete Lehre von der Qualifikation nach Graden, Stufen oder Schritten.}
\end{fragmentpart}
\begin{fragmentpart}{Original \cite[S.~65 Z.~14--16,~18--24]{Weber-1986}}
\enquote{ANZILOTTI$[$FN 104$]$ $[$...$]$ veröffentlichte in der hier interessierenden Zeitspanne eine kurze Urteilsanmerkung in der \textsl{Rivista di Diritto Internazionale}$[$FN 106$]$, $[$...$]$ daß die \textsl{lex fori} zwar für die Auslegung der Kollisionsnorm zuständig sei, danach aber das vermittels dieser Kollisionsnorm gefundene Recht die Herrschaft übernehme, und zwar auch hinsichtlich der Begrifflichkeit und der Einteilungen$[$FN 108$]$. Diese kurze Stelle bei Anzilotti hat viel Beachtung gefunden; auf ihr beruht letztlich die vor allem in Italien stark aufkommende, später allgemein verbreitete Lehre von der Qualifikation nach Graden, Stufen oder Schritten$[$FN 109$]$.

$[$FN 104$]$ Zu seiner Person siche MORELLI, Anzilotti; PERASSI, Anzilotti.

$[$FN 106$]$ ANZILOTTI, Anmerkung zur Entscheidung vom 1. 7. 1914; $[$...$]$

$[$FN 108$]$ „Una volta poi determinata la legge regolatrice dell’obbligazione secondo l'art. 58, spetterà a questa di stabilire quali norme sono applicabili, e qindi anche se le norme civili o quelle commerciali, perchè il nostro ordinamento giuridico vuole che quel rapporto sia regolato come in quel dato ordinamento straniero“, S. 614.

$[$FN 109$]$ S. u. S. 133ff, 157ff, 173 ff u.a.}
\end{fragmentpart}
\begin{fragmentpart}{Anmerkung}
Die Fußnote wird ohne Nennung der Quelle aus originalem Text von Weber (1986) zusammengebaut.
\end{fragmentpart}
\end{fragment}
\phantomsection{}
\belowpdfbookmark{Fragment 51 8--11,~12--15}{Lm-Fragment-051-08}
\hypertarget{Lm-Fragment-051-08}{}
\begin{fragment}
\begin{fragmentpart}{Dissertation S.~51 Z.~8--11,~12--15 (Verschleierung)}
\enquote{Mit dieser Zeit, den Jahren bis zum Ausbruch des Ersten Weltkrieges, endet auch die erste Rezeptions- und Diskussionsphase des Rechtsphänomens \textquotedbl{}Qualifikation\textquotedbl{}. Die Entdeckung der Qualifikation war die Frucht der Hochphase des Internationalen Privatrechts, nämlich des letzten Jahrzehnts des 19. Jh. $[$...$]$ Auch diejenigen Internationalisten, die sich grundsätzlicher mit dem internationalen Privatrecht befaßten, behandelten längst nicht mehr die allgemeinen Themen der neunziger Jahren $[$sic$]$(19. Jh.), sondern Einzelprobleme, vor allem solche der Haager IPR-$[$Konferenzen, auf denen nunmehr, im Schatten der Friedenskonferenzen, noch einige Abkommen geschlossen wurden.$]$}
\end{fragmentpart}
\begin{fragmentpart}{Original \cite[S.~75,~76,~77 Z.~75--15,~1--5]{Weber-1986}}
\enquote{$[$Seite 75$]$

Spätestens mit dieser Zeit, mit den Jahren um den Ausbruch des Ersten Weltkrieges, endet auch die erste Rezeptionsphase des juristischen Instituts \textsl{Qualifikation}. 

Die Entdeckung der Qualifikation war eine Frucht der eigentlichen Hochphase des Internationalen Privatrechts, nämlich des letzten Jahrzehnts des letzten Jahrhunderts. $[$...$]$ 

$[$Seite 76$]$

$[$...$]$ Auch diejenigen Internationalisten, die 

$[$Seite 77$]$

sich noch in stärkerem Maße mit dem IPR befaßten, behandelten vorwiegend nicht mehr die allgemeinen Themen der neunziger Jahre, sondern Einzelprobleme, vor allem solche der Haager IPR-Konferenzen, die, nunmehr, im Schatten der Friedenskonferenzen, noch einige Abkommen beschlossen.}
\end{fragmentpart}
\begin{fragmentpart}{Anmerkung}
Das Original wird \textquotedbl{}eingedampft\textquotedbl{}, bleibt aber unverkennbar; ein kurzer Satz wird dazwischengeschoben. Ein Hinweis auf die Quelle unterbleibt. Fortsetzung in \hyperlink{Lm-Fragment-052-01}{Fragment 052 01}.
\end{fragmentpart}
\end{fragment}
\phantomsection{}
\belowpdfbookmark{Fragment 52 1--2}{Lm-Fragment-052-01}
\hypertarget{Lm-Fragment-052-01}{}
\begin{fragment}
\begin{fragmentpart}{Dissertation S.~52 Z.~1--2 (Verschleierung)}
\enquote{$[$Auch diejenigen Internationalisten, die sich grundsätzlicher mit dem internationalen Privatrecht befaßten, behandelten längst nicht mehr die allgemeinen Themen der neunziger Jahren $[$sic$]$(19. Jh.), sondern Einzelprobleme, vor allem solche der Haager IPR-$]$Konferenzen, auf denen nunmehr, im Schatten der Friedenskonferenzen, noch einige Abkommen geschlossen wurden.

Speziell für Deutschland kam weiter hinzu, daß am 1. Januar 1900 das BGB in Kraft getreten war.}
\end{fragmentpart}
\begin{fragmentpart}{Original \cite[S.~76,~77 Z.~1--7]{Weber-1986}}
\enquote{Auch diejenigen Internationalisten, die

$[$Seite 77$]$

sich noch in stärkerem Maße mit dem IPR befaßten, behandelten vorwiegend nicht mehr die allgemeinen Themen der neunziger Jahren, sondern Einzelprobleme, vor allem solche der Haager IPR-Konferenzen, die, nunmehr, im Schatten der Friedenskonferenzen, noch einige Abkommen beschlossen. 

Speziell für Deutschland kommt ein weiteres hinzu. Am 1. Januar 1900 war das neue BGB in Kraft getreten.}
\end{fragmentpart}
\begin{fragmentpart}{Anmerkung}
Fortsetzung von \hyperlink{Lm-Fragment-051-08}{Fragment\_051\_08}. Auch weiterhin kein Verweis auf Weber trotz fast wörtlicher Übernahmen.
\end{fragmentpart}
\end{fragment}
\phantomsection{}
\belowpdfbookmark{Fragment 53 18--20,~116--117}{Lm-Fragment-053-16}
\hypertarget{Lm-Fragment-053-16}{}
\begin{fragment}
\begin{fragmentpart}{Dissertation S.~53 Z.~18--20,~116--117 (BauernOpfer)}
\enquote{Qualifikation ist nach ARMINJON die Bestimmung der rechtlichen Natur von Personen, Sachen und Handlungen.$[$107$]$ Vor ihrer Qualifikation seien diese Tatsachen, juristisch gesehen, nichts.$[$108$]$

$[$107$]$ ARMINJON, a.a.O. (Fn. 105), S. 438.

$[$108$]$ Ibid, S. 437.}
\end{fragmentpart}
\begin{fragmentpart}{Original \cite[S.~82 Z.~7--11,~102--104]{Weber-1986}}
\enquote{Lex fori-Qualifikation heißt auch für ARMINJON, daß der Inhalt der kollisionsrechtlichen Begriffe dem eigenen Sachrecht zu entnehmen ist$[$26$]$. Qualifikation ist nach ihm die Bestimmung der rechtlichen Natur von Personen, Sachen und Handlungen$[$27$]$. Vor ihrer Qualifikation seien die Tatsachen juristisch gesehen nichts$[$28$]$.

$[$26$]$ S. 431.

$[$27$]$ S. 438.

$[$28$]$ S. 437.}
\end{fragmentpart}
\begin{fragmentpart}{Anmerkung}
Verweis in der vorausgehenden Fußnote 106 auf die Quelle mit \textquotedbl{}Cf.\textquotedbl{} 19 zusammenhängende Wörter übernommen, damit auch die Umdrehung der Reihenfolge der Aussagen von Arminjon (dort S. 438, 437).
\end{fragmentpart}
\end{fragment}
\phantomsection{}
\belowpdfbookmark{Fragment 54 6--14}{Lm-Fragment-054-06}
\hypertarget{Lm-Fragment-054-06}{}
\begin{fragment}
\begin{fragmentpart}{Dissertation S.~54 Z.~6--14 (BauernOpfer)}
\enquote{Anhand teilweise ähnlicher Erwägungen gelangt NIBOYET zu fast identischen Ergebnissen. Er befürwortet die Qualifikation lege fori und zwar auch für die Unterscheidung zwischen Mobilien und Immobilien. Für ihn stellt die Qualifikation ein Problem der Auslegung, der Definition kollisionsrechtlicher Begriffe dar. Als einer der ersten befaßt er sich ausführlich mit dem Problem der Qualifikation bei zwei- und mehrseitigen Staatsverträgen. Hierzu schlägt er allerdings keine allgemeine Lösung vor, sondern fordert, die vertragschließenden Parteien sollten Qualifikationsfragen im Wortlaut des Vertrages ausdrücklich mitregeln.$[$FN 111$]$

$[$FN 111$]$ Cf. WEBER, a.a.O., S. 83-85 m.w.N..}
\end{fragmentpart}
\begin{fragmentpart}{Original \cite[S.~83--84 Z.~4--5,~6--7,~19--23,~1--2]{Weber-1986}}
\enquote{$[$Seite 83$]$

Auf teilweise ähnlichen Wegen gelangt Jean-Paulin NIBOYET zu fast identischen Ergebnissen. $[$...$]$ Er vertritt die Qualifikation \textsl{lege fori} und zwar auch was die Abgrenzung der Mobilien von den Immobilien angeht. $[$...$]$

Insgesamt faßt NIBOYET die Qualifikation als ein Problem der Auslegung, der Definition der kollisionsrechtlichen Begriffe auf$[$FN 42$]$.

Als einer der ersten schließlich befaßt er sich ausführlich mit dem Problem
der Qualifikation bei zwei- und mehrseitigen Staatsverträgen. Hierzu schlägt er allerdings keine allgemeine Lösung vor, sondern fordert, die 

$[$Seite 84$]$

vertragsschließenden Parteien sollten Qualifikationsfragen im Wortlaut des Vertrages ausdrücklich mitregeln$[$FN 43$]$.

$[$42$]$ 2. Aufl., S. 501.

$[$43$]$ 2. Aufl., S. 516-523; diesem Thema schenkt NIBOYET dann besonders in den späten zwanziger und in den dreißiger Jahren seine Aufmerksamkeit, s. considérations; rôle; problème.}
\end{fragmentpart}
\begin{fragmentpart}{Anmerkung}
Lm verweist hier ausschließlich auf Weber. Die wörtlichen Übernahmen werden aber nicht gekennzeichnet.
Direkter Anschluss zu \hyperlink{Lm-Fragment-054-15}{Fragment\_054\_15}.
\end{fragmentpart}
\end{fragment}
\phantomsection{}
\belowpdfbookmark{Fragment 54 16--22,~112--113}{Lm-Fragment-054-16}
\hypertarget{Lm-Fragment-054-16}{}
\begin{fragment}
\begin{fragmentpart}{Dissertation S.~54 Z.~16--22,~112--113 (Verschleierung)}
\enquote{In diesem ersten Aufsatz übernimmt er den Begriff \textsl{qualification} für die englische Sprache,$[$113$]$ der sich dort freilich nicht durchsetzte.$[$114$]$ Sein Ausgangspunkt ist das traditionelle angloamerikanische Verständnis von Recht überhaupt und vom IPR im besonderen: \textsl{\textquotedbl{}$[$...$]$ law is based upon the existence of physical force on the part of organized society$[$115$]$ $[$...$]$ all rights are created by the forum\textquotedbl{}.}$[$116$]$

$[$113$]$ C.f supra, Kapitel II 2.

$[$114$]$ LORENZEN, The Qualification, Classification or Characterization Problem in the Conflict of Laws, Yale L.J. 50 (1941), S. 743-761, bemerkt, daß der in Kontinentaleuropa durchgesetzte Begriff \textsl{\textquotedbl{}qualification\textquotedbl{}} im englischsprachigen Raum keinen Erfolg hatte.

$[$115$]$ LORENZEN, a.a.O. (Fn. 112), S. 276.

$[$116$]$ Ibid, S. 280.}
\end{fragmentpart}
\begin{fragmentpart}{Original \cite[S.~91 Z.~4--6,~12--16]{Weber-1986}}
\enquote{In diesem ersten Aufsatz übernimmt Lorenzen auch den Begriff \textsl{qualification} ins Englische, wo er sich aber, anders als in anderen europäischen Sprachen$[$95$]$, nicht durchsetzen konnte$[$96$]$.

$[$...$]$

Sein Ausgangspunkt ist dabei das traditionelle angloamerikanische Verständnis vom Recht überhaupt$[$97$]$ und vom IPR im besonderen: „$[$. . .$]$ law is based upon the existence of physical force on the part of organized society $[$. . .$]$“$[$98$]$ --- „$[$. . .$]$ all rights are created by the forum.“$[$99$]$

$[$95$]$ S. o. S. 14. 

$[$96$]$ S. u. S. 167.

$[$97$]$ Vgl. AUSTIN, S. 10, 13f, 17f etc.

$[$98$]$ LORENZEN, Theory, S. 276.

$[$99$]$ S. 280.}
\end{fragmentpart}
\begin{fragmentpart}{Anmerkung}
Anschluss an \hyperlink{Lm-Fragment-054-06}{Fragment\_054\_06}. Im Zusammenhang mit Lorenzen wird Weber nicht erwähnt. Auch die Fußnoten 115 und 116 sind bei Weber zu finden.

Wie auch in \hyperlink{Lm-Fragment-077-09}{Fragment\_077\_09} und \hyperlink{Lm-Fragment-086-03}{Fragment\_086\_03} enthält FN 112 eine kurzgefasste Gliederung des zitierten Werks:
:Lorenzen, The Theory of Qualifications and the Conflict of Laws, Columbia L.Rev.XX (1920), S. 247-282; mit einem Teil über die kontinentalen Ansichten (247-259), einem Teil über die Entdeckung des Problems und seine allgemeinen Lehren (259-264) und einem Teil über das anglo-amerikanische Recht (264-282).
\end{fragmentpart}
\end{fragment}
\phantomsection{}
\belowpdfbookmark{Fragment 56 115--118}{Lm-Fragment-056-115}
\hypertarget{Lm-Fragment-056-115}{}
\begin{fragment}
\begin{fragmentpart}{Dissertation S.~56 Z.~115--118 (Verschleierung)}
\enquote{$[$128$]$ Urteil des 4. Senats des RG v. 7.7.1932, SeuffArch 86, S. 353-358, in dem die Frage der Beurteilung eines griechischen Anspruches als erbrechtlich oder nicht dem griechischen Erbstatut entnommen wurde. Dies ist auch das erste Urteil des RG, in dem der Begriff „Qualifikation“ auftaucht.}
\end{fragmentpart}
\begin{fragmentpart}{Original \cite[S.~153 Z.~24--27,~102--103]{Weber-1986}}
\enquote{$[$...$]$ Urteil des 4. Senats des Reichsgerichts aus dem Jahre 1932$[$113$]$, in dem die Frage der Beurteilung eines griechischen Anspruches (als erbrechtlich oder nicht) dem griechischen Recht als Erbstatut entnommen wurde, $[$...$]$

$[$113$]$ V. 7. 7. 1932, SeuffArch 86, 353-358 (dies ist auch das erste Urteil des RG, in dem, seit BARTIN ihn eingeführt hat, der Begriff 'Qualifikation' auftaucht, S. 353 f; $[$...$]$}
\end{fragmentpart}
\begin{fragmentpart}{Anmerkung}
Weber wird mit einer anderen Stelle (112-114) in der vorausgegangenen Fußnote als weiterführender Verweis zum Autor WIGNY erwähnt, der diese Entscheidung kommentiert hat. Wohl eher Verschleierung als Bauernopfer.
\end{fragmentpart}
\end{fragment}
\phantomsection{}
\belowpdfbookmark{Fragment 57 103--108}{Lm-Fragment-057-103}
\hypertarget{Lm-Fragment-057-103}{}
\begin{fragment}
\begin{fragmentpart}{Dissertation S.~57 Z.~103--108 (Verschleierung)}
\enquote{$[$FN 130$]$ D'HÉRÉ, Des problèmes posés par la qualification des rapports juridiques en droit international privé, 1938. Im ersten Teil seiner Arbeit untersucht D'HÉRÉ die Bedeutung von Qualifikationskonflikten und befürwortet eine mittlere Einschätzung. Im zweiten Teil geht er der Frage nach dem Qualifikationsstatut nach. Im Ergebnis spricht er sich für eine Qualifikationsmethode lege causae aus, insbesondere anhand des Urteils, das supra (Fn. 128) genannt wurde.

$[$Seite 56$]$

$[$FN 128$]$ Urteil des 4. Senats des RG v. 7.7.1932, SeuffArch 86, S. 353-358, $[$...$]$}
\end{fragmentpart}
\begin{fragmentpart}{Original \cite[S.~186 Z.~1--2,~4--6,~9--12]{Weber-1986}}
\enquote{$[$Seite 185$]$

Im

$[$Seite 186$]$

ersten Teil seiner Arbeit untersucht D'HÉRÉ die Bedeutung von Qualifikationskonflikten, $[$...$]$. D'HÉRÉ befürwortet eine mittlere Einschätzung$[$FN 356$]$. Im zweiten Teil geht er der Frage nach dem Qualifikationsstatut nach$[$FN 357$]$. $[$...$]$ Im Ergebnis spricht sich D'HÉRÉ für eine Qualifikationsmethode \textsl{lege causae} aus, nach dem Muster der deutschen Entscheidung RG SeuffArch 86, 353$[$FN 359$]$, die ihm aus der Darstellung bei WIGNY bekannt ist$[$FN 360$]$.}
\end{fragmentpart}
\begin{fragmentpart}{Anmerkung}
Weber wird in der übernächsten Fußnote 132 erwähnt, die sich aber nicht auf d'Héré oder Meierhof bezieht.
\end{fragmentpart}
\end{fragment}
\phantomsection{}
\belowpdfbookmark{Fragment 57 109--114}{Lm-Fragment-057-109}
\hypertarget{Lm-Fragment-057-109}{}
\begin{fragment}
\begin{fragmentpart}{Dissertation S.~57 Z.~109--114 (Verschleierung)}
\enquote{$[$FN 131$]$ MEIERHOF, La portée des qualifications en droit international privé, 1938. Er gliedert seine Darstellung der verschiedenen Lehrmeinungen nach ihrer Nähe oder Ferne zur Lehre BARTINS: er bildet eine Gruppe von Vertretern der „klassischen Theorie“ (BARTIN), eine Gruppe, die nur dadurch gekennzeichnet sei, daß es sich um Gegner der klassischen Theorie handele (DESPAGNET, RABEL, FRANKENSTEIN und NEUNER vereint!) und eine dritte „mittlere Gruppe“ (z.B. BECKETT)}
\end{fragmentpart}
\begin{fragmentpart}{Original \cite[S.~185 Z.~19--21,~110--114]{Weber-1986}}
\enquote{MEIERHOF zum Beispiel gliedert seine Darstellung der verschiedenen Lehrmeinungen nach ihrer Nähe oder Ferne zur Lehre BARTINS$[$FN 354$]$.

$[$FN 354$]$ MEIERHOF bildet eine Gruppe von Vertretern der 'klassischen Theorie' (BARTIN), eine Gruppe, die nur dadurch gekennzeichnet ist, daß es sich um Gegner der klassischen Theorie handele (wodurch sich in dieser Gruppe so unterschiedliche Lehren wie die von DESPAGNET, RABEL, FRANKENSTEIN und NEUNER vereint finden) und eine dritte, 'mittlere' Gruppe (zu der MEIERHOF beispielsweise Beckett rechnet $[$...$]$)}
\end{fragmentpart}
\begin{fragmentpart}{Anmerkung}
Weber wird in der nächsten Fußnote 132 erwähnt, die sich aber nicht auf Meierhof bezieht. Bei Weber finden sich die Fußnote (und ihr Bezugssatz) in fast identischer Form.
\end{fragmentpart}
\end{fragment}
\phantomsection{}
\belowpdfbookmark{Fragment 58 1--6,~8--11}{Lm-Fragment-058-01}
\hypertarget{Lm-Fragment-058-01}{}
\begin{fragment}
\begin{fragmentpart}{Dissertation S.~58 Z.~1--6,~8--11 (BauernOpfer)}
\enquote{BECKETT wählt den Begriff \textsl{classification}, weil für ihn die Einordnung des Qualifikationsgegenstandes (Tatsachen, Normen oder Rechte) in die kollisionsrechtlichen Kategorien im Vordergrund steht.$[$FN 140$]$ Er befaßt sich ausführlich mit dem Aspekt des Qualifikationsstatuts, nicht aber mit anderen Fragen des Gesamtproblems;$[$FN 141$]$ für ihn ist Klassifikation in jedem kollisionsrechtlichen Fall notwendig und stellt ein fundamentales Problem des IPR dar.$[$FN 142$]$ $[$...$]$ in der Klassifikation auf der Basis der Rechtsvergleichung und \textsl{analytical jurisprudence}.$[$FN 143$]$ Sein Ansatz erscheint bei gleicher Zielsetzung praktikabler und realistischer als der RABELS. Was die Schärfe der analytischen Betrachtung angeht, bleibt er hinter RABEL zurück.$[$FN 144$]$

$[$FN 140$]$ Schon auf der ersten Seite der veröffentlichten Fassung seines „speziellen“ Vortrages (\textsl{special university lecture}) am King's College London setzt er sich mit der terminologischen Frage auseinander (S. 46, Fn. 3). Ferner spricht er sich für \textsl{analytical jurisprudence} aus.

$[$FN 141$]$ Cf. BECKETT, a.a.O. (Kapitel II, Fn. 11), S. 49-81 m.w.N. über die Rechtsprechung. $[$...$]$

$[$FN 142$]$ Ibid. S. 46, 81. $[$...$]$

$[$FN 143$]$ Seine eigene Ansicht gibt er auf S. 58-60. Über RABEL s. infra, Kapitel IV 1 a.

$[$FN 144$]$ Cf. WEBER, a.a.O., S. 168, und besonders kritisch MENDELSSOHN BARTHOLDY, B.Yb. I.L. 1935, passim.}
\end{fragmentpart}
\begin{fragmentpart}{Original \cite[S.~167--168 Z.~19--21,~3--4,~9--11,~16--17,~19--21,~22--23]{Weber-1986}}
\enquote{$[$Seite 167$]$

BECKETT wählt den Begriff \textsl{classification}$[$FN 222$]$, weil für ihn die Einordnung
von Tatsachen, Normen oder Rechten (näher auf den Qualifikationsgegenstand
geht er nicht ein) in die kollisionsrechtlichen Kategorien im Vorder-

$[$Seite 168$]$ 

grund steht$[$FN 223$]$. $[$...$]$ für BECKETT, daß Klassifikation in jedem IPR-Fall notwendig und ein fundamentales Problem des IPR sei$[$FN 224$]$.

$[$...$]$ zur kurzen Skizzierung seiner eigenen Ansicht $[$FN 225$]$ $[$...$]$ die auf der Basis der Ergebnisse der Rechtsvergleichung gewonnen werden --- BECKETT spricht insoweit von \textsl{analytical jurisprudence}, $[$...$]$ BECKETTS Ansatz ist daher bei gleicher Zielsetzung wohl praktikabler und realistischer als der RABELS, $[$...$]$ Was die Schärfe der analytischen Betrachtung angeht$[$FN 229$]$, bleibt er allerdings weit hinter RABEL zurück $[$...$]$; er befaßt sich auch fast ausschließlich mit dem Aspekt des Qualifikationsstatuts, nicht mit anderen Teilfragen des Gesamtproblems.

$[$FN 223$]$ BECKETT, Question, S. 46.

$[$FN 224$]$ S. 46, 81.

$[$FN 225$]$ S. 58-60.}
\end{fragmentpart}
\begin{fragmentpart}{Anmerkung}
Patchwork aus einer Vielzahl von textidentischen Originalzeilen. Einzig die Reihenfolge wird einmal verändert und die Fußnoten enthalten vereinzelt Ergänzungen. Am Schluss erfolgt dann ein Hinweis auf die eigentliche Quelle, wobei potentielle Leserinnen und Leser dies aufgrund des bis dahin vom Autoren demonstrierten Einsatzes von Quellenverweisen einzig auf die letzte Aussage und nicht auf den gesamten Abschnitt beziehen.
\end{fragmentpart}
\end{fragment}
\phantomsection{}
\belowpdfbookmark{Fragment 60 9--13}{Lm-Fragment-060-09}
\hypertarget{Lm-Fragment-060-09}{}
\begin{fragment}
\begin{fragmentpart}{Dissertation S.~60 Z.~9--13 (Verschleierung)}
\enquote{ROBERTSON faßt in seiner Studie auch die Ergebnisse der Abhandlungen englischsprachiger Autoren, dazu noch die CHESHIREs, $[$FN 157$]$ kritisch zusammen$[$FN 158$]$ und errichtet darauf sein eigenes Stufensystem der Qualifikation,$[$FN 159$]$, das einen frühen Höhepunkt der angloamerikanischen Lehre darstellt.
----
$[$FN 157$]$ ROBERTSON, ibid, zitiert (S. 22f., Fn. 46) CHESHIRE, Private International Law, London 1938, S. 24-45 und versagt sich eine Kritik, da CHESHIRE in dem Vorwort seines Lehrbuchs auf frühere Diskussionen mit ROBERTSON hinweist und bemerkt: \textquotedbl{}it may be that he $[$ROBERTSON$]$ is criticizing his former views as well as those of Dr. CHESHIRE\textquotedbl{}.

$[$FN 158$]$ ROBERTSON, ibid., S. 3-58.

$[$FN 159$]$ Ibid., S. 59-286.}
\end{fragmentpart}
\begin{fragmentpart}{Original \cite[S.~172 Z.~9--13]{Weber-1986}}
\enquote{ROBERTSON faßt die Ergebnisse der erwähnten englischsprachigen Autoren, dazu noch
die CHESHIREs$[$FN 260$]$, geschickt kritisch zusammen und baut darauf ein eigenes
Stufensystem der Qualifikation, das einen Höhepunkt der angloamerikanischen
Lehre darstellt.
----
$[$FN 260$]$ Zu CHESHIRE s. u . S. 179; völlig unberücksichtigt läßt ROBERTSON demgegenüber die Ausführungen des Walisers DAVIES in dessen Haager Kurs von 1937 (DAVIES, règles). Das ist um so auffälliger als ROBERTSON die Veröffentlichungen im Haager Recueil an sich durchaus heranzog (die Kurse BARINs, MAURYs, RAAPEs). (DAVIES gibt eine knappe Problemübersicht --- S. 490-502 --- und diskutiert einige Fälle aus der englischen Rechtsprechung).}
\end{fragmentpart}
\begin{fragmentpart}{Anmerkung}
Lm verweist in diesem Zusammenhang nicht auf Weber --- sonst wäre die Übernahme grenzwertig. Durch eine leichte Umstellung wird bei Lm aus Cheshire ein nicht englischsprachiger Autor --- offenbar ein Bearbeitungsversehen, weil Lm selbständig aus Cheshire zitiert.
\end{fragmentpart}
\end{fragment}
\phantomsection{}
\belowpdfbookmark{Fragment 61 20--23}{Lm-Fragment-061-20}
\hypertarget{Lm-Fragment-061-20}{}
\begin{fragment}
\begin{fragmentpart}{Dissertation S.~61 Z.~20--23 (Verschleierung)}
\enquote{Die Abhandlung ROBERTSONS beeindruckt durch umfängliche Verarbeitung von Fällen aus Literatur und Rechtsprechung$[$FN 167$]$ sowie durch die Sorgfalt, die der einzelnen Entscheidung gewidmet wird, zum Beispiel bei der präzisen Besprechung des Falles \textsl{Bartholo}.$[$FN 168$]$

$[$FN 167$]$ ROBERTSON, Characterization, S. 157-279 (fast die Hälfte der Monographie).

$[$$[$FN 168$]$ Cf. a.a.O. (Fn. 51, 154). Bei ROBERTSON, ibid., S. 65, 90, 159-163.$]$}
\end{fragmentpart}
\begin{fragmentpart}{Original \cite[S.~175 Z.~10--14]{Weber-1986}}
\enquote{ROBERTSONS Untersuchung von Fällen aus Literatur und Rechtsprechung verschiedener Länder beeindruckt durch ihren Umfang (fast die Hälfte des Buches) wie durch die Sorgfalt, die der einzelnen Entscheidung gewidmet wird, zum Beispiel bei der genauen Aufarbeitung der wirklichen Fakten des alten Klassikers von der maltesischen Witwe$[$FN 284$]$.

$[$FN 284$]$ ROBERTSON, Characterization, S. 138ff.}
\end{fragmentpart}
\begin{fragmentpart}{Anmerkung}
Der Fall Bartholo ist der Fall der maltesischen Witwe. 

Viele Übereinstimmungen in den Formulierungen, inhaltlich übereinstimmend, kein Hinweis auf die eigtl. Quelle des Gedankens. Absolut grenzwertig und im Anbetracht des Umfelds sicherlich ein Plagiat.
\end{fragmentpart}
\end{fragment}
\phantomsection{}
\belowpdfbookmark{Fragment 62 101--109}{Lm-Fragment-062-101}
\hypertarget{Lm-Fragment-062-101}{}
\begin{fragment}
\begin{fragmentpart}{Dissertation S.~62 Z.~101--109 (Verschleierung)}
\enquote{Aus einem Wechsel, der dem Recht des amerikanischen Bundesstaates Tennessee untersteht, wird vor einem deutschen Gericht geklagt, weil der Verpflichtete des „eigenen“ Wechsels mittlerweile umgezogen ist. Der Beklagte beruft sich auf Verjährung. Nach dem Recht von Tennessee verjähren Wechsel nach sechs Jahren, nach deutschem Recht (\textsl{forum}) nach drei Jahren. Das Gericht entscheidet für den Kläger, obwohl das deutsche materielle Verjährungsrecht, da nicht lex causae, unanwendbar sei, ebenso wie das amerikanische Verjährungsrecht, weil dieses dort als prozessual angesehen werde und das deutsche Gericht nicht amerikanisches Prozeßrecht anwenden könne.}
\end{fragmentpart}
\begin{fragmentpart}{Original \cite[S.~28--29 Z.~113--114,~101--105]{Weber-1986}}
\enquote{$[$Seite 28$]$

Aus einem Wechsel, der dem Recht des amerikanischen Bundesstaates Tennessee untersteht, wird vor einem deutschen Gericht geklagt. Der Beklagte beruft sich auf Verjährung. 

$[$Seite 29$]$

Nach dem Recht von Tennessee verjähren Wechsel nach sechs Jahren, nach dem Recht des Forums nach drei Jahren. Das Gericht entscheidet für den Kläger, weil das deutsche materielle Verjährungsrecht, da nicht \textsl{lex causae}, unanwendbar sei, ebenso wie das amerikanische Verjährungsrecht, weil dieses dort als prozessual angesehen werde und das deutsche Gericht nicht amerikanisches Prozeßrecht anwenden könne.}
\end{fragmentpart}
\begin{fragmentpart}{Anmerkung}
Lm fügt ein Detail hinzu und bringt mit einer minimalen Änderung (\textquotedbl{}obwohl\textquotedbl{} anstelle von \textquotedbl{}weil\textquotedbl{}) eine andere Bewertung unter. Kein Verweis auf Weber.
\end{fragmentpart}
\end{fragment}
\phantomsection{}
\belowpdfbookmark{Fragment 64 3--13,~101--116}{Lm-Fragment-064-03}
\hypertarget{Lm-Fragment-064-03}{}
\begin{fragment}
\begin{fragmentpart}{Dissertation S.~64 Z.~3--13,~101--116 (BauernOpfer)}
\enquote{Als dritter Vertreter sei schließlich COOK genannt, der von FALCONBRIDGE und ROBERTSON grundlegend abweicht und deshalb eine originelle Position unter den englischsprachigen Autoren einnimmt.$[$FN 184$]$ Damit erscheint er als ein Fundamentalkritikers$[$sic$]$.$[$FN 185$]$ COOK begreift die Qualifikation, wie auch früher FRANKENSTEIN,$[$FN 186$]$ als ein Problem sprachlicher Art.$[$FN 187$]$ Besonders in bezug auf die \textsl{secondary characterization} greift er die Frage der Unterscheidung zwischen materiellem Recht und Prozeßrecht auf und bemerkt, daß der Begriff des Materiellrechtlichen und des Prozeßrechtlichen in jedem Land für je unterschiedliche Zwecke durchaus unterschiedlich gebraucht wird.$[$FN 188$]$ Im Ergebnis führt sein sprachlicher Ansatz zu einer Verselbständigung der kollisionsrechtlichen Begriffe vom eigenen oder fremden Sachrecht.$[$FN 189$]$

$[$FN 184$]$ Cf. besonders COOK, Characterization in the Conflict of Laws, Yale L.J. 51 (1941), S. 191-212. Der Aufsatz gehört zu der Aufsatzreihe, die er in ergänzter Form 1942 zu dem Buch \textsl{The Logical and Legal Bases of the Conflict of Laws} (Cambridge 1942) zusammenfügte.

$[$FN 185$]$ Cf. Weber, a.a.O., S. 177 Fn. 300.

$[$FN 186$]$ FRANKENSTEIN, Internationales Privatrecht, Bd. I, Berlin 1926, S. III, 279.

$[$FN 187$]$ „An examination of the history of human thought, whether in the field of philosophy,of logic or of science, $[$...$]$\textquotedbl{} Cook, a.a.O. (Fn. 184), S. 191.

$[$FN 188$]$ Cook, a.a.O. (Kapitel I, Fn. 35), S. 333-358 (347-353).

$[$FN 189$]$ Im bezug auf Anknüpfungspunkte (Domizil), cf. Cook, a.a.O. (Fn. 184), S. 211 f. und bezüglich der Autonomie der Unterscheidung zwischen materiellrechtlichen und prozeßrechtlichen Begriffen cf. Cook, a.a.O. (Fn. 188), S. 349-356.}
\end{fragmentpart}
\begin{fragmentpart}{Original \cite[S.~177--179 Z.~10--15,~19--20,~34--36,~1--2]{Weber-1986}}
\enquote{$[$S. 177$]$

Als Vertreter von gegenüber FALCONBRIDGE und ROBERTSON grundlegend abweichenden oder originellen Positionen unter den englischsprachigen Autoren sei abschließend Walter Wheeler COOK genannt.

Bedeutsam ist vor allem COOKS$[$FN 298$]$ seit 1919 erschienene Aufsatzfolge zu internationalprivatrechtlichen Themen, die er in ergänzter Form 1942 zum Buch \textsl{The Logical and Legal Bases of the Conflict of Laws} zusammenfügte. $[$...$]$

$[$...$]$ nimmt Cook die Position des Fundamentalkritikers ein$[$FN 300$]$. $[$...$]$ Er ist damit der erste nach FRANKENSTEIN$[$FN 302$]$, der die Qualifikationsproblematik als im Kern sprachlicher Art begreift. Hauptangriffspunkt COOKS ist die \textsl{secondary characterization} $[$...$]$

Der Begriff des Materiellrechtlichen und des Prozeßrechtlichen zum Beispiel werde in jedem Land für je unterschiedliche Zwecke durchaus unterschiedlich gebraucht.

$[$S. 178$]$

In praktischer Hinsicht führt COOKS sprachlicher Ansatz $[$...$]$ zur Verselbständi-

$[$S. 179$]$

gung der kollisionsrechtlichen Begriffe vom (eigenen oder fremden) Sachrecht$[$FN 308$]$.


$[$FN 298$]$ Zu seiner Person: $[$CULP$]$, S. 71.

$[$FN 300$]$ Denn auch im „intellectual garden“, so COOK, einen Satz von Gilbert LEWIS aufnehmend, sei \textsl{„the removal of the weeds $[$. . .$]$ as constructive in effect as the planting and cultivation of the useful vegetables“} (S. IX). Von CAVERS wird ihm mit Recht entgegengehalten: \textsl{„This $[$. . .$]$ assumes that we have vegetables to plant and planters to plant them. $[$. . .$]$ suspicions begin to grow $[$. . .$]$ that our manpower is badly allocated, that all our gardeners are weeders and none planters.\textquotedbl{}} (CAVERS, Bespr. Cook, Bases, S. 1172f).

$[$FN 302$]$ S. o. S. 94ff.

$[$...$]$

$[$FN 308$]$ So, deutlicher noch als bei COOK, auch z. B. HANCOCK,, Torts, S. 70, 72 (mit Bezug auf COOK, ohne Erwähnung Rabels).}
\end{fragmentpart}
\begin{fragmentpart}{Anmerkung}
Der Verweis auf Fußnote 185 in Weber (1986) ist ein echter Verweis. Er lässt aber nicht erkennen, dass der gesamte hier dargestellte, mit zahlreichen weiteren Fußnoten belegte Text sinngemäß und zu einem erheblichen Teil auch wörtlich von Weber übernommen wurde.
\end{fragmentpart}
\end{fragment}
\phantomsection{}
\belowpdfbookmark{Fragment 66 1--6}{Lm-Fragment-066-01}
\hypertarget{Lm-Fragment-066-01}{}
\begin{fragment}
\begin{fragmentpart}{Dissertation S.~66 Z.~1--6 (Verschleierung)}
\enquote{$[$Sein Fern$]$ziel ist das selbständige Grundkonzept der Internationalisten eines „allgemeingültigen Kollisionsrechts“.$[$FN 4$]$ Der Weg liegt für ihn nicht in der \textsl{a priori} Bestimmung oder in der Ableitung solcher Kollisionsnormen, sondern in der Auslegung und Ausformung vorhandenen Kollisionsrechts mit Hilfe der Rechtsvergleichung. Damit soll der Inhalt von einzelnen Begriffen in Normen des Kollisionsrechts vom Sachrecht gelöst werden.$[$FN 5$]$

$[$$[$FN 4$]$ RABEL, RabelsZ 1931, S. 287 a.E.: „Noch sind, abgesehen von den wenigen großen Prinzipienverschiedenheiten, die meisten Sätze der heutigen internationalen Privatrechte jung und weich genug, daß die Wirkung solcher Bemühung ohne weiteres auf Internationalisierung hingeht. ... Erlösen wir die Kollisionsrechte aus den Fesseln der lex fori, so werden sie sich dank der Rechtsvergleichung einander anpassen.“

$[$FN 5$]$ Ibid., S. 249, 256f.; vgl auch HUSSERL, JZ 1956, S. 434.$]$}
\end{fragmentpart}
\begin{fragmentpart}{Original \cite[S.~115 Z.~12--17]{Weber-1986}}
\enquote{RABELS Fernziel ist der alte Traum der Internationalisten: das allgemein gültige Kollisionsrecht$[$FN 64$]$. Der Weg dorthin liegt für ihn aber nicht in aprioristischer Bestimmung oder Ableitung solcher Kollisionsnormen$[$FN 65$]$, sondern in der Auslegung des Kollisionsrechts gemäß dem mit Hilfe der Rechtsvergleichung zu findenden Inhalt der Begriffe nach der Loslösung des Kollisionsrechts vom Sachrecht.

$[$$[$FN 64$]$ S. Q 72 = Z 287.$]$}
\end{fragmentpart}
\begin{fragmentpart}{Anmerkung}
Lm hat die von Weber beschriebene Quelle offensichtlich selbst eingesehen (Fußnoten 4 und 5). Lm legt aber nicht offen, dass Gedankengang und Formulierung von Weber übernommen werden, auf den in diesem Zusammenhang nicht verwiesen wird.
\end{fragmentpart}
\end{fragment}
\phantomsection{}
\belowpdfbookmark{Fragment 71 7--17,~113--115}{Lm-Fragment-071-06}
\hypertarget{Lm-Fragment-071-06}{}
\begin{fragment}
\begin{fragmentpart}{Dissertation S.~71 Z.~7--17,~113--115 (BauernOpfer)}
\enquote{NEUNERs Werk ist kein Diskussionsbeitrag innerhalb der Lliteratur zur Qualifikation, sondern eine Fundamentalkritik an ihr.$[$FN 31$]$ NEUNER dekuvriert die Qualifikationstheorien als „ein$[$en$]$ Versuch, durch begriffliche Operationen alle Probleme zu lösen, welche Auslegung und Anwendung der Kollisionsnorm erwachsen lassen“.$[$FN 32$]$ Er selbst hält es jedoch für „unmöglich, in deduktiver Weise aus $[$den Kollisionsnormen$]$ für jeden Fall eine Lösung zu erschließen“,$[$FN 33$]$ aus begrifflichen Systematisierungen normative Folgerungen zu ziehen. Daher unternimmt er den Versuch, auf anderem Wege die „Schwierigkeiten zu überwinden, welche die Qualifikationstheorie durch begriffliche Operationen zu beseitigen trachtet“.$[$FN 34$]$ Dieser andere Weg bestehe in der „richtige$[$n$]$ Fassung des Sinnes der Kollisionsnorm“.$[$FN 35$]$
----

$[$FN 31$]$ Cf. WEBER, a.a.o, S. 120-123.

$[$FN 32$]$ NEUNER, Sinn ..., S. 131. $[$...$]$

$[$FN 33$]$ Ibid., S. 131.

$[$FN 34$]$ Ibid., S. 5 (im Vorwort).

$[$FN 35$]$ Ibid., S. 5 (erster Satz des Vorwortes).}
\end{fragmentpart}
\begin{fragmentpart}{Original \cite[S.~121 Z.~1--3,~15--22,~22--24,~110--112]{Weber-1986}}
\enquote{NEUNERs Werk ist kein (weiterer) Diskussionsbeitrag innerhalb der Literatur zur Qualifikation, sondern eine Fundamentalkritik an ihr. $[$...$]$

Die Qualifikationstheorie ist für NEUNER „ein Versuch, durch begriffliche Operationen alle Probleme zu lösen, welche Auslegung und Anwendung der Kollisionsnorm erwachsen lassen“$[$FN 99$]$. Er hält es jedoch für „unmöglich, in deduktiver Weise aus $[$der Kollisionsnorm$]$ für jeden Fall eine Lösung zu erschließen“$[$FN 99$]$, aus begrifflichen Systematisierungen normative Folgerungen
zu ziehen$[$FN 100$]$. Daher befaßt sich NEUNER $[$...$]$ mit dem Versuch, auf anderem Wege die „Schwierigkeiten zu überwinden, welche die Qualifikationstheorie durch begriffliche Operationen zu beseitigen trachtet“$[$FN 102$]$.

Dieser andere Weg bestehe in der „richtige$[$n$]$ Fassung des Sinnes der Kollisionsnorm“$[$FN 103$]$,

----

$[$FN 99$]$ NEUNER, Sinn, S. 131.

$[$FN 102$]$ S. 5.

$[$FN 103$]$ S. 5; $[$...$]$}
\end{fragmentpart}
\begin{fragmentpart}{Anmerkung}
Nicht nur die Zitate stimmen überein, was an sich vielleicht nicht erwähnenswert wäre. Hier sind aber auch weite Teile der Zwischentexte identisch. Weber wird in $[$FN 121$]$ eingangs erwähnt.
\end{fragmentpart}
\end{fragment}
\phantomsection{}
\belowpdfbookmark{Fragment 75 1--2}{Lm-Fragment-075-01}
\hypertarget{Lm-Fragment-075-01}{}
\begin{fragment}
\begin{fragmentpart}{Dissertation S.~75 Z.~1--2 (Verschleierung)}
\enquote{$[$Er ist der Ansicht, daß es keine einheitliche Qualifikations-$]$methode für alle Fälle gäbe. Das liege letzlich an der Struktur der Kollisionsnormen.$[$58$]$

$[$58$]$ Im allgemeinen sei nach der lex fori zu qualifizieren, gelegentlich aber auch gemäß der lex causae (Ibid, S. 77-84.). Wenn eine Kollisionsnorm so beschaffen sei, daß sie ihre Voraussetzungen und ihre Wirkungen zugleich bezeichne, dann sei nach der lex causae zu qualifizieren (Id., S. 82f.)}
\end{fragmentpart}
\begin{fragmentpart}{Original \cite[S.~147 Z.~10--12,~15--18]{Weber-1986}}
\enquote{Er ist der Ansicht, daß es keine einheitliche Qualifikationsmethode für alle Fälle gibt. Im allgemeinen sei nach der lex fori zu qualifizieren, gelegentlich aber auch gemäß der lex causae$[$78$]$: $[$...$]$ Je nach Struktur der Kollisionsnorm$[$80$]$ sei aber anders zu entscheiden. Wenn eine Kollisionsnorm so beschaffen sei, daß sie ihre Voraussetzungen und ihre Wirkungen zugleich bezeichne, dann sei nach der lex causae zu qualifizieren.

$[$78$]$ Lewald, Règles, S. 77f.

$[$80$]$ Kritisch dazu Sauser-Hall, règles, S. 45f.}
\end{fragmentpart}
\begin{fragmentpart}{Anmerkung}
Webers Text wird gerafft und zum Teil in eine Fußnote verlagert. Weber wird in diesem Zusammenhang nicht erwähnt.
\end{fragmentpart}
\end{fragment}
\phantomsection{}
\belowpdfbookmark{Fragment 75 11--17}{Lm-Fragment-075-11}
\hypertarget{Lm-Fragment-075-11}{}
\begin{fragment}
\begin{fragmentpart}{Dissertation S.~75 Z.~11--17 (BauernOpfer)}
\enquote{Was die Qualifikationslehre angeht, bleibt WOLFF bei seiner Lehre von der Qualifikation nach der lex causae. Diese Lehre geht auf die internationalistische Schule zurück und schließt sich der Theorie von DESPAGNET an.$[$FN 64$]$ In diesem Zusammenhang diskutiert er mehrere Fälle, darin erstmals ausführlich den \textsl{Tennessee-Wechsel}-Fall, wie dieser vom Reichsgericht entschieden worden ist.$[$FN 65$]$ (Nicht-Anwendung der lex causae-Lehre, wie man meinen sollte, da das Gericht ja das anwendbare deutsche Prozeßrecht $[$über den Umfang der anzuwendenden Prozeßrechtsnormen und entsprechend das als Vertragsstatut berufene amerikanische Recht über den Umfang der anzuwendenden amerikanischen materiellen Wechselrechtsnormen bestimmen ließ).$]$

$[$FN 64$]$ Ibid., IPR, S. 48-50; PIL, S. 154-160; DESPAGNET, Clunet 1898, S. 261ff. und 272ff.

$[$FN 65$]$ Cf. supra (Kapitel III,, Fn. 169) und WOLFF, ibid., PIL, S. 161f.}
\end{fragmentpart}
\begin{fragmentpart}{Original \cite[S.~148 Z.~20--28]{Weber-1986}}
\enquote{Was die Qualifikationslehre angeht, so bleibt WOLFF bei seiner Lehre von der Qualifikation nach der \textsl{lex causae}. In diesem Zusammenhang diskutiert er erstmals ausführlich den Tennessee-Wechsel-Fall$[$FN 91$]$ wie er in RGZ 71, 21 entschieden worden war$[$FN 92$]$ – ein Musterbeispiel früher Anwendung der \textsl{lex causae} -Methode, wie man meinen sollte, da das Gericht ja das anwendbare deutsche Prozeßrecht über den Umfang der anzuwendenden Prozeßrechtsnormen bestimmen ließ und entsprechend das als Vertragsstatut berufene amerikanische Recht über den Umfang der anzuwendenden amerikanischen materiellen Wechselrechtsnormen. 
 
$[$FN 91$]$ Im deutschen Buch ist die Frage der Qualifikation bei angloamerikanischen Verjährungsregeln nur kurz – mit unklarer Tragweite – angetippt (s. o. S. 129).

$[$FN 92$]$ S. o. S. 28f.}
\end{fragmentpart}
\begin{fragmentpart}{Anmerkung}
Weber wird zuvor in Fußnote 63 genannt \textquotedbl{}Kritisch zu WOLFF, WEBER; a.a.O., S. 148f.\textquotedbl{} Alternativ ein verschärftes Bauernopfer, oder eine Verschleierung.
\end{fragmentpart}
\end{fragment}
\phantomsection{}
\belowpdfbookmark{Fragment 76 1--9}{Lm-Fragment-076-01}
\hypertarget{Lm-Fragment-076-01}{}
\begin{fragment}
\begin{fragmentpart}{Dissertation S.~76 Z.~1--9 (BauernOpfer)}
\enquote{$[$(Nicht-Anwendung der lex causae-Lehre, wie man meinen sollte, da das Gericht ja das anwendbare deutsche Prozeßrecht$]$ über den Umfang der anzuwendenden Prozeßrechtsnormen und entsprechend das als Vertragsstatut berufene amerikanische Recht über den Umfang der anzuwendenden amerikanischen materiellen Wechselrechtsnormen bestimmen ließ). Daher überrascht es nicht, daß WOLFF nun diese Entscheidung als unbefriedigend, ja als absurd und fehlerhaft, bezeichnet.$[$FN 66$]$ Der Fehler liege darin, daß das RG nicht gesehen habe, daß der englische Begriff \textsl{„procedure“} weiter sei als der deutsche Begriff Prozeß, und daß es deswegen dem deutschen Gericht nicht verwehrt gewesen wäre, die amerikanische Verjährungsvorschrift anzuwenden.$[$FN 67$]$

$[$FN 66$]$ Ibid., Der Absatz hat den Titel \textsl{„absurd results“}. Cf. ROBERTSON, Characterization, S. 252f.

$[$FN 67$]$ Ibid., S. 162; cf. a.a.O. (Fn. 65). „English courts use the word 'procedure' in a wider sense than that in which French, Italian and German law speak of procédure, precedura, procedimento, Prozess“. Vgl. MENDELSSOHN BARTHOLDY, B.Yb.I.L. 1935, passim.}
\end{fragmentpart}
\begin{fragmentpart}{Original \cite[S.~148--149 Z.~23--28,~1--2]{Weber-1986}}
\enquote{$[$Seite 148$]$

... ein Musterbeispiel früher Anwendung der \textsl{lex causae}-Methode, wie man meinen sollte, da das Gericht ja das anwendbare deutsche Prozeßrecht über den Umfang der anzuwendenden Prozeßrechtsnormen bestimmen ließ und entsprechend das als Vertragsstatut berufene amerikanische Recht über den Umfang der anzuwendenden amerikanischen materiellen Wechselrechtsnormen. Daher überrascht es, daß WOLFF nun 

$[$S. 149$]$

diese Entscheidung als überraschend und unbefriedigend, ja als absurd und
fehlerhaft$[$93$]$ bezeichnet. 

Der Fehler liege darin, daß das deutsche Gericht nicht gesehen habe, daß der englische Begriff \textsl{procedure} weiter sei als der deutsche Begriff \textsl{Prozeß}, und daß es deswegen dem deutschen Gericht nicht verwehrt gewesen wäre, die amerikanische Verjährungsvorschrift anzuwenden.

$[$93$]$ WOLFF, PIL, 1. Aufl., S. 162f; ähnliche Schwierigkeiten mit dieser Entscheidung hat auch ROBERTSON, s. u. S. 174.}
\end{fragmentpart}
\begin{fragmentpart}{Anmerkung}
Fortführung von \hyperlink{Lm-Fragment-075-11}{Fragment\_075\_11}. Weber wird zuvor in Fußnote 63 auf der Vorderseite genannt: \textquotedbl{}Kritisch zu WOLFF, WEBER; a.a.O., S. 148f.\textquotedbl{} Alternativ ein verschärftes Bauernopfer, oder eine Verschleierung. 
Durch Verschieben des Wortes \textquotedbl{}nicht\textquotedbl{} von hinter \textquotedbl{}überrascht es\textquotedbl{} zu vor \textquotedbl{}Anwendung der lex causae\textquotedbl{} dreht Lm mit einem minimalen Eingriff in den Text die Bewertung von Weber herum.
\end{fragmentpart}
\end{fragment}
\phantomsection{}
\belowpdfbookmark{Fragment 77 9--20}{Lm-Fragment-077-09}
\hypertarget{Lm-Fragment-077-09}{}
\begin{fragment}
\begin{fragmentpart}{Dissertation S.~77 Z.~9--20 (BauernOpfer)}
\enquote{VON STEIGER hat eine vom \textsl{common sense} geprägte unprätentiöse Studie geschrieben, die in klarer Sprache versucht, das Qualifikationsproblem aus der Sicht und mit der Fragestellung derer zu durchleuchten, bei denen solche Probleme in der Praxis auftauchen können, des Richters, des Anwalts, des Rechtssuchenden. Dabei geht es VON STEIGER mehr um Praktikabilität und Rechtssicherheit als um die theoretische Deduktion von höheren Begriffen als dem einen, oder um die Suche nach der für den Einzelfall gerechtesten aller Lösungen als dem anderen Extrem.$[$74$]$ Bezeichnend ist, dass VON STEIGER der praktischen Überprüfung seiner Methode an konkreten Problembereichen fast die doppelte Seitenzahl einräumt wie der Formulierung und Begründung dieser Methode.$[$75$]$ Hauptthema seiner Habilitation sind der Gegenstand und die Methode der Qualifikation. 

$[$74$]$ So WEBER, a.a.O., S. 159.

$[$75$]$ VON STEIGER, a.a.O. (Fn. 73). Im Allgemeinen Teil diskutiert er die Theorie (S. 1-73) und die Praxis des (schweizerischen) Bundesgerichts (S. 73-79), während er im Besonderen teil seine Ergebnisse anhand von konkreten Fallgruppen überprüft: I. Privatrecht – Prozeßrecht, S. 80-107, II. Form – Inhalt $[$der Rechtsgeschäfte$]$, s. 108-125, III. Handlungsfähigkeit, s. 125-134, IV. Die persönlichen Wirkungen der Ehe, S. 134-147, v. Güterrecht – Erbrecht, S. 147-168, VI. Ehe und Ehescheidung, S. 169-186, VII. Sachenrecht, S. 186-194.}
\end{fragmentpart}
\begin{fragmentpart}{Original \cite[S.~159--160 Z.~17--28]{Weber-1986}}
\enquote{Diese$[$162$]$ ist eine von \textsl{common sense} geprägte unprätentiöse Arbeit, die in klarer Sprache versucht, das Qualifikationsproblem aus der Sicht und mit der Fragestellung derer zu durchleuchten, bei denen solche Probleme in der Praxis auftauchen können: dem Richter, dem Anwalt, dem Rechtssuchenden. Es geht STEIGER dabei mehr um Praktikabilität und Rechtssicherheit als um theoretische Deduktionen von höheren Begriffen als dem einen oder die Suche nach der für den Einzelfall gerechtesten aller Lösungen als dem anderen Extrem. Bezeichnend ist, daß STEIGER der praktischen Überprüfung seiner Methode an konkreten Problembereichen fast die doppelte Seitenzahl einräumt wie der Formulierung und Begründung dieser Methode.

Bei STEIGER steht erstmals in einer großen Arbeit zur Qualifikation nicht die Frage nach dem Qualifikationsstatut im Vordergrund, sondern die nach $[$S. 160$]$ Qualifikationsgegenstand und Qualifikationsmethode. 

$[$162$]$ STEIGER, Rechtsfrage.}
\end{fragmentpart}
\begin{fragmentpart}{Anmerkung}
Die Quelle wird als Beleg für die ersten zwei Sätze zitiert. Die wörtlichen Übernahmen werden aber nicht gekennzeichnet, ebensowenig die nachfolgende enge Anlehnung an den Gedankengang und Wortlaut.
\end{fragmentpart}
\end{fragment}
\phantomsection{}
\belowpdfbookmark{Fragment 80 1--17,~107--111}{Lm-Fragment-080-01}
\hypertarget{Lm-Fragment-080-01}{}
\begin{fragment}
\begin{fragmentpart}{Dissertation S.~80 Z.~1--17,~107--111 (BauernOpfer)}
\enquote{Am wertvollsten für den Gegenstand dieser Studie erscheint in NIEDERERS Monographie die detaillierte Untersuchung von Struktur und Wesen der Kollisionsnorm, ein Aspekt, dem sich die Autoren des Qualifikationsproblems seit Beginn der 1930er Jahre in wachsendem Ausmaß zugewandt haben. Hier geht es zentral um den Vorgang des \textsl{Qualifizierens} und des \textsl{Subsumierens} im allgemeinen. Das Qualifikations\textsl{problem} stellt sich für NIEDERER nur als Schwierigkeit bei der Definition kollisionsrechtlicher Begriffe dar. Daraus kann als Qualifikations\textsl{konflikt} die Situation entstehen, daß unterschiedliche Definitionen zu unterschiedlichen Ergebnissen führen.$[$FN 89$]$

NlEDERERS Arbeit markiert aufgrund klarer theoretischer Formulierung der „Funktion“ als Zentralbegriff des IPR einen gewissen Abschluß der kontinentaleuropäischen Ansätze zu einer Stufenqualifikation. Außerdem ist NIEDERER der erste, der in einer größeren Arbeit zur Qualifikation dem Begriff der Funktion, der ja schon bei KAHN,$[$FN 90$]$ danach aber nur vereinzelt aufgetaucht war, eine zentrale Position einräumt. Freilich geschieht dies erst am Ende seiner Arbeit auf wenigen Schlußseiten, so daß die sich daraus ergebenden Fragen kaum angesprochen werden.$[$FN 91$]$

----

$[$FN 89$]$ Ibid., S. 38ff., 49, 56. $[$...$]$

$[$FN 90$]$ Cf. WEBER, a.a.O., S. 32 m.w.N. RABEL hat das Thema ebenfalls (implizit) aufgegriffen, allerdings in seinem Aufsatz von 1931. Cf. supra, Fn. 10.

$[$FN 91$]$ NIEDERER, a.a.O. (Fn. 72), S. 163. Es muß vielleicht an dieser Stelle noch erwähnt werden, daß NIEDERER weder die angloamerikanische Literatur (mit der Ausnahme LORENZENs), noch die italienischen Lehren berücksichtigt, noch die Arbeiten von RAAPE.}
\end{fragmentpart}
\begin{fragmentpart}{Original \cite[S.~162,~163 Z.~25--32,~108--109,~14--21]{Weber-1986}}
\enquote{Wertvollster Teil von NIEDERERS Arbeit ist die detaillierte Untersuchung von Struktur und Wesen der Kollisionsnorm, einem Aspekt, dem sich die Autoren zu Qualifikationsproblemen seit Beginn der dreißiger Jahre in wachsendem Ausmaß zugewandt haben, sowie dem Vorgang des Qualifizierens und allgemein des Subsumierens$[$FN 186$]$. Als Qualifikations\textsl{problem} stellt sich dabei für ihn das Problem der Definition der kollisionsrechtlichen Begriffe dar, als daraus möglicher Qualifikations\textsl{konflikt} die Situation, daß unterschiedliche Definitionen zu unterschiedlichen Ergebnissen führen$[$FN 187$]$.

$[$S. 163$]$

NIEDERERs Arbeit markiert, durch seine klare theoretische Formulierung eines solchen Vorgehens, einen gewissen Abschluß der kontinentaleuropäischen Ansätze zu einer Stufenqualifikation. Außerdem ist er der erste, der in einer großen Arbeit zur Qualifikation dem Begriff der Funktion, der ja schon bei KAHN$[$FN 193$]$, danach aber nur ganz vereinzelt$[$FN 194$]$, aufgetaucht war, eine zentrale Position einräumt --- allerdings erst am Ende seiner Arbeit, auf wenigen Schlußseiten, so daß die sich nun daraus ergebenden Fragen kaum angesprochen sind$[$FN 195$]$.

----

$[$FN 186$]$ NIEDERER, Qualifikation, S. 38 ff.

$[$FN 187$]$ S. 49, 56.

$[$FN 193$]$ S. o. S. 32. 

$[$FN 194$]$ Z.B. o. S. 142

$[$FN 195$]$ Die Anwendung der Methode NIEDERERs auf ein Einzelgebiet versucht Widmer, Vertragsrecht. --- In späteren Arbeiten (Einführung, 1. Aufl., S. 248-252) sieht NIEDERER eine Konvergenz von flexibler lex fori-Qualifikation und praktikabler autonomer Qualifikation. $[$...$]$}
\end{fragmentpart}
\begin{fragmentpart}{Anmerkung}
Quelle wird mittig (Fn 90) genannt. Die Übereinstimmungen gehen jedoch bis in den Schriftsatz (dieselben Teilworte wurden kursiv gesetzt).
\end{fragmentpart}
\end{fragment}
\phantomsection{}
\belowpdfbookmark{Fragment 80,~81 21--22,~1--2}{Lm-Fragment-080-21}
\hypertarget{Lm-Fragment-080-21}{}
\begin{fragment}
\begin{fragmentpart}{Dissertation S.~80,~81 Z.~21--22,~1--2 (Verschleierung)}
\enquote{RAAPE entwickelt in seinem Lehrbuch,$[$FN 93$]$, im Staudinger Kommentar$[$FN 94$]$ und in seinem Haager Kurs$[$FN 95$]$ seine viel zitierte Drei-Fragen-Methode. RAAPE sucht $[$dabei nicht wie RABEL neue Wege, sondern differenzierte Lösungsansätze auf der Grundlage der herkömmlichen lex-fori-Qualifikation.$]$

----
$[$FN 93$]$ RAAPE, Deutsches IPR, 1. Aufl., Bd. 1, Berlin 1938, 5. Aufl., Berlin – Frankfurt a.M. 1961, S. 102-116, 6. Aufl. von STURM neubearbeitet, München 1977, S. 275-285.

$[$FN 94$]$ A.a.O. (Fn. 71), S. 15-20.

$[$FN 95$]$ RAAPE, Les rapports juridiques entre parents er $[$sic$]$ enfants comme point de départ d’une explication pratique d’anciens er $[$sic$]$ de nouveaux problèmes fondamentaux du droit international privé, RCADI 50 (1934), S. 401-544 (517-537). Interessanterweise benutzt er für das Problem folgende Bezeichnung : Le problème du classement (Einreihungsproblem) (Qualification).}
\end{fragmentpart}
\begin{fragmentpart}{Original \cite[S.~139,~140 Z.~1--5]{Weber-1986}}
\enquote{Darin präzisiert er seine im Kommentar von 1931 teilweise noch vage 

$[$S. 140$]$

gebliebenen Anschauungen$[$FN 21$]$, entwickelt diese fort und bringt sie mit seiner später viel zitierten ‚Drei-Fragen-Methode‘ zu einem gewissen Abschluß$[$FN 22$]$.
RAAPE sucht dabei nicht wie RABEL ganz neue Wege, sondern differentzierte Lösungansätze auf der Grundlage der herkömmlichen lex fori-Qualifikation.

----
$[$FN 21$]$ S. o S. 123f.

$[$FN 22$]$ In den weiteren Auflagen seines Lehrbuches bleibt das Kapitel zur Qualifikation praktisch unverändert, bis hin zur 5. Auflage von 1961.}
\end{fragmentpart}
\begin{fragmentpart}{Anmerkung}
Die Quelle wird hier nicht im Zusammenhang mit RAAPE genannt --- auf S. 80 im Zusammenhang mit NIEDERER, auf S. 82 im Zusammenhang mit WENGLER.

Der größere Teil des von der Quelle übernommenen Textes befindet sich auf der nachfolgenden Seite.
\end{fragmentpart}
\end{fragment}
\phantomsection{}
\belowpdfbookmark{Fragment 81 1--7}{Lm-Fragment-081-01}
\hypertarget{Lm-Fragment-081-01}{}
\begin{fragment}
\begin{fragmentpart}{Dissertation S.~81 Z.~1--7 (Verschleierung)}
\enquote{$[$RAAPE sucht$]$ dabei nicht wie RABEL neue Wege, sondern differenzierte Lösungsansätze auf der Grundlage der herkömmlichen lex fori-Qualifikation.

Für ihn steht das Auffinden einer praktischen Methode im Vordergrund.$[$FN 96$]$ Qualifikation ist für RAAPE die Ermittlung der Beschaffenheit einer Sachnorm unter dem Blickwinkel der Kollisionsnorm. Qualifikationsgegenstand ist demnach, und das stellt einen deutlichen Unterschied zu RABELS Lehre dar, stets eine Sachnorm.$[$FN 97$]$

$[$FN 96$]$ Raape, IPR5, S. 107.

$[$FN 97$]$ Ibid., S. 109f., lllf.}
\end{fragmentpart}
\begin{fragmentpart}{Original \cite[S.~140 Z.~3--5,~7--8,~11--14]{Weber-1986}}
\enquote{RAAPE sucht dabei nicht wie RABEL ganz neue Wege, sondern differenziertere Lösungsansätze auf der Grundlage der herkömmlichen \textsl{lex fori}-Qualifikation.

$[$...$]$ für ihn steht das Finden einer praktikablen Methode im Vordergrund. $[$...$]$ Qualifikation nun ist für RAAPE die Ermittlung der Beschaffenheit einer Sachnorm unter dem Blickwinkel der Kollisionsnorm, Qualifikationsgegenstand demnach --- und das macht einen deutlichen Unterschied zu RABELS Lehre$[$FN 25$]$ --- stets eine Sachnorm$[$FN 26$]$. 

$[$FN 26$]$ RAAPE, IPR, Bd. 1, 1. Aufl., S. 69.}
\end{fragmentpart}
\begin{fragmentpart}{Anmerkung}
Fast wortwörtlich übereinstimmend aber dennoch ohne Quellenverweis.
\end{fragmentpart}
\end{fragment}
\phantomsection{}
\belowpdfbookmark{Fragment 82 14--19}{Lm-Fragment-082-14}
\hypertarget{Lm-Fragment-082-14}{}
\begin{fragment}
\begin{fragmentpart}{Dissertation S.~82 Z.~14--19 (BauernOpfer)}
\enquote{Er unterscheidet verschiedene Arten von Qualifikationskonflikten, den \textsl{Enumerationskonflikt} und den \textsl{Subsumtionskonflikt}.$[$102$]$ In einem Aufsatz problematisiert er als erster das heikle Verhältnis der Vorfrage zur Qualifikation.$[$103$]$ Das Verhältnis von \textsl{renvoi} und \textsl{ordre public} zur Qualifikation hatte seit KAHN und BARTIN immer wieder das Interesse der Internationalprivatrechtler gefunden.$[$104$]$

$[$103$]$ Cf. WEBER, a.a.O., S. 141f.

$[$104$]$ Auf diese Probleme kann die vorliegende Arbeit nicht näher eingehen. Cf. aber z.B. KEGEL, a.a.O. (Kapitel I, Fn. 5), passim.}
\end{fragmentpart}
\begin{fragmentpart}{Original \cite[S.~141--142 Z.~28--29,~10--14]{Weber-1986}}
\enquote{In letzterem unterscheidet er verschiedene Arten von Qualifikationskonflikten wie ,Enumerationskonflikte' und 'Subsumtionskonflikte'$[$37$]$ $[$...$]$

$[$S. 141$]$

$[$...$]$

Sein Aufsatz zur Vorfrage ist der erste überhaupt, in dem das Verhältnis von Vorfrage zu Qualifikation problematisiert wird$[$43$]$. Das Verhältnis von Rückverweisung und ordre public zur Qualifikation hatte seit KAHN und BARTIN und einschließlich dieser beiden$[$44$]$ immer wieder das Interesse der Internationalprivatrechtler gefunden.


$[$37$]$ Wengler, Doppelbesteuerung, S. 12.

$[$43$]$ Nachdem MELCHIOR in seinem Lehrbuch von 1932 (s. o. S. 124ff) das Thema ,Vorfrage' überhaupt erstmals (und zwar international, vgl. z. B. Morris, Bespr. Breslauer, Succession, S. 611 und Loussouarn, Dr. I. P., ab 8. Aufl., --- dort S. 422 -; s. zur genauen Chronologie aber auch Wengler, IPR, 12. Aufl., Bd. VI, 1, S. Xf) als eigenständiges Problem herausgearbeitet und ausführlich dargestellt hatte; s. aber auch schon Bar, Theorie, Bd. 1, S. 112; Kahn, Gesetzeskollisionen, S. A 87 = J 101.

$[$44$]$ S. o. S. 25f, 72, 74f.}
\end{fragmentpart}
\begin{fragmentpart}{Anmerkung}
Weber wird als weiterführender Verweis für den zweiten Satz angegeben. In FN 102 wird Weber nicht erwähnt.
\end{fragmentpart}
\end{fragment}
\phantomsection{}
\belowpdfbookmark{Fragment 85 5--20}{Lm-Fragment-085-05}
\hypertarget{Lm-Fragment-085-05}{}
\begin{fragment}
\begin{fragmentpart}{Dissertation S.~85 Z.~5--20 (BauernOpfer)}
\enquote{Neben den bereits erwähnten Autoren$[$FN 16$]$ ist an dieser Stelle noch AGO zu erwähnen.$[$FN 17$]$ An AGOS Buch fällt zunächst die bewunderungswürdige Beherrschung und Verarbeitung der ausländischen, insbesondere der deutschen, Literatur auf. Was nun die Qualifikation angeht, so leitet er aus seinem System Folgerungen ab, die sich im Ergebnis wenig von der in Italien schon seit ANZILOTTI vertretenen Stufenqualifikation$[$FN 18$]$ mit dem Primat der lex fori unterscheiden.$[$FN 19$]$ Allerdings findet man bei ihm schon vier Stufen, in die er das Problem der Auslegung (einschließlich der Frage nach einem Renvoi) eines internationalprivatrechtlichen Falles eingliedert.$[$FN 20$]$ Die Bestimmung des Anknüpfungpunktes sieht er nicht als Problem der Qualifikation; dennoch sei die Bestimmung der Qualifikation und der Anknüpfung nach ähnlichen Kriterien, insbesondere im Sinne der lex fori vorzunehmen.$[$FN 21$]$ AGO differenziert anhand der Staatsangehörigkeit zwischen der Auslegung des als Anknüpfungspunkt der Kollisionsnorm verwendeten Begriffes und der Feststellung, ob dieser Anknüpfungspunkt verwirklicht ist. Er zeigt auf, daß das alte Konzept einen einheitlichen Staatsangehörigkeitsbegriff voraussetzt, der mit dem der lex fori hinreichend übereinstimmt.$[$FN 22$]$

$[$FN 16$]$ Cf. supra, Kapitel III, Fn. 80, 85-87, 89, 91, 94-95, Kapitel IV, Fn. 21-26 und Kapitel IV 1 b.

$[$FN 17$]$ AGO, Teoria del diritto internazionale privato, Padova 1934, S. 135-220; wiederholt und zum Teil erweitert hat AGO seine Lehre in seinem Haager Kurs von 1935, Règles générales des conflits des lois, RCADI 58 (1936 IV), S. 243-469 (313-379). S. auch WEBER, a.a.O, S. 181-183.

$[$FN 18$]$ Cf. supra, Kapitel III, Fn. 86.

$[$FN 19$]$ AGO, teoria, a.a.O. (Fn 17), S. 145, 151f.

$[$FN 20$]$ AGO, RCADI, a.a.O. (Fn. 17), S. 314.

$[$FN 21$]$ AGO, teoria, a.a.O. (Fn. 17), S. 195.

$[$FN 22$]$ Ibid, S. 195f.; vgl. auch ROBERTSON, Characterization, S. 115.}
\end{fragmentpart}
\begin{fragmentpart}{Original \cite[S.~182--183 Z.~6--23,~183--1]{Weber-1986}}
\enquote{$[$Seite 182$]$

An AGOs Buch fällt zunächst die --- selbst für die allgemein gut informierten italienischen Autoren jener Zeit --- bewundernswürdige Beherrschung und Verarbeitung der ausländischen und insbesondere der deutschen Literatur auf$[$FN 326$]$.

Was nun die Qualifikation angeht, so leitet AGO aus seinem System, trotz dessen auffälliger Besonderheiten, Folgerungen ab, die sich im Ergebnis wenig von der in Italien schon seit ANZILOTTI vertretenen Stufenqualifikation mit dem Primat der \textsl{lex fori} für das Verständnis der Kollisionsnorm unterscheidet$[$FN 327$]$. Allerdings sind es bei AGO nun schon vier Stufen geworden, in die er das Problem der Auslegung (inclusive der Frage nach einem \textsl{renvoi}) beim internationalprivatrechtlichen Fall gliedert$[$FN 328$]$. Die Bestimmung des Anknüpfungspunktes, der er --- als einer der ersten seit  KAHN – ein eigenes Kapitel widmet, sieht AGO nicht als Problem der Qualifikation; dennoch sei es nach ähnlichen Kriterien, insbesondere im Sinne der \textsl{lex fori}, zu lesen$[$FN 329$]$. Eine bemerkenswerte generelle Differenzierung nimmt  AGO dabei vor zwischen der Auslegung des als Anknüpfungspunkt von der Kollisionsnorm verwendeten Begriffes und der Feststellung, ob dieser Anknüpfungspunkt verwirklicht ist. $[$...$]$ AGO zeigt$[$FN 330$]$, daß dies einen einheitlichen Staatsangehörigkeitsbegriff vor-

$[$Seite 183$]$
aussetzt, der --- und das ist das Entscheidende --- mit dem der lex fori übereinstimmt$[$FN 331$]$.

Wiederholt und zum Teil erweitert hat Ago seine Lehre in seinem Haager Kurs von 1935$[$FN 332$]$.

$[$FN 326$]$ Siehe auch seinen großen Besprechungsaufsatz zu den deutschen IPR-Neuerscheinungen der Jahre 1931-1933;  AGO, dottrina; zur Qualifikation dort S. 211-217; dazu wiederum s. RAAPE, Bespr. Ago, dottrina, S. 287f.

$[$FN 327$]$ AGO, teoria, S. 145, 151 f.

$[$FN 328$]$ AGO, règles, S. 314.

$[$FN 329$]$ AGO, teoria, S. 195.

$[$FN 330$]$ Unter Hinweis auf ANZILOTTI und CAVAGLIERI. Die meisten Autoren sehen das Problem gar nicht, vgl. z.B . ROBERTSON, Characterization, S. 115; RABEL, Conflict, S. 60.

$[$FN 331$]$ S. 195 f.

$[$FN 332$]$ AGO, Règles, dort besonders S. 313-379; siehe ferner AGO, prescrizione.}
\end{fragmentpart}
\begin{fragmentpart}{Anmerkung}
Die Quelle wird anfangs in einer Fußnote mit \textquotedbl{}cf\textquotedbl{} erwähnt. Der Text wird gerafft, ein Satz wird in der Fußnote 17 verarbeitet.
\end{fragmentpart}
\end{fragment}
\phantomsection{}
\belowpdfbookmark{Fragment 86 3--9}{Lm-Fragment-086-03}
\hypertarget{Lm-Fragment-086-03}{}
\begin{fragment}
\begin{fragmentpart}{Dissertation S.~86 Z.~3--9 (Verschleierung)}
\enquote{\textbf{3. Niederlande}

In den Niederlanden finden wir 1933 die Monographie von SALOMON,$[$FN 25$]$ der Erwägungen zum Recht und seiner Begrifflichkeit präsentiert.$[$FN 26$]$ SALOMON kommt zu dem Ergebnis, daß es die Qualifikation als einheitliche Lösungsmethode für einen bestimmten Problemtypus gar nicht gäbe, sondern daß sich dahinter verschiedene Probleme des Rechts an sich und des IPR im besonderen verbergen.$[$FN 27$]$

$[$25$]$ SALOMON, Het Qualificatieproblem $[$sic$]$ in het Internationaal Privaatecht $[$sic$]$, Amsterdam 1933 (150 S.)

$[$26$]$ Nachdem er die Qualifikationstheorien kurz beschrieben, ibid, S. 4-35, und die Qualifikation in der Problematik des IPR eingeordnet hat, ibid, S. 36-66, diskutiert er besonders die Begriffe \textsl{recht} (Recht) und \textsl{wet} (Gesetz) in ihrem Zusammenhang mit dem IPR, ibid, S. 37-94, sowie auch die Begriffsanpassung von Rechtsbegriffen und Rechtsnormen, ibid., S. 97-134.

$[$FN 27$]$ Ibid, S. 136f., I42f.}
\end{fragmentpart}
\begin{fragmentpart}{Original \cite[S.~136 Z.~25--31]{Weber-1986}}
\enquote{\textsl{2. Literatur aus den Niederlanden}

In den Niederlanden legt Salomon 1933 eine hunderfünfzigseitige $[$sic$]$ Universitätsschrift vor, in der er auf der Grundlage weit ausholender Erwägungen zum Recht und seiner Begrifflichkeit zum Ergebnis kommt, daß es \textsl{die} Qualifikation als einheitliche Lösungsmethode für einen bestimmten Problemtypus gar nicht gebe, sondern daß sich dahinter die verschiedenen Probleme des Rechts an sich und des IPR im besonderen verbergen$[$223$]$.

$[$223$]$ SALOMON, Qualificatieprobleem, S. 136f, 142 f.}
\end{fragmentpart}
\begin{fragmentpart}{Anmerkung}
Weber (1986) wird in diesem Abschnitt nicht erwähnt. Einzig Fußnote 26 unterscheidet Original und Nachahmung.

Strukturelle Parallele zu \hyperlink{Lm-Fragment-077-09}{Fragment\_077\_09} und \hyperlink{Lm-Fragment-054-16}{Fragment\_054\_16}, in denen Lm ebenfalls Webers Angaben in einer Fußnote mit einem Überblick zur Gliederung der jeweiligen Arbeit ergänzt.
\end{fragmentpart}
\end{fragment}
\phantomsection{}
\belowpdfbookmark{Fragment 91 3--10}{Lm-Fragment-091-03}
\hypertarget{Lm-Fragment-091-03}{}
\begin{fragment}
\begin{fragmentpart}{Dissertation S.~91 Z.~3--10 (BauernOpfer)}
\enquote{Seit Ende der 1920er Jahre war die Qualifikationstheorie als Problemfeld bekannt und zum kollisionsrechtlichen Gemeingut geworden. Resümierend läßt sich feststellen, daß der Diskussionsstand und die –ansätze seit 1905 unverändert geblieben waren. Für das Jahr 1945 läßt sich diese Aussage \textsl{cum grano salis} wiederholen: Qualifikation war nun zwar international als Grundthema des internationalen Privatrechts präsent, aber auf anderem Niveau als 20 Jahre zuvor: quantitativ und qualitativ erweitert und fortentwickelt, geographisch universal ausgedehnt.$[$1$]$

$[$1$]$ Cf. WEBER, a.a.O. (Fn. 1), S. 190-195 (190).}
\end{fragmentpart}
\begin{fragmentpart}{Original \cite[S.~190 Z.~5--10,~13--16]{Weber-1986}}
\enquote{Ende der zwanziger Jahre --- so habe ich im vorigen Kapitel geschrieben$[$383$]$, hatte das Thema \textsl{,Qualifikation}' sich durchgesetzt, war es zum internationalprivatrechtlichen Gemeingut geworden. 

Jetzt gilt es zu präzisieren: \textsl{,Qualifikation}' war zum Gemeingut geworden, aber auf dem Diskussionsstand von 1905. Nach den Schriften von KAHN und BARTIN, von DESPAGNET und GEMMA war wenig Neues hinzugekommen. $[$...$]$ 

Für das Jahr 1945 läßt sich die Aussage wiederholen: \textsl{‚Qualifikation}' war nun international als Grundthema des IPR präsent --- aber auf anderem Niveau als 15 Jahre zuvor, quantitativ und qualitativ erweitert und fortentwickelt, geographisch universal ausgedehnt.

$[$383$]$ O. S. 106.}
\end{fragmentpart}
\begin{fragmentpart}{Anmerkung}
Ein eindeutiger Verweis auf die Quelle in Fn. 1, der aber nicht erkennen lässt, dass der gesamte Gedankengang und auch dessen Formulierung mit nur leichten Bearbeitungen übernommen wurde.
\end{fragmentpart}
\end{fragment}
\phantomsection{}
\belowpdfbookmark{Fragment 93 14--27}{Lm-Fragment-093-14}
\hypertarget{Lm-Fragment-093-14}{}
\begin{fragment}
\begin{fragmentpart}{Dissertation S.~93 Z.~14--27 (Verschleierung)}
\enquote{RIGAUX schreibt, daß die Qualifikation lege causae positive und negative Konflikte zur Folge haben könne. Es seien Fälle denkbar, bei denen ein bestimmter nationaler Richter$[$14$]$ ebenso wie ein hypothetischer überstaatlicher
Beurteiler bei einer Qualifikation lege causae nicht in der Lage seien, die Entscheidung für eine bestimmte nationale Rechtsordnung zu treffen. Das gelte besonders für eine Gruppe von Fällen, in denen der Konflikt nicht sichtbar sei, weil lediglich eine einseitige Kollisionsnorm in Frage stehe. Ein Beispiel sei der Fall \textsl{Gourié},$[$15$]$ in dem die Anwendung des französischen Gesetzes vom 14. Juli 1819 davon abhing, ob ein unzweifelhaft fremdem Recht unterstehender Anspruch erbrechtlich oder ehegüterrechtlich zu qualifizieren war.$[$16$]$ Man denke hier zunächst nicht an einen Konflikt, weil ja nur eine lex causae für die Qualifikation in Betracht zu kommen scheine. Aber gerade der Fall \textsl{Gourié} zeige, daß auch in solchen Fällen Qualifikationskonflikte (\textsl{in casu} zwischen französischem Recht und dem Recht von Pennsylvania) auftreten können.

$[$14$]$ Cf. ibid., Rn. 311, S. 477-480 (479).

$[$15$]$ Cour de Paris 6.1.862, Recueil Sirey 1862, 2, 337; cf. BARTIN, Clunet 1897, S. 726f.; ibid., S. 49.

$[$16$]$ RIGAUX, a.a.O. (Fn. 11), Rn. 49; weitere Fälle dort, Rn. 50. RIGAUX spricht hier von Qualifikationsproblemen im Gegensatz zu Qualifikationskonflikten.}
\end{fragmentpart}
\begin{fragmentpart}{Original \cite[S.~55--57 Z.~28--29,~1--4,~32--34,~1--8]{Steindorff-1958}}
\enquote{Die Qualifikation lege causae kann demnach positive und negative Konflikte zur Folge haben$[$6$]$.

$[$S. 56$]$

Es erscheinen hier also Fälle, in denen ein bestimmter nationaler Richter$[$1$]$ ebenso wie ein hypothetischer überstaatlicher Beurteiler bei einer Qualifikation lege causae nicht in der Lage ist, die Entscheidung für eine bestimmte nationale Rechtsordnung zu treffen.

$[$...$]$ ferner in einer Gruppe von Fällen, in denen der Konflikt nicht sichtbar wird, weil lediglich eine einseitige Kollisionsnorm (vielfach für die Beantwortung einer Teilfrage nach der lex fori)

$[$S. 57$]$

in Frage steht. Ein Beispiel ist der Fall Gourié$[$1$]$, in dem die Anwendung des französischen Gesetzes vom 14. 7. 1819 davon abhing, ob ein unzweifelhaft fremdem Recht unterstehender Anspruch erbrechtlich oder ehegüterrechtlich zu qualifizieren war$[$2$]$. Man denkt hier zunächst nicht an einen Konflikt, weil ja nur eine lex causae für die Qualifikation in Betracht zu kommen scheint. Aber gerade der Fall Gourié zeigt, daß auch in solchen Fällen Qualifikationskonflikte (in casu zwischen französischem und amerikanischem Recht) auftreten können.

$[$S. 56 FN 1$]$ 1 Vgl. RIGAUX, S. 479.

$[$S. 57 FN 1$]$ Cour de Paris 6.1.1862, S. 1862. 2. 337; RIGAUX, S. 49.

$[$S. 57 FN 2$]$ RIGAUX, S. 48; weitere Fälle dort, S. 61. Rigaux spricht hier von Qualifikationsproblemen im Gegensatz zu Qualifikationskonflikten.}
\end{fragmentpart}
\begin{fragmentpart}{Anmerkung}
Der nur leicht veränderte Text wird einschließlich Fußnoten ohne jeden Hinweis von Steindorff übernommen. Steindorffs Erwähnung des amerikanischen Rechts wird präzisiert zum Recht von Pennsylvania. Steindorffs Fußnoten S. 56, 1 und S. 57, 1 werden ergänzt, dabei im letzteren Fall (wohl versehentlich?) RIGAUX entfernt (\textquotedbl{}ibid\textquotedbl{} und die passende Seitenzahl bleiben stehen).
\end{fragmentpart}
\end{fragment}
\phantomsection{}
\belowpdfbookmark{Fragment 94 1--7}{Lm-Fragment-094-01}
\hypertarget{Lm-Fragment-094-01}{}
\begin{fragment}
\begin{fragmentpart}{Dissertation S.~94 Z.~1--7 (KomplettPlagiat)}
\enquote{KAHN und BARTIN haben, wie schon gezeigt worden ist, die von SAVIGNY begründete Fragestellung des internationalen Privatrechts beibehalten. Sie strebten eine Qualifikationslösung an, die der Aufgabenstellung SAVIGNYS gerecht würde. Bemerkenswerterweise weist RlGAUX aber auch darauf hin, daß es sich bei der ursprünglichen Stellung des Qualifikationsproblems um einen Versuch zur Nationalisierung der Rechtsverhältnisse gehandelt habe, der den heutigen Bedürfnissen des Kollisionsrechts nicht mehr entspreche.$[$FN 17$]$

$[$FN 17$]$ Ibid., Rn. 312, S. 480-484 (481f.).}
\end{fragmentpart}
\begin{fragmentpart}{Original \cite[S.~58 Z.~6--7,~9--11,~101--104]{Steindorff-1958}}
\enquote{Kahn und Bartin haben, wie schon gezeigt worden ist, die von Savigny begründete Fragestellung des internationalen Privatrechts beibehalten. $[$...$]$ Vielmehr erstrebten sie eine Qualifikationslösung, die der Aufgabenstellung Savignys
gerecht würde$[$FN 1$]$.

$[$FN 1$]$ Rigaux, S. 481 f., weist richtig darauf hin, daß es sich bei der ursprünglichen Stellung des Qualifikationsproblems um einen Versuch zur Nationalisierung der Rechtsverhältnisse gehandelt hat, der den heutigen Bedürfnissen des internationalen Privatrechts nicht mehr entspreche.}
\end{fragmentpart}
\begin{fragmentpart}{Anmerkung}
Fließtext und Fußnote 1 werden nahezu wörtlich übernommen, der Verweis auf Rigeaux ergänzt. Kein Verweis auf Steindorff.
\end{fragmentpart}
\end{fragment}
\phantomsection{}
\belowpdfbookmark{Fragment 100 1--16}{Lm-Fragment-100-01}
\hypertarget{Lm-Fragment-100-01}{}
\begin{fragment}
\begin{fragmentpart}{Dissertation S.~100 Z.~1--16 (VerschärftesBauernopfer)}
\enquote{$[$ANCEL nimmt in dieser Diskussion sowie in der Frage verschiedener$]$ Qualifikationsstufen oder -schritte eine Sonderstellung ein: Er ist der Ansicht, die Trennung zwischen den zwei Qualifikationsschritten mit jeweils verschiedenen Qualifikationsgegenständen führe zu einer unwünschenswerten und unzulässigen Aufspaltung zwischen Anknüpfungsgegenstand und Qualifikationsgegenstand. Dies werde unnötige Qualifikationskonflikte verursachen.$[$FN 44$]$

Sein Vorschlag zur Lösung des Qualifikationsproblems besteht darin, in beiden Schritten ein einheitliches Qualifikationsobjekt anzunehmen: das sog. \textsl{„projet“}. Dies sei in der Beziehung (\textsl{lien}) zwischen Tatsachenbehauptung und Rechtsbegehren zu sehen.$[$FN 45$]$ Das \textsl{lien}, von dem ANCEL spricht, mag zwar in beiden Schritten oder auf beiden Ebenen das Qualifikationsobjekt bilden. Konkretisiert man indessen diesen von ANCEL benutzten Begriff im Detail, so ergibt sich, daß im ersten Schritt gerade die Rechtsfrage zwischen Tatsachen- und Rechtsbehauptung vermittelt und daß dieselbe Rolle auf der zweiten Stufe von dem zu qualifizierenden Rechtssatz wahrgenommen wird. ANCEL hat folglich
nichts anderes als einen Oberbegriff für die \textsl{termini} „Rechtsfrage“ und „Rechtssatz“ geschaffen.$[$FN 46$]$

\{$[$FN 45$]$ ANCEL, a.a.O. (Fn. 38), S. 216-260 ( 221, 224), 558; DERS., a.a.O. (Fn. 39), S. 234-240.\}

$[$FN 46$]$ Zur Kritik an ANCEL s. weiter HEYN, a.a.O. (Fn. 5), S. 29f.; KROPHOLLER, IPR3, S. 103; BATIFFOL/LAGARDE, DIP8, Rn. 291-1, S. 476f., R. 294, S. 480f., Rn. 296, S. 484f.}
\end{fragmentpart}
\begin{fragmentpart}{Original \cite[S.~29--30 Z.~21--30,~1--5]{Heyn-1986}}
\enquote{Insofern nimmt Ancel eine Sonderposition ein. Er meint$[$FN 54$]$, die Trennung zwischen den zwei Subsumtionsschritten mit jeweils verschiedenen Subsumtionsobjekten führe zu einer unzulässigen Aufspaltung zwischen Anknüpfungsgegenstand und Qualifikationsobjekt. Dies verursache Qualifikationskonflikte. Seine Lösung besteht deshalb darin, auf beiden Stufen ein einheitliches Qualifikationsobjekt anzunehmen, --- das sogenannte „projet“ . Dieses sei in der Beziehung („lien“) zwischen Tatsachenbehauptung und Rechtsbegehren zu sehen.

Will man sich mit der Analyse Ancels auseinandersetzen, gilt es folgendes zu
beachten: der „lien“ , von dem Ancel spricht, mag zwar in beiden Stufen das
Subsumtionsobjekt bilden. Konkretisiert man diesen von Ancel benutzten Be-

$[$S. 30$]$

griff aber, so zeigt sich, daß es im ersten Schritt gerade die Rechtsfrage ist, die zwischen Tatsachen- und Rechtsbehauptung vermittelt, und daß diese Rolle auf der zweiten Stufe von dem zu qualifizierenden Rechtssatz wahrgenommen wird. Ancel hat folglich nichts anderes als einen Oberbegriff für die Termini Rechtsfrage und Rechtssatz geschaffen.}
\end{fragmentpart}
\begin{fragmentpart}{Anmerkung}
Heyn wird in Fußnote 46 als erster von drei Verweisen \textquotedbl{}$[$z$]$ur Kritik an Ancel\textquotedbl{} genannt. Alternativ könnte das Fragment als \textquotedbl{}Verschleierung\textquotedbl{} gewertet werden.
\end{fragmentpart}
\end{fragment}
\phantomsection{}
\belowpdfbookmark{Fragment 114 3--12}{Lm-Fragment-114-03}
\hypertarget{Lm-Fragment-114-03}{}
\begin{fragment}
\begin{fragmentpart}{Dissertation S.~114 Z.~3--12 (Verschleierung)}
\enquote{Zwischen Europa und den Vereinigten Staaten hat sich in den letzten Jahren im IPR eine Kluft aufgetan, von der noch ungewiß ist, ob und wie sie in Zukunft wieder geschlossen werden kann. Der Gegensatz ist pointierter Ausdruck der traditionell unterschiedlichen Rechtsmethode: Während die IPR-Gesetze in Kontinentaleuropa und die systematische Erfassung von Kollisionsregeln im Vordergrund stehen, konzentriert man sich in den Vereinigten Staaten mehr auf das Ab wägen der sachlichen Argumente im konkreten Fall, die häufig sogleich dem materiellen Recht entnommen werden.$[$FN 10$]$ Das Bemühen gilt also primär einer sachgerechten Fortentwicklung des Fallrechts.$[$FN 11$]$ An einer Kodifikation des Kollisionsrechts durch die Bundesstaaten fehlt es.}
\end{fragmentpart}
\begin{fragmentpart}{Original \cite[S.~76 Z.~2--12]{Kropholler-1997}}
\enquote{Zwischen Europa und den Vereinigten Staaten, die kein für alle Bundesstaaten einheitliches Kollisionsrecht besitzen, hat sich in den letzten Jahrzehnten im IPR eine Kluft aufgetan, von der noch ungewiß ist, ob und wie sie in Zukunft wieder geschlossen werden kann. Der Gegensatz ist pointierter Ausdruck der traditionell unterschiedlichen Rechtsmethode: Während in Kontinentaleuropa die IPR-Gesetze und die systematische Erfassung von Kollisionsregeln im Vordergrund stehen, konzentriert man sich in den Vereinigten Staaten mehr auf das Abwägen der Argumente im konkreten Fall, die häufig sogleich dem materiellen Recht entnommen werden. Das Bemühen gilt also primär einer sachgerechten Fortentwicklung des Fallrechts. An einer Kodifikation des Kollisionsrechts durch die Gliedstaaten fehlt es im allgemeinen$[$FN 31$]$.}
\end{fragmentpart}
\begin{fragmentpart}{Anmerkung}
Kropholler wird hier nicht erwähnt, insbesondere auch nicht in den sehr ausführlichen FN 10 und 11.
\end{fragmentpart}
\end{fragment}
\phantomsection{}
\belowpdfbookmark{Fragment 117 11--13}{Lm-Fragment-117-11}
\hypertarget{Lm-Fragment-117-11}{}
\begin{fragment}
\begin{fragmentpart}{Dissertation S.~117 Z.~11--13 (KomplettPlagiat)}
\enquote{Nach der Methode der „governmental interest analysis“, die auf CURRIE$[$FN 24$]$ zurückgeht, soll der Richter im konkreten Fall untersuchen, welcher Staat ein Interesse an der Anwendung seiner Normen hat.
----
$[$FN 24$]$ CURRIE, Notes on Methods and Objectives in the Conflict of Laws, Duke L.J. 1959, S.171-181; DERS., The Verdict of the Quiescent Years: Mr. Hill and the Conflict of Laws, U. Chi.L.Rev. 28 (1961), S. 258-295, seine Antwort auf HILL, Governmental Interest and the Conflict of Laws- A Reply to Professor Currie, U.Chi.L.Rev 27 (1960), 463-504; BRILMAYER, Interest Analysis and the Myt $[$sic$]$ of Legislative Intent, Mich. L.Rev. 78 (1980), S. 392-431.}
\end{fragmentpart}
\begin{fragmentpart}{Original \cite[S.~77 Z.~1--3]{Kropholler-1997}}
\enquote{Nach der Methode der „\textsl{governmental interest analysis}“, die auf \textsl{Currie} zurückgeht, soll der Richter im konkreten Fall untersuchen, welcher Staat ein Interesse an der Anwendung seiner Normen hat$[$FN 34$]$.
----
$[$FN 34$]$ \textsl{Currie} hat als Feind genereller Kollisionsregeln geradezu gesagt: „We would be better off without conflict rules“; siehe \textsl{Currie}, Selected Essays on the Conflict of Laws (1963) und seine kurze Zusammenfassung in Colum. L. Rev. 63 (1963) 142 f.}
\end{fragmentpart}
\begin{fragmentpart}{Anmerkung}
Kropholler wird hier nicht erwähnt. Fußnote 25 verwendet dasselbe Zitat von Currie wie Fußnote 34 bei Kropholler, führt diesen Gedanken aber weiter.
\end{fragmentpart}
\end{fragment}
\phantomsection{}
\belowpdfbookmark{Fragment 118 5--33}{Lm-Fragment-118-05}
\hypertarget{Lm-Fragment-118-05}{}
\begin{fragment}
\begin{fragmentpart}{Dissertation S.~118 Z.~5--33 (VerschärftesBauernopfer)}
\enquote{In einem seiner letzten Aufsätze faßte CURRIE seine Position in folgenden --- frei übersetzten --- Thesen zusammen:$[$FN 27$]$
:(1) Steht die Anwendung fremden, von der lex fori inhaltlich abweichenden, Rechts zur Debatte, so sind zunächst die Zwecksetzungen, die sich in den fraglichen Rechtssätzen ausdrücken, und dann diejenigen Umstände zu untersuchen, unter denen die beiden Staaten vernünftigerweise daran interessiert sind, diese Zwecke zu realisieren. Diese Untersuchung bedient sich der üblichen Auslegungs- und Deutungstechniken.
:(2) Ergibt sich, daß im konkreten Fall nur ein Staat an der Verwirklichung seiner Ziele interessiert ist, so ist das Recht dieses Staates anzuwenden.
:(3) Offenbaren sich entgegengesetzte Interessen beider Staaten, so ist zu prüfen, ob bei einer erneuten, bescheideneren und zurückhaltenderen Interpretation der Zwecke und Interessen des einen oder des anderen Staates ein Konflikt vermieden werden kann.
:(4) Erweist sich dabei, daß ein Konflikt legitimer Interessen beider Staaten unvermeidbar ist, so ist die lex fori anzuwenden.
:(5) Ist das Forum selbst desinteressiert, besteht aber ein unvermeidbarer Konflikt zwischen den Interessen zweier anderer Staaten und kommt eine \textsl{a limine-}Abweisung nicht in Frage, so sollte das Gericht die lex fori anwenden.
:(6) Interessenkonflikte zwischen den Staaten führen zu unterschiedlichen, durch den Prozeßort bestimmten Entscheidungen des gleichen Problems. Erscheint dies für besondere Fragen als ernste Beeinträchtigung eines wichtigen Bundesinteresses an einheitlichen Entscheidungen, so sollte das Gericht nicht versuchen, auf Kosten des legitimen Interesses seines eigenen Staates eine Lösung  zu improvisieren, sondern es dem Kongreß überlassen zu bestimmen, welches Interesse zurückstehen soll.

Als erste Aufgabe bei der Entscheidung eines kollisionsrechtlichen Falles beschreibt CURRIE die Bestimmung der governmental policy des Forums. Das
traditionelle internationale Privatrecht habe bisher die Interessen der Staaten $[$(die \textsl{public policy}) ignoriert.$]$

$[$FN 27$]$ CURRIE, Comments on \textsl{Babcock v. Jackson}. A Recent Development in the Conflict of Laws, Columbia L.Rev. 63 (1963), 1233-1243 (1242f.); so ähnlich CURRIE, in Duke L.J., S.
178; Cf. ferner JOERGES, a.a.O. (Fn. 7), S. 39f., 43-50.}
\end{fragmentpart}
\begin{fragmentpart}{Original \cite[S.~39,~40,~43,~49 Z.~18--27,~1--17,~11--12,~23--25]{Joerges-1971}}
\enquote{In einem seiner letzten Aufsätze faßte CURRIE seine Position in folgenden
- frei übersetzten --- Thesen zusammen:
:„(1) Steht die Anwendung fremden, von der lex fori inhaltlich abweichenden$[$FN 10$]$ Rechts zur Debatte, so sind zunächst die Zwecksetzungen, die sich in den fraglichen Rechtssätzen ausdrücken, und dann diejenigen Umstände zu untersuchen, unter denen die beiden Staaten vernünftigerweise daran interessiert sind, diese Zwecke zu realisieren. Diese Untersuchung bedient sich der üblichen Auslegungs- und Deutungstechniken.
:(2) Ergibt sich, daß im konkreten Fall nur ein Staat an der Verwirklichung seiner Ziele interessiert ist, so ist das Recht dieses Staates anzuwenden.
:(3) Offenbaren sich entgegengesetzte Interessen beider Staaten, so ist zu prüfen, ob bei einer erneuten, bescheideneren und zurückhaltenderen Interpretation der Zwecke und Interessen des einen oder des anderen Staates ein Konflikt vermieden werden kann.
:(4) Erweist sich dabei, daß ein Konflikt legitimer Interessen beider Staaten unvermeidbar ist, so ist die lex fori anzuwenden.
:(5) Ist das Forum selbst desinteressiert, besteht aber ein unvermeidbarer Konflikt zwischen den Interessen zweier anderer Staaten und kommt eine a limine-Abweisung nicht in Frage, so sollte --- bis jemand auf eine bessere Idee kommt --- das Gericht die lex fori anwenden.
:(6) Interessenkonflikte zwischen den Staaten führen zu unterschiedlichen, durch den Prozeßort bestimmten Entscheidungen des gleichen Problems. Erscheint dies für besondere Fragen als ernste Beeinträchtigung eines wichtigen Bundesinteresses an einheitlichen Entscheidungen, so sollte das Gericht nicht versuchen, auf Kosten des legitimen Interesses seines eigenen Staates eine Lösung zu improvisieren, sondern es dem Kongreß überlassen zu bestimmen, welches Interesse zurückstehen soll“$[$FN 11$]$.

$[$S. 43$]$

Als erste Aufgabe bei der Entscheidung eines kollisionsrechtlichen Falles
beschreibt CURRIE die Bestimmung der governmental policy des Forums.

$[$S. 49$]$

Das traditionelle IPR, so lautet einer der Haupteinwände CURRIEs, habe die Interessen der Staaten (die \textsl{public policy}) ignoriert$[$FN 62$]$.



$[$FN 11$]$ CURRIE, Comment on Babcock v. Jackson: Colum. L. Rev. 63 (1963) 1233, 1242 ff. (Ältere Zusammenfassungen sind stärker als Gegensatz zum Ersten Restatement formuliert und weniger stark differenziert, vgl. CURRIE, Selected Essays, ch. V $[$1958$]$ 188 f. = ch. IV $[$1959$]$ 183 f.; Übersetzung bei HEINI, Neuere Strömungen 53.)}
\end{fragmentpart}
\begin{fragmentpart}{Anmerkung}
Kurze Textübernahmen aus drei unterschiedlichen Seiten rahmen ein langes Plagiat einer Übersetzung ein (nicht zu verwechseln mit Übersetzungsplagiat). Der Verfasser macht sich die Übersetzung eines komplexen englischen Rechtstextes in die deutsche Sprache zu eigen und gibt seine Quelle als \textquotedbl{}Cf. ferner\textquotedbl{} an. Fortsetzung auf in \hyperlink{Lm-Fragment-119-01}{Fragment\_119\_01}.
\end{fragmentpart}
\end{fragment}
\phantomsection{}
\belowpdfbookmark{Fragment 119 1--2,~101--104}{Lm-Fragment-119-01}
\hypertarget{Lm-Fragment-119-01}{}
\begin{fragment}
\begin{fragmentpart}{Dissertation S.~119 Z.~1--2,~101--104 (Verschleierung)}
\enquote{Der zweite Schritt in der Analyse eines kollisionsrechtlichen Falles ist die Bestimmung eines möglichen governmental interest.$[$FN 28$]$

$[$FN 28$]$ Zum Interessenbegriff gibt er eine allgemeine Definition: „An interest, as I use the term, is the product of (a) a governmental policy and (b) the concurrent existence of an appropriate relationship between the state having the policy and the transaction, the parties or the litigation“. CURRIE, Selected Essays on the Conflict of Laws, Durham 1963, S. 621, 737.}
\end{fragmentpart}
\begin{fragmentpart}{Original \cite[S.~45 Z.~21--22,~25--30]{Joerges-1971}}
\enquote{Der zweite Schritt in der Analyse eines kollisionsrechtlichen Falles ist die Bestimmung eines möglichen „governmental interest“. 

$[$...$]$ Zum Interessenbegriff gibt er eine allgemeine Definition: „An ,interest' as I use the term, is the product of
(a) a governmental policy and
(b) the concurrent existence of an appropriate relationship between the state having the policy and the transaction, the parties, or the litigation“$[$FN 35$]$.

$[$35$]$ AaO 621, vgl. 737.}
\end{fragmentpart}
\begin{fragmentpart}{Anmerkung}
Forsetzung von \hyperlink{Lm-Fragment-118-05}{Lm/Fragment\_118\_05}. Dort wird Joerges eingangs als \textquotedbl{}Cf. ferner\textquotedbl{} genannt. Könnte alternativ als Bauernopfer gewertet werden.
\end{fragmentpart}
\end{fragment}
\phantomsection{}
\belowpdfbookmark{Fragment 126 1--5,~101--106}{Lm-Fragment-126-01}
\hypertarget{Lm-Fragment-126-01}{}
\begin{fragment}
\begin{fragmentpart}{Dissertation S.~126 Z.~1--5,~101--106 (BauernOpfer)}
\enquote{\textsl{d. Bevorzugungsprinzipien (Principles of Preference)}

CAVERS, berühmt geworden als „the man who --- like the former master Aldricus --- wants the court to do justice in the individual case“,$[$FN 61$]$ hat seine Position in einer 1965 erschienenen Studie neu definiert.$[$FN 62$]$ Laut EHRENZWEIG einer der wichtigsten Beiträge unserer Zeit zu dieser Frage.$[$FN 63$]$ CAVERS:

$[$FN 61$]$ So KEGEL, RCADI 1964 II, S. 110; s. auch CAVERS, The Conditional Seller's Remedies and the Choice-of-Law Process, Some Notes on Shanahan, N.Y.U.L.Rev. 35 (1960), S. 1122, 1139.

$[$FN 62$]$ CAVERS, The Choice-of-Law Process, Ann Arbor 1965.

$[$FN 63$]$ EHRENZWEIG, A Counter-Revolution in Conflicts Law? From Beale to Cavers, Harv. L. Rev. 80 (1966), S. 377-401 (378).

$[$FN 64$]$ CAVERS, a.a.O. (Fn. 62), S. 63; Übersetzung bei JOERGES, a.a.O. (Fn. 7), S. 42.

$[$65$]$ CAVERS, \textsl{ibid.}, JOERGES, ibid.

$[$66$]$ CAVERS, a.a.O. (Fn. 62), S. 63.}
\end{fragmentpart}
\begin{fragmentpart}{Original \cite[S.~42 Z.~5--10,~101--108]{Joerges-1971}}
\enquote{\textsl{3. Cavers: Bevorzugungsprinzipien („Principles of Preference“)}

CAVERS, berühmt geworden als „the man who --- ‚like the former master ALDRICUS‘$[$FN 17$]$ --- wants the court to do justice in the individual case“$[$FN 18$]$, hat seine Position in einem im Jahre 1965 erschienenen Buch über die Rechtsanwendungsproblematik$[$FN 19$]$ --- nach der Voraussage EHRENZWEIGS „einer der wichtigsten Beiträge unserer Zeit zu dieser Frage“$[$FN 20$]$ --- neu definiert:

$[$FN 17$]$ KEGEL, Crisis 110.

$[$FN 18$]$ So glossiert CAVERS, Process 73 seinen „unsuccessful attempt at the communication of thought“ (CAVERS, The Conditional Seller’s Remedies and the Choice-of-Law Process --- Some Notes on Shanahan: N. Y. U. L. Rev. 35 $[$1960$]$ 1126, 1139).

$[$FN 19$]$ CAVERS, The Choice-of-Law Process (1965).

$[$FN 20$]$ EHRENZWEIG, A Conter-Revolution in Conflicts Law? From Beale to Cavers:
Harv. L. Rev. 80 (1966) 377, 378.

$[$FN 21$]$ CAVERS, Process 63.}
\end{fragmentpart}
\begin{fragmentpart}{Anmerkung}
In beiden Arbeiten folgt ein identisches Zitat aus Cavers, das aber nicht als Plagiat gewertet wird, weil der Verfasser in Fußnote 64 Joerges korrekt als Übersetzer angibt, zudem das Zitat aufteilt und zwischendrin selbst zusammenfasst. Die Fußnoten von Joerges zu diesem Zitat werden allerdings auch komplett übernommen, ein Beleg wird eine Fußnote höher geschoben. 

Joerges wird zudem in Fußnote 65 erwähnt, aber ohne erkenntlich zu machen, dass der Gedankengang und weitgehend auch seine Formulierung von Joerges stammen.
\end{fragmentpart}
\end{fragment}
\phantomsection{}
\belowpdfbookmark{Fragment 131 4--7,~8--14}{Lm-Fragment-131-04}
\hypertarget{Lm-Fragment-131-04}{}
\begin{fragment}
\begin{fragmentpart}{Dissertation S.~131 Z.~4--7,~8--14 (Verschleierung)}
\enquote{Das Restatement Second, das unter Führung von REESE in der Zeit zwischen 1952 und 1971 entstand, mußte versuchen, aus der Meinungsvielfalt in der Theorie das \textsl{maximum} von praktikablen Regeln oder wenigstens Richtlinien herauszuarbeiten.$[$FN 98$]$ $[$...$]$ Die in sec. 6 des Restatement Second vorangestellten \textsl{choice of law principles} sind das Ergebnis eines solchen Kompromisses, der Raum dafür läßt, Kollisionsfälle nicht nur nach Regeln, sondern aufgrund unterschiedlicher Methoden (\textsl{approaches}) zu lösen.$[$FN 99$]$ In der sachlichen Beurteilung ist man sich heute weitgehend einig: War das erste Restatement zu starr, so mangelt es dem zweiten an klaren Konturen.

$[$FN 98$]$ AMERICAN LAW INSTITUTE, Restatement of the Law Second, Conflict of Laws 2d (1971), St. Paul / Minn. 1971; vgl. HANOTIAU, a.a.O. (Fn. 7), S. 155-163.

$[$FN 999 REESE, General Course on Private International Law, RCADI 150 (1976 II), S. 1-193, (37-84); DERS., Conflict of Laws and the Restatement Second, Law \& Cont. Probl. 28 (1963), S. 679-699.}
\end{fragmentpart}
\begin{fragmentpart}{Original \cite[S.~78 Z.~12--19]{Kropholler-1997}}
\enquote{Das \textsl{Restatement Second} (1971), das unter Führung von \textsl{Reese} entstand, mußte aus der Meinungsvielfalt in der Theorie das Maximum an praktikablen Regeln oder wenigstens Richtlinien herauszuholen versuchen. Die in sec. 6 vorangestellten „choice of law principles“ sind das Ergebnis eines Kompromisses, der Raum dafür läßt, Kollisionsfälle überhaupt nicht nach Regeln, sondern aufgrund unterschiedlicher Methoden („approaches“) zu lösen$[$FN 39$]$. War das Erste Restatement zu starr, so mangelt es dem Zweiten --- besonders im Schuldrecht --- an klaren Konturen.}
\end{fragmentpart}
\begin{fragmentpart}{Anmerkung}
Der Text von Kropholler wird nur leicht geändert, zwei Sätze (einer davon hier nicht wiedergegeben) dazwischengeschoben. Auf Kropholler wird nicht verwiesen.
\end{fragmentpart}
\end{fragment}
\phantomsection{}
\belowpdfbookmark{Fragment 133 1--12}{Lm-Fragment-133-01}
\hypertarget{Lm-Fragment-133-01}{}
\begin{fragment}
\begin{fragmentpart}{Dissertation S.~133 Z.~1--12 (Verschleierung)}
\enquote{In der Rechtsprechung werden verschiedene Ansichten vertreten. Trotz der \textsl{Revolution} gegen das erste Restatement wird dieses überraschenderweise noch immer in vielen US-Staaten herangezogen. Andere Gerichte berufen sich auf das zweite Restatement, unter dessen Mantel freilich verschiedene methodische Ansätze möglich sind und auch angewandt werden. Zur \textsl{governmental interest analysis}, dem \textsl{better law approach} und anderen modernen Theorien bekennen sich die Gerichte bislang nur in einzelnen Staaten. Insgesamt werden von der Praxis etwa zehn verschiedene kollisionsrechtliche Ansätze bemüht. Es bleibt abzuwarten, ob die amerikanischen Gerichte im Schuldrecht oder anderen Rechtsgebieten, die vom zweiten Restatement kollisionsrechtlich nicht behandelt werden, wieder zu einer einheitlichen Haltung und zu einer Befolgung bestimmter Regeln zurückfinden.}
\end{fragmentpart}
\begin{fragmentpart}{Original \cite[S.~78--79 Z.~20--27,~1--4]{Kropholler-1997}}
\enquote{$[$Seite 78$]$ 

Die \textsl{gerichtliche Praxis} ist in die verschiedensten Lager gespalten$[$FN 40$]$. Trotz der „Revolution“ gegen das Erste Restatement wird dieses überraschenderweise noch immer in mehreren Gliedstaaten zugrunde gelegt; indes ist die Tendenz rückläufig. Die meisten Gerichte berufen sich auf das Zweite Restatement, unter dessen Mantel freilich verschiedene methodische Ansätze möglich sind und auch benutzt werden. Zur „governmental interest analysis“, dem „better law approach“ und anderen modernen Theorien bekennen sich die Gerichte bislang nur in vereinzelten Gliedstaaten. Insgesamt herrscht in der Praxis 

$[$Seite 79$]$ 

ergebnisorientierter Eklektizismus$[$FN 41$]$, der die Vorhersehbarkeit der Entscheidung vielfach stark erschwert. Es bleibt abzuwarten, ob die amerikanischen Gerichte im Schuldrecht wieder zu einer einheitlichen Haltung und zu einer Befolgung bestimmter Regeln zurückfinden.}
\end{fragmentpart}
\begin{fragmentpart}{Anmerkung}
Kropholler wird in diesem Zusammenhang nicht erwähnt. Zuletzt wird er auf S. 132, Fußnote 105, für ein anderes Thema zitiert.
\end{fragmentpart}
\end{fragment}
\phantomsection{}
\belowpdfbookmark{Fragment 138 12--18}{Lm-Fragment-138-12}
\hypertarget{Lm-Fragment-138-12}{}
\begin{fragment}
\begin{fragmentpart}{Dissertation S.~138 Z.~12--18 (Verschleierung)}
\enquote{In Europa fand die amerikanische „Revolution“ im IPR bislang keine Entsprechung. Vom Schrifttum wurde die Entwicklung jenseits des Atlantiks zwar sorgfältig registriert,$[$FN 132$]$ die Kritik überwog jedoch.$[$FN 133$]$ Nur vereinzelt und mit Einschränkungen wurde eine \textsl{governmental interest analysis} empfohlen$[$FN 134$]$ oder ein \textsl{better law approach} befürwortet.$[$FN 135$]$ Diese Vorschläge stießen auf den Einwand, sie seien mit der Struktur des deutschen Kollisionsrechts nicht zu vereinbaren.$[$FN 136$]$}
\end{fragmentpart}
\begin{fragmentpart}{Original \cite[S.~79 Z.~5--11]{Kropholler-1997}}
\enquote{6. In \textsl{Europa} fand die amerikanische „Revolution“ im IPR bislang keine Entsprechung$[$FN 42$]$.

a) Von der \textsl{Wissenschaft} wurde die Entwicklung jenseits des Atlantik zwar sorgfältig registriert$[$FN 43$]$, aber die Kritik überwog$[$FN 44$]$. Nur vereinzelt und mit Einschränkungen wurde eine „governmental interest analysis“ empfohlen$[$FN 45$]$ oder ein „better law approach“ befürwortet$[$FN 46$]$. Indes stießen diese Vorschläge auf den Einwand, sie seien mit der Struktur unseres IPR nicht zu vereinbaren$[$FN 47$]$.}
\end{fragmentpart}
\begin{fragmentpart}{Anmerkung}
Die Fußnoten 132-136 überschneiden sich teilweise mit Fußnoten 42-47 von Kropholler, aber nicht in bedenklicher Weise. Kropholler wird in diesem Zusammenhang nicht erwähnt.

Forsetzung auf der nächsten Seite: \hyperlink{Lm-Fragment-139-01}{Lm/Fragment\_139\_01}
\end{fragmentpart}
\end{fragment}
\phantomsection{}
\belowpdfbookmark{Fragment 139 1--3}{Lm-Fragment-139-01}
\hypertarget{Lm-Fragment-139-01}{}
\begin{fragment}
\begin{fragmentpart}{Dissertation S.~139 Z.~1--3 (Verschleierung)}
\enquote{Bei der gesetzgeberischen Reform des deutschen EGBGB (1986) wurden alle sich auf amerikanische Lehren gründende Reformvorschläge ausdrücklich abgelehnt.$[$FN 137$]$}
\end{fragmentpart}
\begin{fragmentpart}{Original \cite[S.~79 Z.~12--14]{Kropholler-1997}}
\enquote{Bei der \textsl{gesetzgeberischen Reform} des EGBGB durch das IPRNG von 1986 wurden alle auf amerikanischen Lehren fußenden Reformvorschläge ausdrücklich abgelehnt $[$...$]$}
\end{fragmentpart}
\begin{fragmentpart}{Anmerkung}
Fortsetzung von \hyperlink{Lm-Fragment-138-12}{Fragment\_138\_12}.
\end{fragmentpart}
\end{fragment}
\phantomsection{}
\belowpdfbookmark{Fragment 139 4--12}{Lm-Fragment-139-04}
\hypertarget{Lm-Fragment-139-04}{}
\begin{fragment}
\begin{fragmentpart}{Dissertation S.~139 Z.~4--12 (Verschleierung)}
\enquote{In den USA sind die Gerichte einzel- und bundesstaatlicher Jurisdiktionen nicht durch eine umfassende Kodifikation des EPR gebunden. Sie haben meist nur die (leichtere) Aufgabe, \textsl{interstate conflicts} zwischen materiell divergierenden \textsl{statutes} amerikanischer Gliedstaaten zu lösen. Dagegen hat es der europäische Richter, gerade was das private Wirtschaftsrecht angeht, mit echten Auslandsfällen zu tun. Durch seine Bindung an geschriebenes Kollisionsrecht hat er meist gar nicht die Freiheit, diese Fälle mittels einer Methode kollisionsrechtlicher Rechtsfindung unter einschließender Abwägung Sachrechts zu entscheiden.}
\end{fragmentpart}
\begin{fragmentpart}{Original \cite[S.~80 Z.~7--12,~13--15]{Kropholler-1997}}
\enquote{In den Vereinigten Staaten sind die Gerichte im allgemeinen nicht durch eine Kodifikation des IPR gebunden, und sie haben meist nur die (leichtere) Aufgabe, „interstate conflicts“ zwischen divergierenden „statutes“ der amerikanischen Gliedstaaten zu lösen. Dagegen hat der deutsche Richter es mit echten Auslandsfällen zu tun, und er hat durch seine Bindung an das geschriebene $[$...$]$ IPR meist gar nicht die Freiheit, diese Fälle nur mittels einer bloßen Methode kollisionsrechtlicher Rechtsfindung unter Abwägung der beteiligten materiellen Rechte zu entscheiden.}
\end{fragmentpart}
\begin{fragmentpart}{Anmerkung}
Der dem dargestellten Ausschnitt unmittelbar folgende Satz nimmt genau wie die Vorlage im Anschluss an die hier wiedergegebene Passage Bezug auf die Begriffsgruppe »Rechtssicherheit, Entscheidungseinklang und Regelbildung«, kommt aber zu einer anderen Einschätzung. Dies würde also eine legitime gedankliche Auseinandersetzung mit der Quelle darstellen; auf die Quelle Kropholler (1997) wird allerdings im Zusammenhang überhaupt nicht verwiesen.
\end{fragmentpart}
\end{fragment}
\phantomsection{}
\belowpdfbookmark{Fragment 143 10--13}{Lm-Fragment-143-10}
\hypertarget{Lm-Fragment-143-10}{}
\begin{fragment}
\begin{fragmentpart}{Dissertation S.~143 Z.~10--13 (BauernOpfer)}
\enquote{Während gewisse Hauptfragen des Besonderen Teils wie z.B. Geschäftsfähigkeit, Form der Rechtsgeschäfte fast in allen vorhandenen Gesetzen geregelt sind, gibt es einen solchen festen Kanon des Allgemeinen Teils nicht. Einzig der Vorbehalt eines ordre public erscheint fast überall.$[$FN 1$]$

$[$$[$FN 1$]$ NEUHAUS, Zur Einführung, in MAKAROV (Hrsg.), Quellen des Internationalen Privatrechts --- Nationale Kodifikation, 3. Aufl. bearbeitet von \textsl{Kropholler, Neuhaus und Waehler}, Tübingen 1978, S. 6f. $[$...$]$$]$}
\end{fragmentpart}
\begin{fragmentpart}{Original \cite[S.~6--7 Z.~1--4]{Makarov-1978}}
\enquote{Während
$[$S. 7$]$ gewisse Hauptfragen des Besonderen Teils --- Geschäftsfähigkeit, Eingehung und Auflösung der Ehe usw. --- fast in allen Gesetzen geregelt sind, gibt es einen solchen festen Kanon des Allgemeinen Teiles nicht. Einzig der Vorbehalt des ordre public erscheint fast überall.}
\end{fragmentpart}
\begin{fragmentpart}{Anmerkung}
Die Quelle wird in Fußnote 1 genannt, die weitgehend wörtliche Übernahme aber nicht gekennzeichnet.
\end{fragmentpart}
\end{fragment}
\phantomsection{}
\belowpdfbookmark{Fragment 144 11--20}{Lm-Fragment-144-11}
\hypertarget{Lm-Fragment-144-11}{}
\begin{fragment}
\begin{fragmentpart}{Dissertation S.~144 Z.~11--20 (BauernOpfer)}
\enquote{Das Wort „Qualifikation“ findet sich in dem in der Wissenschaft meistgebrauchten Sinn „der Subsumtion unter eine Kollisionsnorm“ in wenigen Gesetzen (cf. Tabelle A). Zumeist beschränken sich Kodifikationen wie z.B. die portugiesische (Art. 15) darauf, den Ausdruck eines bestimmten Problems der sekundären Qualifikation zu bestimmen, nämlich die Auswahl anwendbarer Normen innerhalb des maßgebenden Rechts. Andere Rechtsordnungen, wie die brasilianische (Artt. 8, 9 Einführungsgesetz zum ZGB) lassen die Bedeutung des Wortes im Ungewissen. Künftige Gesetze werden die umstrittene Vokabel besser vermeiden, zumal die einschlägigen Fragen kaum mit einer knappen gesetzlichen Regel zu beantworten sind.$[$FN 7$]$

$[$$[$FN 7$]$ NEUHAUS, in MAKAROV, a.a.O. (Fn. 1), S. 8.$]$}
\end{fragmentpart}
\begin{fragmentpart}{Original \cite[S.~8 Z.~18--26]{Makarov-1978}}
\enquote{c) Das Wort „Qualifikation“ findet sich in dem wissenschaftlich meistgebrauchten Sinn der Subsumtion unter eine Kollisionsnorm (vgl. dazu II) in keinem einzigen Gesetz. Vielmehr beschränkt Portugal (Art. 15) den Ausdruck auf ein bestimmtes Problem der sekundären Qualifikation, nämlich die Auswahl der anwendbaren Normen innerhalb des maßgebenden Rechts (vgl. XII 4 b), und im brasilianischen Recht (Artt. 8, 9) bleibt die Bedeutung des Wortes ungewiß. Künftige Gesetze werden die umstrittene Vokabel wohl besser vermeiden, zumal die einschlägigen Fragen kaum mit einer knappen gesetzlichen Regel zu beantworten sind.}
\end{fragmentpart}
\begin{fragmentpart}{Anmerkung}
Fußnote 7 verweist auf die Quelle. Der genaue Umfang und dass teilweise, zB. der komplette letzte Satz, wörtlich übernommen wurde, ist nicht ausgewiesen.
\end{fragmentpart}
\end{fragment}
\phantomsection{}
\belowpdfbookmark{Fragment 145 23--24}{Lm-Fragment-145-23}
\hypertarget{Lm-Fragment-145-23}{}
\begin{fragment}
\begin{fragmentpart}{Dissertation S.~145 Z.~23--24 (BauernOpfer)}
\enquote{Für das kollisionsrechtliche Einheitsrecht, das in aller Regel in Konventionen enthalten ist, haben sich einheitliche Grundbegriffe, wie sie für das einheitliche $[$Sachrecht bestehen, erst im beschränkten Umfang herausgebildet.$]$}
\end{fragmentpart}
\begin{fragmentpart}{Original \cite[S.~328 Z.~3--5]{Kropholler-1975}}
\enquote{Für das kollisionsrechtliche Einheitsrecht, das in aller Regel in Konventionen enthalten ist, haben sich einheitliche Grundbegriffe --- wie für das einheitliche Sachrecht --- erst in beschränktem Umfang herausgebildet.}
\end{fragmentpart}
\begin{fragmentpart}{Anmerkung}
Fortsetzung auf der nächsten Seite (siehe \hyperlink{Lm-Fragment-146-01}{Lm/Fragment\_146\_01}), auf der die passende Stelle bei Kropholler in Fußnote 15 erwähnt wird -- zwei Sätze weiter unten am Ende des Absatzes.
\end{fragmentpart}
\end{fragment}
\phantomsection{}
\belowpdfbookmark{Fragment 146 1--9,~12--23}{Lm-Fragment-146-01}
\hypertarget{Lm-Fragment-146-01}{}
\begin{fragment}
\begin{fragmentpart}{Dissertation S.~146 Z.~1--9,~12--23 (BauernOpfer)}
\enquote{$[$Für das kollisionsrechtliche Einheitsrecht, das in aller Regel in Konventionen
enthalten ist, haben sich einheitliche Grundbegriffe, wie sie für das einheitliche$]$ Sachrecht bestehen, erst im beschränkten Umfang herausgebildet. Die Erarbeitung der Grundbegriffe ist hier --- nicht anders als im autonomen IPR --- primär Aufgabe der Wissenschaft und nicht der Legislative. Die Qualifikation wird im einheitlichen Kollisionsrecht erleichtert, wenn die Grundsätze beachtet werden, die allgemein für die Begriffswahl$[$FN 13$]$ und für die Auslegung des Einheitsrechts$[$FN 14$]$ entwickelt worden sind.$[$FN 15$]$

Durch weitblickende, im Ergebnis vergleichende \textsl{Rechtsetzung} lassen sich Qualifikationsschwierigkeiten in der späteren Rechtsanwendung vielfach vermeiden. $[$...$]$

Zur \textsl{Rechtsanwendung} ist folgendes zu bemerken: Der Gegenstand der Qualifikation läßt sich nicht einheitlich bestimmen.$[$FN 17$]$ Die Qualifikationsmethode müßte im staatsvertraglichen IPR als Sonderproblem der Auslegung vereinheitlichten Rechts denselben Maximen unterworfen werden, die allgemein für die Interpretation des Einheitsrechts gelten. Da demnach eine \textsl{internationaliserungs}fähige und \textsl{einheitliche} Interpretation$[$FN 18$]$ erstrebenswert ist, verbietet sich im allgemeinen die im autonomen IPR vielfach vertretene Qualifikation nach der lex fori oder nach der lex causae. Im Endeffekt handelt es sich hier um eine eigene (RABELsche) IPR- und Qualifikationsmethode. Dem Sinn und Zweck eines Staatsvertrages über vereinheitlichtes Kollisionsrecht wird vielmehr nur eine von den nationalen Rechtsordnungen losgelöste, sog. \textsl{autonome} Qualifikation gerecht.$[$FN 19$]$
----
$[$FN 13$]$ Kropholler, a.a.O. (Fn. 6), § 18 III, S. 246-249.

$[$FN 14$]$ Ibid., § 19, S. 258-292.

$[$FN 15$]$ Ibid., S. 328.

$[$...$]$

$[$FN 17$]$ Ibid., S. 329f.

$[$FN 18$]$ Ibid., § 17 III, S. 240-243. Cf. \textsl{Antwerp United Diamonds v. Air Europe (a firm)}, $[$1995$]$ 3 All ER 424-432 (C. A.); Fothergill v. Monarch Airlines Ltd, $[$1980$]$ 2 All ER 696-721 (H.L.).

$[$FN 19$]$ Kropholler, a.a.O. (Fn. 6), S. 330f.}
\end{fragmentpart}
\begin{fragmentpart}{Original \cite[S.~328--331 Z.~3--7,~20--25,~30--33,~18--25,~31--32,~1--2]{Kropholler-1975}}
\enquote{$[$S. 328$]$

Für das kollisionsrechtliche Einheitsrecht, das in aller Regel in Konventionen
enthalten ist, haben sich einheitliche Grundbegriffe — wie für das
einheitliche Sachrecht --- erst in beschränktem Umfang herausgebildet. Die Erarbeitung der Grundbegriffe ist hier nicht anders als im autonomen IPR primär eine Aufgabe der Wissenschaft und nicht der Legislative. $[$...$]$

Die \textsl{Qualifikation} wird im einheitlichen Kollisionsrecht erleichtert, wenn die Grundsätze beachtet werden, die allgemein für die Begriffswahl (oben § 18 III, S. 246 ff.) und für die Auslegung des Einheitsrechts (oben § 19) entwickelt worden sind$[$FN 2$]$.

1. Durch weitblickende \textsl{Rechtsetzung} lassen sich Qualifikationsschwierigkeiten
in der späteren Rechtsanwendung vielfach vermeiden. $[$...$]$

$[$S. 329$]$

2. In der \textsl{Rechtsanwendung} sind hinsichtlich des Gegenstandes und der Methode der Qualifikation Besonderheiten zu vermerken.

a) Der \textsl{Gegenstand} der Qualifikation läßt sich nicht für alle Kollisionsnormen
einheitlich bestimmen. $[$...$]$

$[$S. 330$]$

Die Qualifikationsmethode, also die Art der Auslegung eines Verweisungsbegriffs im Rahmen des Subsumtionsvorganges, muß im staatsvertraglichen IPR --- als Sonderproblem der Auslegung vereinheitlichten Rechts --- denselben Maximen unterworfen werden, die allgemein für die Interpretation des Einheitsrechts gelten (vgl. oben § 17 III, S. 240 ff.). Da demnach eine internationalisierungsfähige und einheitliche Interpretation zu erstreben ist, verbietet sich im allgemeinen die im autonomen IPR vielfach vertretene Qualifikation nach der lex fori oder nach der lex causae. $[$...$]$

Dem Sinn und Zweck eines Staatsvertrages über einheitliches Kollisionsrecht wird vielmehr nur

$[$S. 331$]$

eine von den nationalen Rechtsordnungen losgelöste, sogenannte \textsl{autonome} Qualifikation gerecht.
----
$[$FN 2$]$ Zum Begriff der Qualifikation vgl. NEUHAUS 64 ff. Alle Grundbegriffe des autonomen IPR, die im folgenden als bekannt vorausgesetzt werden, sind in diesem Werk meisterhaft erläutert.}
\end{fragmentpart}
\begin{fragmentpart}{Anmerkung}
Fortsetzung von \hyperlink{Lm-Fragment-145-23}{Fragment\_145\_23}. Der Text von Kropholler wird durch Auslassungen stark reduziert, zwei Sätze werden dazwischengeschoben (davon ist einer nicht hier wiedergegeben).
\end{fragmentpart}
\end{fragment}
\phantomsection{}
\belowpdfbookmark{Fragment 160 12--17}{Lm-Fragment-160-12}
\hypertarget{Lm-Fragment-160-12}{}
\begin{fragment}
\begin{fragmentpart}{Dissertation S.~160 Z.~12--17 (BauernOpfer)}
\enquote{Die Qualifikationsfrage ist in Art. 10 des IPR-Entwurfs geregelt. Im chinesischen IPR enthält die Qualifikation nicht die sog. „zweistufige Qualifikation“. Wenn das Gericht die inländischen Kollisionsnormen anwendet, wird nach der lex fori qualifiziert. Wenn die Qualifikation nach der lex fori schlecht paßt, z.B. beim \textsl{renvoi}, kann nach dem zur Anwendung gewählten ausländischen Recht qualifiziert werden.$[$FN 79$]$

$[$FN 79$]$ Cf. LIN, Die gegenwärtige Entwicklung des chinesischen Internationalen Privatrechts --- IPR-Gesetzesentwurf in der VR China, IPRax 1995, S. 334-337 (335).}
\end{fragmentpart}
\begin{fragmentpart}{Original \cite[S.~335 Z.~51--58]{Ma-1995}}
\enquote{Außerdem wird in Art. 10 des IPR-Entwurfs die Qualifikation geregelt. Im chinesischen IPR enthält die „Qualifikation“ nicht die sog. „zweistufige Qualifikation“. Wenn das Gericht die inländischen Kollisionsnormen anwendet, wird nach der „lex fori“ qualifiziert. Wenn die Qualifikation nach der chinesischen „lex fori“ schlecht paßt, kann nach dem zur Anwendung gewählten ausländischen Recht qualifiziert werden.}
\end{fragmentpart}
\begin{fragmentpart}{Anmerkung}
Die Quelle wird, für den Verfasser ungewöhnlich, ohne stilistische oder inhaltliche Überarbeitung übernommen, obwohl sich beides angeboten hätte. Der zweite Satz wirkt ungelenk, der dritte thematisiert eine Selbstverständlichkeit (nämlich dass ein Gericht die eigenen Kollisionsnormen anwendet), der vierte trifft eine Aussage, die für eine Arbeit über die Qualifikation zu undifferenziert ist (nämlich dass eine Qualifikation „schlecht paßt“). Ma wird in der Fußnote genannt, allerdings unter dem Vornamen, Lin. Aus dem Verweis wird nicht ersichtlich, dass der Verfasser 47 zusammenhängende Wörter aus der Quelle übernommen hat.
\end{fragmentpart}
\end{fragment}
\phantomsection{}
\belowpdfbookmark{Fragment 164 17--23,~110--112}{Lm-Fragment-164-17}
\hypertarget{Lm-Fragment-164-17}{}
\begin{fragment}
\begin{fragmentpart}{Dissertation S.~164 Z.~17--23,~110--112 (VerschärftesBauernopfer)}
\enquote{(Art. 18 ... verlangt...) die Qualifikation nach der lex fori: Sie ist immer nach kubanischem Recht vorzunehmen. Inhaltlich entspricht die Regel dem Art. 12 Abs. 1 span. C.C.:
:\textsl{Art. 18. $[$Qualifikation$]$}
:Die Qualifikation der tatsächlichen Ereignisse der Rechtsgeschäfte zur Bestimmung der anwendbaren Norm im Fall des Gesetzeskonfliktes geschieht immer nach kubanischem Recht.$[$FN 98$]$

$[$FN 98$]$ Fast 100 Jahre galt in Kuba der spanische Código Civil von 1889 einschließlich seiner Vorschriften zum IPR. Seit dem 16. Juli 1987 hat Kuba ein eigenes Zivilgesetzbuch, das am 12. April 1988 in Kraft getreten ist. Cf. im übrigen HUZEL, Neues internationales Privatrecht in Kuba, IPRax 10 (1990), S. 416-419 (417, 418).}
\end{fragmentpart}
\begin{fragmentpart}{Original \cite[S.~416--418 Z.~9--13,~4--6,~8--11]{Huzel-1990}}
\enquote{Fast 100 Jahre galt in Kuba der spanische Código Civil (CC) von 1889$[$FN 1$]$ einschließlich seiner Vorschriften zum IPR$[$FN 2$]$. Seit 16. 7. 1987 hat Kuba ein eigenes Zivilgesetzbuch$[$FN 3$]$, in Kraft getreten aufgrund seiner dritten Schlußbestimmung am 12. 4. 1988. $[$...$]$

$[$S. 417$]$

Art. 18 schreibt die \textsl{Qualifikation} nach der lex fori vor: Sie ist „immer nach kubanischem Recht“ vorzunehmen. Inhaltlich entspricht die Regel Art. 12.1 span. CC$[$FN 23$]$. $[$...$]$

$[$S. 418$]$

Art. 18. $[$Qualifikation$]$

Die Qualifikation der tatsächlichen Ereignisse der Rechtsgeschäfte zur Bestimmung der anwendbaren Norm im Fall des Gesetzeskonfliktes geschieht immer nach kubanischem Recht.}
\end{fragmentpart}
\begin{fragmentpart}{Anmerkung}
Textstücke aus drei Seiten werden collagiert und nur leicht verändert. Der Verfasser eignet sich dabei auch den Vergleich mit dem spanischen Recht und die Übersetzung von Huzel einschließlich der eingefügten nichtamtlichen Überschrift an. Der Verweis \textquotedbl{}Cf. im übrigen HUZEL\textquotedbl{} lässt nicht erkennen, dass Huzel der Autor bzw. Übersetzer dieses Textes ist.
\end{fragmentpart}
\end{fragment}
\phantomsection{}
\belowpdfbookmark{Fragment 169 8--22}{Lm-Fragment-169-08}
\hypertarget{Lm-Fragment-169-08}{}
\begin{fragment}
\begin{fragmentpart}{Dissertation S.~169 Z.~8--22 (BauernOpfer)}
\enquote{Die Qualifikation im rumänischen IPR$[$114$]$ erfolgt gemäß Art. 3 des IPR-Gesetzes grundsätzlich nach rumänischem Recht, unabhängig davon, ob es sich um ein Rechtsinstitut im ganzen oder nur um ein konkretes Rechtsverhältnis mit ausländischem Element handelt. Entsprechend bestimmt Art. 159 Abs. II, daß die Unterscheidung zwischen Verfahrens- und materiellrechtlichen Regeln  ausschließlich nach der lex fori vorzunehmen ist. Nach dem allgemeinen Grundsatz in Art. 3, wie auch aufgrund von Art. 159 Abs. II, erfolgt die Bestimmung der maßgeblichen Kollisionsnorm aus dem Katalog der im EPRG von 1992$[$115$]$ enthaltenen und für die rumänischen Justiz- und Verwaltungsbehörden verbindlichen Regeln nach den vom rumänischen Recht aufgestellten Definitionen und Beurteilungskriterien. Mit anderen Worten: die Qualifikation erfolgt lege fori.

In besonderen Fällen (z.B. beim \textsl{renvoi}) kann jedoch üblicherweise die Bestimmung der maßgeblichen Kollisionsnorm auch über die Qualifikation gemäß der lex causae erfolgen. Wenn Gegenstand der Qualifikation die Unterscheidung zwischen beweglichen und unbeweglichen Sachen ist, entscheidet das Recht des Ortes, an dem sie sich befinden oder an dem sie belegen sind (\textsl{lex rei sitae}) über ihre Rechtsnatur und damit auch über den Inhalt der sie betreffenden dinglichen Rechte (Art. 50). Ebenso kann die Feststellung der Legitimität von Rechtshandlungen \textsl{stricto sensu} sowohl gemäß der lex fori erfolgen, als auch nach Art. 107 gemäß dem Recht des Staates, in dem die Rechtshandlung stattgefunden hat. Weil die hier angeführten Bestimmungen eine Qualifikation $[$lege causae nur ausnahmsweise gestatten, um die Prädominanz der lex fori abzubauen, dürfen sie nicht über den von ihnen umschriebenen Anwendungsbereich hinaus erweiternd ausgelegt werden.$[$116$]$$]$



$[$114$]$ Über die Qualifikation nach rumänischem Recht cf. Căpăţînă, Identification de loi applicable à une situation concrète contenant des éléments d'extranéité, dans le cas d'un conflit de qualifications, Revue roumaine des sciences sociales, Sér. de sciences juridiques 10 (1966), S. 97-118.

$[$115$]$ Cf. Căpăţînă, Das neue rumänische IPR, RabelsZ 58 (1994), S. 465-522.

$[$116$]$ So Capatina $[$sic$]$, ibid., S. 476f.}
\end{fragmentpart}
\begin{fragmentpart}{Original \cite[S.~476--477 Z.~476,~18,~477,~12]{Capatina-1994}}
\enquote{Die Qualifikation$[$12$]$ erfolgt, unabhängig davon, ob es sich um ein Rechtsinstitut im ganzen oder nur um ein konkretes Rechtsverhältnis mit ausländischem Element handelt, gemäß Art. 3 grundsätzlich nach rumänischem Recht. Entsprechend bestimmt Art. 159 II, daß die Unterscheidung zwischen Verfahrens- und materiellrechtlichen Regelungen ausschließlich nach rumänischem Recht vorzunehmen ist. Nach dem allgemeinen Grundsatz in Art. 3 wie auch aufgrund von Art. 159 II erfolgt die Bestimmung der maßgeblichen Kollisionsnorm aus dem Katalog der im IPR-Gesetz enthaltenen und für die rumänischen Justiz- und Verwaltungsbehörden verbindlichen nach den vom rumänischen Recht aufgestellten Definitionen und Beurteilungskriterien. Mit anderen Worten, die Qualifikation erfolgt lege fori.

In besonderen Fällen kann jedoch nach dem IPR-Gesetz die Bestimmung der maßgeblichen Kollisionsnorm auch über die Qualifikation gemäß der lex

$[$Seite 612$]$

causae erfolgen. So wird die lex fori dann verdrängt, wenn Gegenstand der Qualifikation die Unterscheidung zwischen beweglichen und unbeweglichen Sachen ist. Deren Rechtsnatur sowie der Inhalt der sie betreffenden dinglichen Rechte beurteilen sich gemäß Art. 50 nach dem Recht des Ortes, an dem sie sich befinden oder an dem sie belegen sind, d. h. nach der lex rei sitae. Ebenso erfolgt für Rechtshandlungen stricto sensu die Feststellung ihres erlaubten oder unerlaubten Charakters nicht gemäß der lex fori, sondern nach Art. 107 gemäß dem Recht des Staates, in dem die Rechtshandlung stattgefunden hat. Weil die hier angeführten Bestimmungen eine Qualifikation lege causae nur ausnahmsweise gestatten, um die Prädominanz der lex fori abzubauen, dürfen sie nicht über den von ihnen umschriebenen Anwendungsbereich hinaus erweiternd ausgelegt werden.


$[$12$]$ Weiterführend O. Căpăţînă, Identification de la loi applicable à une situation concrète contenant des éléments d’extranéité, dans le cas d’un conflit de qualifications: Revue roumaine des sciences sociales, Sér. de sciences juridiques 10 (1966) 97-118; ders., Conflictul de calificari referitor ...}
\end{fragmentpart}
\begin{fragmentpart}{Anmerkung}
Weitgehend wörtliche Übernahme mit nur gelegentlicher sprachlicher Überarbeitung. Die Vorlage wird erst in Fußnote 115 erwähnt, während Fußnote 114 den Eindruck vermittelt, Lm beziehe sich auf den französischsprachigen Aufsatz von Căpăţînă.
\end{fragmentpart}
\end{fragment}
\phantomsection{}
\belowpdfbookmark{Fragment 170 1--3}{Lm-Fragment-170-01}
\hypertarget{Lm-Fragment-170-01}{}
\begin{fragment}
\begin{fragmentpart}{Dissertation S.~170 Z.~1--3 (BauernOpfer)}
\enquote{$[$Weil die hier angeführten Bestimmungen eine Qualifikation$]$ lege causae nur ausnahmsweise gestatten, um die Prädominanz der lex fori abzubauen, dürfen sie nicht über den von ihnen umschriebenen Anwendungsbereich hinaus erweiternd ausgelegt werden.$[$FN 116$]$

$[$FN 116$]$ So CAPATINA $[$sic$]$, ibid., S. 476f.}
\end{fragmentpart}
\begin{fragmentpart}{Original \cite[S.~477 Z.~9--12]{Capatina-1994}}
\enquote{Weil die hier angeführten Bestimmungen eine Qualifikation lege causae nur ausnahmsweise gestatten, um die Prädominanz der lex fori abzubauen, dürfen sie nicht über den von ihnen umschriebenen Anwendungsbereich hinaus erweiternd ausgelegt werden.}
\end{fragmentpart}
\begin{fragmentpart}{Anmerkung}
Die letzten drei Zeilen der in \hyperlink{Lm-Fragment-169-08}{Fragment\_169\_08} dokumentierten Übernahme.
\end{fragmentpart}
\end{fragment}
\phantomsection{}
\belowpdfbookmark{Fragment 170 22--30,~107--108}{Lm-Fragment-170-11}
\hypertarget{Lm-Fragment-170-11}{}
\begin{fragment}
\begin{fragmentpart}{Dissertation S.~170 Z.~22--30,~107--108 (BauernOpfer)}
\enquote{Eine Neuerung ist der Grundsatz aus Art. 1223 Punkt 2 ZGB. Darin wird festgelegt, daß das (Sach-)Recht anzuwenden ist, das zu dem betreffenden Rechtsverhältnis die engsten Verbindungen aufweist. Allerdings soll dieser Grundsatz erst dann Anwendung finden, wenn andere Anknüpfungsregeln gemäß Art. 1223 Punkt 1 ZGB versagt haben. Neu ist auch die \textsl{renvoi}-Regelung des Art. 1230 ZGB. Gemäß Art. 1230 Punkt 1 ZGB sind Verweisungen des Abschnitts VII auf ausländisches Recht grundsätzlich Sachnormverweisungen. $[$FN 119$]$ In Art. 1224 ZGB wird das Qualifikationsproblem nach der $[$lex fori gesetzlich geregelt, ...$]$

$[$FN 118$]$ BOGUSLAWSKIJ \& HÖFER, Neue Entwicklungen im russischen Internationalen Privatrecht, IPRax 18 (1998), S: $[$sic$]$ 41-43 (43).

$[$FN 119$]$ Ausnahmen enthält Art. 1230 Punkt 2 ZGB. Verweisungen aufgrund der Vorschriften über das Personalstatut (Art. 1233), über die Geschäftsfähigkeit (Art. 1235) und über das $[$Namensrecht (Art. 1236), ausländischer Staatsbürger und Staatenloser und über die Vormundschaft und Pflegeschaft (Art. 1239) sind demzufolge Gesamtverweisungen.$]$}
\end{fragmentpart}
\begin{fragmentpart}{Original \cite[S.~43 Z.~46--56,~1--9]{Boguslawskij-1998}}
\enquote{Eine Neuerung ist der Grundsatz aus Art. 1223 Pkt. 2 ZGB. Darin wird festgelegt, daß das Recht anzuwenden ist, das zu dem betreffenden Rechtsverhältnis die engsten Verbindungen aufweist. Allerdings soll der Grundsatz erst dann Anwendung finden, wenn andere Anknüpfungsregeln gem. Art. 1223 Pkt. 1 ZGB versagt haben. In Art. 1224 ZGB wird das Qualifikationsproblem im Sinne der lex-fori-Theorie gesetzlich geregelt. Gemäß Art. 1227 Pkt. 1 ZGB gilt faktisch das Gegenseitigkeitsprinzip. Davon soll dann eine Ausnahme gemacht werden, wenn ausländisches Recht aufgrund eines russischen Gesetzes nach den Grundsätzen der Gegenseitigkeit angewendet werden soll.

Neu ist auch die renvoi-Regelung des Art. 1230 ZGB. Gemäß Art. 1230 Pkt. 1 ZGB sind Verweisungen des Abschnitts VII auf ausländisches Recht grundsätzlich Sachnormverweisungen. Ausnahmen enthält Art. 1230 Pkt. 2 ZGB. Verweisungen aufgrund der Vorschriften über das Personalstatut (Art. 1233), über die Geschäftsfähigkeit (Art. 1235) und über das Namensrecht (Art. 1236) ausländischer Staatsbürger und Staatenloser und über die Vormundschaft und Pflegschaft (Art. 1239) sind demzufolge Gesamtverweisungen.}
\end{fragmentpart}
\begin{fragmentpart}{Anmerkung}
Der Text wird fast vollständig wörtlich übernommen. Ein Satz wird verschoben, ein längerer Satz wird in Fußnote 119 verlagert. Boguslawski und Höfer werden im vorausgehenden Satz als Beleg für \textquotedbl{}einige wesentliche Neuerungen\textquotedbl{} im besprochenen Entwurf genannt.
\end{fragmentpart}
\end{fragment}
\phantomsection{}
\belowpdfbookmark{Fragment 181 129--134}{Lm-Fragment-181-129}
\hypertarget{Lm-Fragment-181-129}{}
\begin{fragment}
\begin{fragmentpart}{Dissertation S.~181 Z.~129--134 (KomplettPlagiat)}
\enquote{$[$FN 11$]$ Die \textsl{deutsche} Rechtsprechung beruft sich seit langem auf den Grundsatz, daß eine im inländischen materiellen Recht vorgenommene Einordnung auch für den Anwendungsbereich der inländischen Kollisionsnormen von Bedeutung ist. Cf. z.B. BGH 12.7.1965, BGHZ 44, 121, 124 = IPRspr. 1964-65 Nr. 95b m.w.N.; BGH 2.3.1979, BGHZ 73, 370, 373 = IPRspr. 1979, Nr. 3b, S. 18; BGH 7.11.1979, BGHZ 75, 241, 249 = IPRspr. 1979 Nr. 75, S. 259f.;}
\end{fragmentpart}
\begin{fragmentpart}{Original \cite[S.~109 Z.~102--106]{Kropholler-1997}}
\enquote{$[$FN 8$]$ Die deutsche Rechtsprechung beruft sich seit langem auf den Grundsatz, daß eine im inländischen materiellen Recht vorgenommene Einordnung auch für den Anwendungsbereich der inländischen Kollisionsnormen von Bedeutung ist. Vgl. z.B. BGH 12.7. 1965, BGHZ 44, 121,124 = IPRspr. 1964-65 Nr. 95 b S. 311 m. w. Nachw.; 2.3.1979, BGHZ 73, 370, 373 = IPRspr. 1979 Nr. 3 b S. 18; 7.11. 1979, BGHZ 75, 241, 249 = IPRspr. 1979 Nr. 75 S. 259 f.}
\end{fragmentpart}
\begin{fragmentpart}{Anmerkung}
Keine Erwähnung von Kropholler. Lediglich die Zitierweise wird angepasst („Vgl.“ \textrightarrow{} „Cf.“, Abk. „BGH“ hinzu).
\end{fragmentpart}
\end{fragment}
\phantomsection{}
\belowpdfbookmark{Fragment 182 109--112}{Lm-Fragment-182-109}
\hypertarget{Lm-Fragment-182-109}{}
\begin{fragment}
\begin{fragmentpart}{Dissertation S.~182 Z.~109--112 (BauernOpfer)}
\enquote{$[$FN 14$]$ Bei einer \textsl{funktionellen} oder \textsl{teleologischen Qualifikation} wird die Funktion / der Zweck des in der Kollisionsnorm gewählten Verweisungsbegriffs mit der Funktion oder dem Zweck des in Rede stehenden materiellen Rechtsinstituts verglichen. Cf. KROPHOLLER, IPR3, § 17.}
\end{fragmentpart}
\begin{fragmentpart}{Original \cite[S.~111 Z.~18--21]{Kropholler-1997}}
\enquote{Bei einer funktionellen oder teleologischen Qualifikation wird die Funktion oder der Zweck des in der Kollisionsnorm gewählten Verweisungsbegriffs mit der Funktion oder dem Zweck des in Rede stehenden materiellen Rechtsinstituts verglichen.}
\end{fragmentpart}
\begin{fragmentpart}{Anmerkung}
Es fehlen (nur) die Anführungszeichen.
\end{fragmentpart}
\end{fragment}
\phantomsection{}
\belowpdfbookmark{Fragment 184 13--18}{Lm-Fragment-184-13}
\hypertarget{Lm-Fragment-184-13}{}
\begin{fragment}
\begin{fragmentpart}{Dissertation S.~184 Z.~13--18 (BauernOpfer)}
\enquote{Qualifikationsfragen betreffen den Anwendungsbereich einzelner Kollisionsnormen. Sie sind daher aus dem Zweck der einzelnen Kollisionsnorm zu beantworten. Man muß feststellen, welchen Gehalt an internationalprivatrechtlicher Gerechtigkeit die einzelne Kollisionsnorm hat, welches internationalprivatrechtliche Interesse sie schützt. Dabei ist zu bedenken, daß es eine umfassende Einheitslösung \textsl{„des“} Qualilifikationsproblems $[$sic$]$ nicht gibt.$[$FN 22$]$

$[$FN 22$]$ So KEGEL, IPR7, § 7 III d, S. 259.}
\end{fragmentpart}
\begin{fragmentpart}{Original \cite[S.~259 Z.~4--10]{Kegel-1995}}
\enquote{Qualifikationsfragen gibt es in Massen. Sie betreffen den „Geltungsbereich“ der einzelnen Kollisionsnormen. Sie sind daher aus dem Zweck der einzelnen Kollisionsnormen zu beantworten. Man muß feststellen, welchen Gehalt an internationalprivatrechtlicher Gerechtigkeit die einzelne Kollisionsnorm hat, welches internationalprivatrechtliche Interesse sie schützt. Eine allgemeine Lösung „des“ Qualifikationsproblems gibt es nicht.}
\end{fragmentpart}
\begin{fragmentpart}{Anmerkung}
Dem Verweis auf \textquotedbl{}So Kegel\textquotedbl{} in Fußnote 22 kann man als offene Paraphrase werten, in der auch etwas längere wörtliche Übernahmen zulässig sein könnten; alternativ könnte man dieses Fragment entsprechend als \textquotedbl{}verdächtig\textquotedbl{} werten. Dafür ist diese Passage allerdings sehr lang.

Kegelsche Stilelemente --- hier die kurzen, prägnanten, oft umgangssprachlich formulierten Sätze --- werden im ersten und letzten Satz entfernt.
\end{fragmentpart}
\end{fragment}
\phantomsection{}
\belowpdfbookmark{Fragment 184 28--30}{Lm-Fragment-184-28}
\hypertarget{Lm-Fragment-184-28}{}
\begin{fragment}
\begin{fragmentpart}{Dissertation S.~184 Z.~28--30 (BauernOpfer)}
\enquote{Herkömmlicherweise wird davon ausgegangen, man müsse drei Ansichten zum \textsl{Qualifikationsstatut} unterscheiden: qualifiziert werden solle nach der lex fori, oder wie dargestellt nach der lex causae oder autonom.}
\end{fragmentpart}
\begin{fragmentpart}{Original \cite[S.~204 Z.~11--14]{Weber-1986}}
\enquote{Die herkömmlichen Darstellungen zur Qualifikation gehen in aller Regel davon aus, daß man drei Ansichten zum Qualifikationsstatut unterscheiden könne: qualifiziert werden solle nach der \textsl{lex fori}, nach der \textsl{lex causae} oder autonom.}
\end{fragmentpart}
\begin{fragmentpart}{Anmerkung}
Beginn einer längeren großteils wortwörtlichen Übernahme. In \hyperlink{Lm-Fragment-185-02}{Fragment 185 02} geht es weiter.
\end{fragmentpart}
\end{fragment}
\phantomsection{}
\belowpdfbookmark{Fragment 185 2--6}{Lm-Fragment-185-02}
\hypertarget{Lm-Fragment-185-02}{}
\begin{fragment}
\begin{fragmentpart}{Dissertation S.~185 Z.~2--6 (BauernOpfer)}
\enquote{Hinzu kommen Varianten und Kombinationen beider Wege. Unter Wortaspekten gesehen: Die Qualifikation \textsl{lege fori} hat ihre Wurzeln in einer positivistisch-nationalistischen Grundhaltung;$[$FN 23$]$ die \textsl{lege causae}-Qualifikation und die autonomen Orientierungen sind Ausdruck einer internationalistischen Grundhaltung.$[$FN 24$]$ $[$...$]$ $[$FN 25$]$

$[$FN 23$]$ Cf. supra. Kapitel III 2 a und Fn. 31.

$[$FN 24$]$ Cf. supra. Kapitel IV 1, 5 und Fn. 32.

$[$FN 25$]$ Cf. Weber, a.a.O., S. 204-214, 231-245; $[$...$]$}
\end{fragmentpart}
\begin{fragmentpart}{Original \cite[S.~204 Z.~14--15,~18--21]{Weber-1986}}
\enquote{(Hinzu kommen dann Varianten und Kombinationen dieser drei hauptsächlichen Richtungen.) $[$...$]$ wobei die positivistische (nationalistische) Grundströmung sich in der Qualifikation \textsl{lege fori} äußern soll, während die Qualifikation \textsl{lege causae} oder die autonome Qualifikation oder beide Ausdrucksformen einer internationalistischen Grundhaltung sein sollen.}
\end{fragmentpart}
\begin{fragmentpart}{Anmerkung}
Im weiteren Verlauf gibt es einen unspezifischen Verweis auf Weber (1986). Die vorliegende wörtliche Übernahme wird aber in Art und Umfang dadurch nicht spezifiziert.
\end{fragmentpart}
\end{fragment}
\phantomsection{}
\belowpdfbookmark{Fragment 186 1--7}{Lm-Fragment-186-01}
\hypertarget{Lm-Fragment-186-01}{}
\begin{fragment}
\begin{fragmentpart}{Dissertation S.~186 Z.~1--7 (VerschärftesBauernopfer)}
\enquote{$[$Es markierte einen Durchbruch, als RABEL$[$FN26$]$ jene Systembegriffe, die von Kollisionsnormen des IPR benutzt werden, aus ihrer begrifflichen Abhängigkeit$]$ vom „dazugehörigen“ Sachrecht löste und damit die Begriffsbildung des internationalen Privatrechts emanzipierte. Eine Kollisionsnorm ziele auf internationale Brauchbarkeit. Sie regele daher unabhängig von der Systematik einzelnen Sachrechts einen bestimmten Sachbereich. Welchen Sachbereich der Systembegriff einer Kollisionsnorm meine, sei wissenschaftlich durch Rechtsvergleichung$[$FN 27$]$ zu ermitteln: was rechtsvergleichend zusammengehöre, gehöre auch internationalprivatrechtlich zusammen.$[$FN 28$]$
----
$[$$[$FN 26$]$ BYSTRICKÝ, in: Wiemann, a.a.O. (Kapitel VII, Fn. 3), S. 61 Fn. 30, informiert, daß es nicht RABEL war, sondern KRCMÁR, Einführung in das Internationale Privatrecht, Bd. 4, Prag 1906, S. 248, (tschechisch), der entsprechende Vorschläge schon 1906 als erster gemacht hat$]$

$[$$[$FN 27$]$ VON BAR, IPR I, Rn. 594 schreibt  $[$...$]$$]$

$[$$[$FN 28$]$ Cf. KEGEL, RabelsZ 1990, S. 1-23. Das Absehen von einer Vergleichung nur einzelner Rechte macht ja gerade den Sinn der rechtsvergleichende Qualifikation aus.$]$}
\end{fragmentpart}
\begin{fragmentpart}{Original \cite[S.~251--252 Z.~43--44,~1--2,~6--12]{Kegel-1995}}
\enquote{Es war daher ein Durchbruch, als Ernst Rabel (Bild vor S. 145) die Systembegriffe, die von Kollisionsnormen des IPR benutzt werden, aus 

$[$S. 252$]$

ihrer Abhängigkeit von irgendeinem materiellen Recht (insbesondere lex fori und lex causae) erlöste, die Begriffsbildung des IPR „emanzipierte“ . $[$...$]$ Eine Kollisionsnorm ziele jedoch auf internationale Brauchbarkeit. Sie regele daher ganz unabhängig von der Systematik irgendeines materiellen Rechts einen bestimmten Sachbereich. Welchen Sachbereich der Systembegriff einer Kollisionsnorm meine, sei wissenschaftlich durch Rechtsvergleichung zu ermitteln: was rechtsvergleichend zusammengehöre, gehöre auch internationalprivatrechtlich zusammen.}
\end{fragmentpart}
\begin{fragmentpart}{Anmerkung}
Die übernächste Fußnote (30) verweist auf die Quelle, aber nur für Kegels Kritik an der Theorie Rabels. Alternativ ist eine Einordnung als \textquotedbl{}Verschleierung\textquotedbl{} denkbar.

Man beachte, dass in FN 28 auf eine andere Schrift Kegels verwiesen wird.
\end{fragmentpart}
\end{fragment}
\phantomsection{}
\belowpdfbookmark{Fragment 187 3--13}{Lm-Fragment-187-03}
\hypertarget{Lm-Fragment-187-03}{}
\begin{fragment}
\begin{fragmentpart}{Dissertation S.~187 Z.~3--13 (BauernOpfer)}
\enquote{Zwei Ausgangspunkte lassen sich ausmachen: In einem System allseitiger Kollisionsnormen darf es methodisch zunächst keinen Unterschied machen, ob eine Rechtsfrage des in- oder des ausländischen Rechts zu qualifizieren ist. Auch eine Differenzierung zwischen geläufigem und fremdartigem Rechtsstoff trägt hier nicht mehr. Angesichts vorhandener Kodifikationen besteht ein Problem insoweit, als ein innerkodifikatorischer Ausgleich zwischen den Kollisionsnormen des Forums zu bewerkstelligen ist.$[$FN 35$]$ Der zweite Gesichtspunkt, auf dem eine autonome IPR-Qualifikation ruht, stellt auf die eben erwähnte Notwendigkeit ab, die kodifizierten Kollisionsnormen des Forums untereinander auszubalancieren und in ihrer Wirkungsweise zu optimieren. Die Systembegriffe sollen nur aus sich selbst heraus isoliert interpretiert werden, eben autonom.

$[$$[$FN 35$]$ Cf. VON BAR, IPR I, Rn. 604, S. 517.$]$}
\end{fragmentpart}
\begin{fragmentpart}{Original \cite[S.~516--517 Z.~35--39,~1--2,~19--24]{Von-Bar-1987}}
\enquote{Zwei Ausgangspunkte wird man insoweit festmachen können. In einem System allseitiger Kollisionsnormen darf es methodisch zunächst keinen Unterschied machen, ob eine Rechtsfrage (oder eine Norm$[$FN 397$]$) des in- oder des auländischen $[$sic$]$ Rechts zu qualifizieren ist, und auch eine Differenzierung zwischen (im Kern) geläufigem und fremdartigem Rechtsstoff trägt hier nicht mehr. Das Problem ist hier nur 

$[$S. 517$]$

noch eines der innerkodifikatorischen Balance zwischen den \textsl{Kollisionsnormen} des Forums. $[$...$]$

Der zweite allgemeine Gesichtspunkt, auf dem die Methode der autonomen IPR-Qualifikation aufbaut, knüpft an die eben erwähnte Notwendigkeit an, die kodifizierten Kollisionsnormen des Forums untereinander auszubalancieren und in ihrer Wirkungsweise zu optimieren. Wir können solches nämlich nicht allein dadurch bewirken, daß die Systembegriffe nur aus sich selbst heraus (isoliert) interpretiert werden.}
\end{fragmentpart}
\begin{fragmentpart}{Anmerkung}
Auf von Bar wird in Fußnote 35 lediglich zum Vergleich verwiesen. Im direkten Anschluss an die oben wiedergegebene plagiierte Passage folgt ein korrekt ausgewiesenes Zitat aus demselben Werk.
\end{fragmentpart}
\end{fragment}
\phantomsection{}
\belowpdfbookmark{Fragment 188 5--12}{Lm-Fragment-188-05}
\hypertarget{Lm-Fragment-188-05}{}
\begin{fragment}
\begin{fragmentpart}{Dissertation S.~188 Z.~5--12 (BauernOpfer)}
\enquote{Im Gegensatz zu einer streng auf die materiellrechtliche lex fori augerichteten Qualifikation stellt eine funktionelle Qualifikation auch die kollisionsrechtliche Einordnung solcher Rechtsinstitute sicher, die dem eigenen materiellen Recht unbekannt sind, weil die Begrifflichkeit des eigenen Systems zwar maßgeblich, aber nicht als allein „subsumtiv vorgeprägt“ erscheint. In diesem Sinne betont der BGH mit Recht, daß die im IPR verwendeten Rechtsbegriffe oft weit ausgelegt werden müssen, um ausländischen Regelungen gerecht werden zu können.$[$FN 40$]$
----
$[$FN 40$]$ So KROPHOLLER, IPR3, § 17 I, S. 112; $[$...$]$}
\end{fragmentpart}
\begin{fragmentpart}{Original \cite[S.~112 Z.~6--12]{Kropholler-1997}}
\enquote{Anders als eine streng auf die materiellrechtliche lex fori ausgerichtete Qualifikation ermöglicht eine funktionelle Qualifikation die kollisionsrechtliche Einordnung auch solcher Rechtsinstitute, die dem eigenen Sachrecht unbekannt sind. Die funktionelle Qualifikation wird deshalb auch vom BGH praktiziert, der mit Recht betont, daß die im IPR verwendeten Rechtsbegriffe oft weit ausgelegt werden müssen, um ausländischen Regelungen gerecht werden zu können$[$FN 4$]$.
----
$[$FN 4$]$ Wegweisend und mit einer entgegenstehenden Rechtsprechung des RG brechend BGH 22.3. 1967, BGHZ47, 324 (336) = IPRspr. 1966—67 Nr. 90 S. 298: $[$...$]$}
\end{fragmentpart}
\begin{fragmentpart}{Anmerkung}
Der Gedankengang wird vollständig übernommen, an einer Stelle leicht ergänzt. Der Text wird teilweise geändert, aber 9 und 18 zusammenhängender Wörter werden nicht als wörtliches Zitat gekennzeichnet.
\end{fragmentpart}
\end{fragment}
\phantomsection{}
\belowpdfbookmark{Fragment 193 2--5}{Lm-Fragment-193-02}
\hypertarget{Lm-Fragment-193-02}{}
\begin{fragment}
\begin{fragmentpart}{Dissertation S.~193 Z.~2--5 (BauernOpfer)}
\enquote{Der lex causae-Theorie fällt es wesentlich leichter als der lex-fori-Theorie, mit Systemlücken des eigenen materiellen Rechts (unbekannten Rechtsinstituten) fertig zu werden.$[$FN 61$]$ Die Lücke wird dann durch angemessene Analogien zu füllen sein.$[$FN 62$]$
----

$[$FN 61$]$ Cf. KEGEL, IPR7, § 7 III1 a), S. 250; Meierhoff, a.a.O. (Kapitel I, Fn. 5), S. 208.

$[$FN 62$]$ Cf. WOLFF, IPR3, S. 57.}
\end{fragmentpart}
\begin{fragmentpart}{Original \cite[S.~250 Z.~18--22]{Kegel-1995}}
\enquote{Der \textsl{lex causae}-Theorie fällt es wesentlich leichter als der \textsl{lex fori}-Theorie, mit \textsl{Systemlücken} des eigenen materiellen Rechts fertig zu werden wie im Fall der ägyptischen Legitimation (oben II 2). \textsl{Wolff} sieht richtig, daß hier das eigene IPR zu ergänzen ist: „Die Lücke wird dann durch angemessene Analogien zu füllen sein.“ (IPR 57.)}
\end{fragmentpart}
\begin{fragmentpart}{Anmerkung}
Kegel wird in Fußnote 61 an erster Stelle erwähnt, Wolff in Fußnote 62. Es wird aber nicht ersichtlich, dass die Formulierungen von Kegel bzw. Wollf stammen, und dass Kegel die passende Stelle von Wolff ausgesucht hat.
\end{fragmentpart}
\end{fragment}
\phantomsection{}
\belowpdfbookmark{Fragment 195 10--21}{Lm-Fragment-195-10}
\hypertarget{Lm-Fragment-195-10}{}
\begin{fragment}
\begin{fragmentpart}{Dissertation S.~195 Z.~10--21 (BauernOpfer)}
\enquote{Das größte Hindernis für eine unvoreingenommene Betrachtung des Qualifikationsproblems war stets die Halbwahrheit, mit der der „Gegenstand“ der Kollisionsnorm fixiert wurde: Nachdem man den „Ansatzwechsel“ vom Herrschaftsbereich von Gesetzen zur Anwendbarkeit von einzelstaatlichem Recht auf ein „Rechtsverhältnis“ (Sachverhalt) zur Hauptleistung SAVIGNYS und somit zur Grundlage des gesamten modernen IPR-Systems stilisiert hatte, war die „Schwierigkeit“ der Qualifikationsfrage programmiert. Denn schließlich geht es bei ihr gerade darum, ob bestimmte ausländische Gesetze von der jeweiligen (heimischen) Kollisionsnorm erfaßt, also für anwendbar erklärt werden; sie kann kaum gelöst werden, wenn man sich mit Hilfe einer lex fori-Theorie sogleich den Blick auf diese konkreten Gesetze wieder verbaut.$[$FN 68$]$ Den Ausweg suchen mehrere Methoden / Theorien einer Stufenqualifikation:

$[$FN 68$]$ Cf. SCHURIG, Kollisionsnorm und Sachrecht, S. 216.}
\end{fragmentpart}
\begin{fragmentpart}{Original \cite[S.~216 Z.~4--16]{Schurig-1981}}
\enquote{Das größte Hindernis für eine unbefangene Betrachtung des Qualifikationsproblems war stets die halbseitige Blindheit, mit der der \textquotedbl{}Gegenstand\textquotedbl{} der Kollisionsnorm fixiert wurde: Nachdem man den \textquotedbl{}Ansatzwechsel\textquotedbl{} vom Gesetz zum \textquotedbl{}Rechtsverhältnis\textquotedbl{} (Sachverhalt) zur Hauptleistung \textsl{Savignys} und Grundlage des gesamten modernen IPR-Systems stilisiert hatte, war die \textquotedbl{}Schwierigkeit\textquotedbl{} der Qualifikationsfrage programmiert. Denn bei ihr geht es gerade darum, ob \textsl{bestimmte ausländische Gesetze} von der jeweiligen Kollisionsnorm erfaßt, also für \textquotedbl{}anwendbar\textquotedbl{} erklärt werden; sie kann kaum gelöst werden, wenn man sich den Blick auf diese Gesetze verbaut. Plastisch veranschaulicht werden diese selbstgeschaffenen Schwierigkeiten und die Versuche zu ihrer Überwindung der besonders in Österreich vertretenen \textsl{\textquotedbl{}Stufenqualifikation\textquotedbl{}}$[$FN 4$]$: 

$[$FN 4$]$ \textsl{Scheucher}, Qual.; $[$...$]$}
\end{fragmentpart}
\begin{fragmentpart}{Anmerkung}
Schurig wird als Quelle genannt.
\end{fragmentpart}
\end{fragment}
\phantomsection{}
\belowpdfbookmark{Fragment 202 11--15,~113--115}{Lm-Fragment-202-11}
\hypertarget{Lm-Fragment-202-11}{}
\begin{fragment}
\begin{fragmentpart}{Dissertation S.~202 Z.~11--15,~113--115 (BauernOpfer)}
\enquote{Die \textsl{unilateralistischen} Systeme wollen das IPR wesentlich \textsl{vereinfachen}, Probleme der allseitigen Systeme ausräumen oder als Scheinprobleme entlarven. Auf diese Weise könne die Qualifikation entfallen: Das Kollisionsrecht des einen und das materielle Recht eines anderen Staates brauchten nicht mehr auf einen Nenner gebracht zu werden.$[$FN 95$]$
----
$[$FN 95$]$ Cf. SCHURIG, ibid, S. 31 m.w.N. Für QUADRI ist es lebenswichtig, das Ergebnis zu vermeiden, daß man ein ausländisches Recht anwenden muß, das auf den Fall selbst gar nicht angewandt sein will.}
\end{fragmentpart}
\begin{fragmentpart}{Original \cite[S.~31,~32 Z.~22--26,~1--4]{Schurig-1981}}
\enquote{Diese Systeme sollen $[$...$]$ das IPR auch wesentlich \textsl{vereinfachen}, Probleme der allseitigen Systeme ausräumen oder als Scheinprobleme entlarven$[$FN 91$]$. So entfällt die Qualifikation: Es brauchen nicht mehr Kollisionsrecht des einen und materielles Recht des anderen Staates auf einen Nenner gebracht zu werden; $[$...$]$

$[$S. 32$]$

Außerdem wird das --- insbesondere für \textsl{Quadri} besonders anstößige$[$FN 95$]$ --- Ergebnis vermieden, daß man ein ausländisches Recht anwenden muß das auf den Fall selbst gar nicht angewandt sein \textquotedbl{}will\textquotedbl{}.
----

$[$FN 91$]$ Vgl. dazu \textsl{Wiethölter}, Eins. KN. 43-87.

$[$FN 95$]$ Vgl. \textsl{Quadri}, Lez. 281 f. $[$...$]$}
\end{fragmentpart}
\begin{fragmentpart}{Anmerkung}
Schurig wird offen als Quelle angegeben, die wörtliche Übernahme ohne entsprechende Kennzeichnung geht aber zu weit.
\end{fragmentpart}
\end{fragment}
\phantomsection{}
\belowpdfbookmark{Fragment 203 9--13}{Lm-Fragment-203-09}
\hypertarget{Lm-Fragment-203-09}{}
\begin{fragment}
\begin{fragmentpart}{Dissertation S.~203 Z.~9--13 (BauernOpfer)}
\enquote{Nach der Lehre vom fakultativen Kollisionsrecht$[$FN 97$]$ sollen die Kollisionsnormen nicht von Amts wegen$[$FN 98$]$ zu beachten sein, sondern nur dann, wenn sich eine der Prozeßparteien auf die Anwendung fremden Rechts beruft.$[$FN 99$]$ Vorausgeschickt sei, daß sich die Literatur überwiegend kritisch zu dieser Lehre äußert.$[$FN 100$]$
----

$[$FN 97$]$ FLESSNER, Fakultatives Kollisionsrecht, RabelsZ 34 (1970), S. 547-584; ZWEIGERT, Zur Armut des IPR an sozialen Werten, RabelsZ 37 (1973), S. 435-452, 445-451; RAAPE/STURM, IPR, S. 306-308; SIMITIS, Über die Entscheidungsfindung im IPR, StAZ 1976, S. 6-15, 15 (ohne die prozessualen Komponente); STURM, Fakultatives Kollisionsrecht: Notwendigkeit und Grenzen, FS Zweigert, 1981, S. 329-351, 330-345; KOERNER, Fakultatives Kollisionsrecht in Frankreich und Deutschland, Tübingen 1995; REICHERT-FACILIDES, Fakultatives und zwingendes Kollisionsrecht, Tübingen 1995; EINSELE, Rechtswahlfreiheit im IPR, RabelsZ 60 (1996), S. 417-447, 419-421; DIES., Besprechung von \textsl{Koerner} und \textsl{Reichert-Facilides}, RabelsZ 60 (1996), S. 509-516.

$[$FN 98$]$ Vgl. z.B. BGH, Urt. v. 6.3.1995 --- II ZR 84)94, JZ 1995, S. 784-786; BHG, Urt. v. 21.9.95 ---  VII ZR 284/94,, RIW 1995, S. 1027f. Anders, wegen des \textsl{adversarial} Verfahrenssystems in common law Ländern; vgl. z.B. \textsl{Busch v Stevens} $[$1962$]$ 1 All ER 412-416 (QBD); \textsl{Bumper Development Corp. Ltd. v Commissioner of Police of the Metropolis and others (Union of India and Others, claimants)} $[$1991$]$ 4 All ER 638-649(C.A.); im letzen $[$sic$]$ Fall ging es um die Prozeßfähigkeit eines Hindu-Tempels und der Court of Appeal hat entsprechend dem Antrag des Klägers nach Hindu Recht qualifiziert. Vgl. auch wieder aus deutscher Perspektive OLG München 25.1.1989 --- 15 U 4470/87, JZ 1991, 370 mit Anm. von EBKE, Internationale Kreditverträge und das internationale Devisenrecht, JZ 1991, 335-342, 338, 341.

$[$FN 99$]$ Cf. FLESSNER, a.a.O. (Fn. 97), S. 582. So auch mit beachtenwerter Präzision, \textsl{de lege lata und de lege ferenda}, EINSELE, a.a.O. (97), S. 417-447. Der Lehre von FLESSNER folgen auch STURM und ZWEIGERT.

$[$FN 100$]$ Vgl. SCHURIG, Kollisionsnorm und Sachrecht, S. 343-350; v. OVERBECK, La théorie des „règles de conflits facultatives“ et l'autonomie de la volonté, FS Vischer, 1983, S. 257-262, 259-262; KROPHOLLER, IPR3, § 7 II 2; KEGEL, IPR7, § 15 II; von BAR, IPR I, Rn. 541; $[$Firsching/ von Hoffmann, IPR5, § 3, Rn. 131; Koerner, a.a.O. (Fn. 97), S. 42-63, 73-80, 99-115, 121-130.$]$}
\end{fragmentpart}
\begin{fragmentpart}{Original \cite[S.~419 Z.~8--12]{Einsele-1996}}
\enquote{Inhalt des fakultativen Kollisionsrechts ist die Lehre, wonach Kollisionsnormen
nicht von Amts wegen zu beachten sein sollen, sondern nur dann, wenn sich eine der Parteien auf die Anwendung fremden Rechts beruft$[$FN 6$]$. Vorausgeschickt sei, daß sich die Literatur überwiegend kritisch zu dieser Lehre äußerte$[$FN 7$]$.
----

$[$FN 6$]$ Vgl. so insbesondere \textsl{Flessner} (vorige Note) 582; ihm folgend \textsl{Fritz Sturm}, Fakultatives Kollisionsrecht: Notwendigkeit und Grenzen, in: FS Zweigert (1981) 329-351 (330-345); \textsl{Konrad Zweigert}, Zur Armut des IPR an sozialen Werten: RabelsZ 37 (1973) 435-452 (445-451).
\textsl{
$[$FN 7$]$ Vgl. etwa} Alfred E. v. Overbeck\textsl{, La théorie des »règles de conflits facultatives« et l’autonomie de la volonté, in: FS Vischer (1983) 257-262 (259-262); }Kropholler\textsl{ §7 II 2 (S. 45); }Kegel\textsl{ § 15 II; }v. Bar\textsl{, IPR I Rz. 541 ; }Neuhaus\textsl{ (oben N. 4) 66; }Firsching/v. Hoffmann\textsl{ § 3 Rz. 131; }Dörthe Koerner\textsl{, Fakultatives Kollisionsrecht in Frankreich und Deutschland (1995) insbesondere 42-63, 73-80, 99-115, 121-130, 135ff. (Studien zum ausländischen und internationalen Privatrecht, 44) (siehe dazu meine Besprechung in diesem Heft S. 509-512); vgl. auch }Erik Jayme\textsl{, Bespr. von Raape/Sturm, IPR I6 (1977): RabelsZ 43 (1979) 566-571 (569f.), der die Ansicht Sturms für widersprüchlich hält; vgl. auch }Gerhard Bolka'', Zum Parteieneinfluß auf die richterliche Anwendung des IPR: ZRvgl. 13 (1972) 241-256 (248f.).}
\end{fragmentpart}
\begin{fragmentpart}{Anmerkung}
Einsele wird neben vielen anderen Namen in Fußnote 97 erwähnt, als weiterführender Verweis (\textquotedbl{}mit beachtenswerter Präzision\textquotedbl{}) nochmals in Fußnote 99.
\end{fragmentpart}
\end{fragment}
\phantomsection{}
\belowpdfbookmark{Fragment 204 1--13}{Lm-Fragment-204-01}
\hypertarget{Lm-Fragment-204-01}{}
\begin{fragment}
\begin{fragmentpart}{Dissertation S.~204 Z.~1--13 (Verschleierung)}
\enquote{Augangspunkt $[$sic$]$ der Überlegungen ist, daß in ausländischen Rechtssystemen (vor allem im adversary system) fremdes Recht nur angewandt wird, wenn die Parteien sich darauf berufen, u.a. weil fremdes Recht zum Teil als Tatsache behandelt wird.$[$101$]$ Entscheidungen nach der lex fori gewährleisten angeblich eine höhere Qualität der Rechtsprechung.$[$102$]$ Darum soll die Anwendung des Kollisionsrechts fakultativ sein. Der Richter soll die Parteien zu einer Erklärung nur auffordern, wenn ihm das ausländische Recht bekannt und das daraus folgende Ergebnis für den konkreten Fall sicher erscheint,$[$103$]$ andernfalls sollte er nach seinem Ermessen verfahren und sich insbesondere bei kleineren Sachen zurückhalten, damit die Parteien nicht in unnötige Zweifel gestürzt oder zu unverhältnismäßigen Untersuchungen veranlaßt werden.$[$104$]$ Ein solches Verfahren soll mit gewissen Einschränkungen im internationalen Vertrags-, Delikts- Sachen-, Erbrecht und sogar in Teilen des Familienrechts möglich sein.$[$105$]$

$[$101$]$ Vgl. Flessner, a.a.O. (Fn. 97), S. 548f.; s. ferner. Fentiman, Foreign Law in English Courts, L.Q.R. 1992, S. 142-156; Taniguchi, Between Verhandlungsmaxime and Adversary System --- in Search for Place of Japanese Civil Procedure, in FS Schwab, 1990, S. 487-501. Cf. auch supra, Kapitel I lb.

$[$102$]$ Cf. Flessner, ibid., S. 550-555 m.w.N.; Einsele, a.a.O. (Fn. 97), S. 421-443.

$[$103$]$ Cf. Flessner, ibid., S. 582.

$[$104$]$ Ibid., S. 513.

$[$105$]$ Ibid., S. 566-577; s. kürzlich Einsele, a.a.O. (Fn. 97), S. 421-443.}
\end{fragmentpart}
\begin{fragmentpart}{Original \cite[S.~49 Z.~9]{Schurig-1981}}
\enquote{Ausgangspunkt der Überlegungen ist, daß in einigen ausländischen Rechtssystemen fremdes Recht nur angewandt wird, wenn die Parteien sich darauf berufen, es z.T. wie Tatsachen beweisen können$[$213$]$. Entscheidungen nach der lex fori gewährleisteten eine höhere Qualität der Rechtsprechung$[$214$]$. Darum solle Kollisionsrecht \textquotedbl{}fakultativ\textquotedbl{} sein, d.h. nur anzuwenden, wenn sich wenigstens eine Partei darauf beruft. $[$...$]$ Der Richter soll die Parteien zu einer Erklärung nur auffordern, \textquotedbl{}wenn ihm das ausländische Recht bekannt und das daraus folgende Ergebnis für den konkreten Fall sicher ist\textquotedbl{}$[$216$]$, andernfalls \textquotedbl{}sollte er nach seinem Ermessen verfahren\textquotedbl{} und sich insbesondere bei kleineren Sachen zurückhalten, damit nicht die Parteien \textquotedbl{}nur in unnötige Zweifel gestürzt oder zu unverhältnismäßigen Untersuchungen veranlaßt\textquotedbl{} werden$[$217$]$. $[$...$]$

$[$S. 50$]$

Ein solches Verfahren soll --- mit gewissen Einschränkungen --- möglich sein im internationalen Vertrags-, Delikts-, Sachen-, Erbrecht und sogar in Teilen des Familienrechts$[$220$]$.

$[$213$]$ ''Flessner', Fak KR. 548f.

$[$214$]$ Ebd. 550-555

$[$215$]$ Ebd. 567f., 578f., 581f.

$[$216$]$ Ebd. 582.

$[$217$]$ Ebd. 583.

$[$220$]$ Ebd. 566-577.}
\end{fragmentpart}
\begin{fragmentpart}{Anmerkung}
Der Text wird weitgehend unverändert übernommen, mitsamt Fußnoten, die etwas ergänzt werden. Schurigs korrekte Kennzeichnung von Flessneres Zitaten wird entfernt. Schurig S. 343-350 wird zuvor in Fußnote 100 als einer von sieben Kritikern des fakultativen Kollisionsrechts genannt.
\end{fragmentpart}
\end{fragment}
\phantomsection{}
\belowpdfbookmark{Fragment 204 22--25}{Lm-Fragment-204-22}
\hypertarget{Lm-Fragment-204-22}{}
\begin{fragment}
\begin{fragmentpart}{Dissertation S.~204 Z.~22--25 (BauernOpfer)}
\enquote{\textsl{c. Autolimitierte Sachnormen/ Eingriffsnormen und Qualifikation}

Moderne Kollisionsrechtssysteme haben ihre eigene \textsl{Janusköpfigkeit}. Die „normale“ Methode zur Auffindung des anwendbaren Rechts hat stets die Masse von Rechtsnormen durch mehr oder weniger allseitige Regeln zur Anwendung $[$berufen.$]$}
\end{fragmentpart}
\begin{fragmentpart}{Original \cite[S.~34 Z.~11--16]{Schurig-1981}}
\enquote{\textsl{2. Ordre-public-Gesetze und Systeme „autolimitierter Sachnormen“}

Was auch immer als „normale“ Methode zur Auffindung des anwendbaren Rechts gelten sollte, stets wurden der Masse von Rechtsnormen, die durch mehr oder weniger „allseitige“ Regeln berufen wurden, solche gegenübergestellt, die „außerhalb“ dieses Kollisionsrechts standen, $[$...$]$

Diese „Janusköpfigkeit“ der Systeme ist es, mit der die Theorie des internatonalen Privatrechts bis heute immer wieder zu kämpfen hat$[$FN 109$]$.

$[$FN 109$]$ Schon \textsl{Savigny}, Syst. VIII 32, $[$...$]$}
\end{fragmentpart}
\begin{fragmentpart}{Anmerkung}
Fragment ist zusammen mit der Fortsetzung auf der nächsten Seite zu sehen. Dort wird auf Schurig als eine unter mehreren Quellen genannt.
\end{fragmentpart}
\end{fragment}
\phantomsection{}
\belowpdfbookmark{Fragment 206 3--19}{Lm-Fragment-206-03}
\hypertarget{Lm-Fragment-206-03}{}
\begin{fragment}
\begin{fragmentpart}{Dissertation S.~206 Z.~3--19 (BauernOpfer)}
\enquote{d. Sonderanknüpfungen und Qualifikation

Von einer anderen Seite her bestimmt die Lehre von der sog. Sonderanknüpfung den Geltungsbereich einer besonderen Klasse von Normen. Die Sonderanknüpfung geht auf einem Gedanken von WENGLER zurück$[$FN 116$]$ und hat sich heute zumindest in Theorie und Praxis der Randgebiete des internationalen Privatrechts allmählich ausgebreitet.$[$FN 117$]$ Anders als bei den ''lois d'application immédiate\textsl{, geht es primär nicht um die vorrangige Anwendung inländischer Sachnormen. Ein treffendes Beispiel ist das Sonderprivatrecht der Artt. 29, 30 EGBGB, die eine }allseitige'' Sonderanknüpfung sonderprivatrechtlicher Vorschriften (für einzelne Typen von Verbraucherverträgen und generell für Arbeitsverträge) vorsehen. Daß es unter diesen Gesetze gibt, die unabhängig vom berufenen anwendbaren Recht anzuwenden sind, war bei der Entstehung der Sonderanknüpfungslehre eine Selbstverständlichkeit, die konstatiert wurde, ohne daß man sich zu einer umfassenden Klassifizierung dieser Gesetze herausgefordert gesehen hätte.$[$FN 118$]$

Die Sonderanküpfung $[$sic$]$ hat eine besondere Bedeutung im internationalen Devisen- und Wirtschaftsrecht,$[$FN 119$]$ auch im Kartellrecht, Wettbewerbs-,$[$120$]$ Ar-$[$beits-$[$FN 121$]$ und Konzernrecht,$[$FN 122$]$ kurz, bei allen öffentlichrechtlichen Eingriffen in private Rechtsverhältnisse$[$FN 123$]$ und auch überall dort, wo der „Schutz des Schwächeren“$[$FN 124$]$ im Vordergrund steht.$]$
----

$[$FN 116$]$ WENGLER, Die Anknüpfung des zwingenden Schuldrechts im internationalen Privatrecht, ZVglRwiss 54 (1941), S. 168-212.

$[$FN 117$]$ Cf Neuhaus, a.a.O. (Kapitel I, Fn. 38), S. 425; SCHWANDER, a.a.O., S. 316-376; SCHURIG, Kollisionsnorm und Sachrecht, S. 39-41, 322-330 m.w.N.; TSOUCA, a.a.O. (Fn. 109), S. 336-351, BECKER, Theorie und Praxis der Sonderanküpfimg im internationalen Privatrecht, Tübingen 1991.

$[$FN 118$]$ Cf. WENGLER, a.a.O. (Fn. 116), S. 168-180; Schurig, ibid., S. 39.

$[$FN 119$]$ Cf. NEUMAYER, Die Notgesetzgebung des Wirtschaftsrechts im internationalen Privatrecht, BerDGesV 2 (1957), S. 35-59; JOERGES, Vorüberlegungen zu einer Theorie des internationalen Wirtschaftsrechts, RabelsZ 43 (1979), S. 6-79, 34-39, 56f.; HÜBNER, Die methodologische Entwicklung des internationalen Wirtschaftsrechts, Konstanz 1980, S. 17-31; $[$...$]$

$[$FN 120$]$ Cf. Meessen, Zu den Grundlagen des internationalen Wirtschaftsrechts, AöR 110 (1985), S. 398-418, 407-416; Deutsch, Wettbewerbstatbestände mit Auslandsbeziehung, Stuttgart 1962, S. 21, 42f. und passim.}
\end{fragmentpart}
\begin{fragmentpart}{Original \cite[S.~39,~41 Z.~9--22,~5--13]{Schurig-1981}}
\enquote{3. \textquotedbl{}Sonderanknüpfung\textquotedbl{}

Von einer ganz anderen Seite beleuchtet den eigenen Geltungsbereich einer besonderen Klasse von Normen die Lehre von der sog. Sonderanknüpfung, die auf einen Gedanken von \textsl{Wengler} zurückgeht$[$FN 140$]$ und sich heute in Theorie und Praxis zumindest der Randgebiete des IPR zusehends ausbreitet$[$FN 141$]$.

Anders als bei den \textquotedbl{}lois d'application immédiate\textquotedbl{}, den autolimitierten Sachnormen, geht es nicht primär um die vorrangige Anwendung einer Gruppe \textsl{inländischer} Sachnormen$[$FN 142$]$. Daß es unter diesen Gesetze gibt, die unabhängig vom berufenen \textquotedbl{}Statut\textquotedbl{} anzuwenden sind, war bei der Entstehung der Sonderanknüpfungslehre eine Selbstverständlichkeit, die konstatiert wurde, ohne daß man sich zu einer umfassenden Klassifizierung dieser Gesetze sonderlich herausgefordert gesehen hätte.$[$FN 143$]$

$[$S. 41$]$

Die \textquotedbl{}Sonderanknüpfung\textquotedbl{} entwickelte sich seitdem zu einem ... kollisionsrechtlichen Instrument... so besonders im internationalen Devisen- und Wirtschaftsrecht$[$FN 157$]$, auch etwa im Kartellrecht$[$FN 158$]$, Wettbewerbsrecht$[$FN 159$]$, Arbeitsrecht $[$FN 160$]$, Konzernrecht $[$FN 161$]$, bei allen öffentlichrechtlichen Eingriffen in private Rechtsverhältnisse$[$FN 162$]$ und auch überall dort, wo es um den \textquotedbl{}Schutz des Schwächeren\textquotedbl{} geht$[$FN 163$]$.
----
$[$FN 140$]$ \textsl{Wengler}, Ankn.

$[$FN 141$]$ Vgl. etwa \textsl{Neuhaus}, Wege 425 $[$...$]$ \textsl{Schwander}, Lois d'appl. imm 316-376 $[$...$]$

$[$FN 142$]$ Häufig wird aber auch beides als \textquotedbl{}Sonderanknüpfung\textquotedbl{} bezeichnet. $[$...$]$

$[$FN 143$]$ Vgl. \textsl{Wengler}, Ankn. 168-180.}
\end{fragmentpart}
\begin{fragmentpart}{Anmerkung}
Schurig wird an dritter Stelle in Fußnote 17 und an zweiter Stelle in Fußnote 18 genannt. Alternativ ist eine Einordnung als Verschleierung denkbar.

Fortsetzung in \hyperlink{Lm-Fragment-207-01}{Fragment\_207\_01} --- dort sind auch die Fußnoten 141-143 (Lm) sowie 156-163 (Schurig) wiedergegeben.
\end{fragmentpart}
\end{fragment}
\phantomsection{}
\belowpdfbookmark{Fragment 207 1--13}{Lm-Fragment-207-01}
\hypertarget{Lm-Fragment-207-01}{}
\begin{fragment}
\begin{fragmentpart}{Dissertation S.~207 Z.~1--13 (BauernOpfer)}
\enquote{$[$Die Sonderanküpfüng hat eine besondere Bedeutung im internationalen Devisen- und Wirtschaftsrecht,$[$FN 119$]$ auch im Kartellrecht, Wettbewerbs-,$[$FN 120$]$ Ar-$]$

beits-$[$FN 121$]$ und Konzernrecht,$[$FN 122$]$ kurz, bei allen öffentlichrechtlichen Eingriffen in private Rechtsverhältnisse$[$FN 123$]$ und auch überall dort, wo der „Schutz des Schwächeren“$[$FN 124$]$ im Vordergrund steht.

Heute erscheint es freilich geboten, zwischen besonderer Anknüpfung und Sonderanknüpfung zu unterscheiden.$[$FN 125$]$ Unter der besonderen Anküpfung können wir eine Anknüpfung eigener oder fremder Sachnormen verstehen, die wegen der besonderen kollisionsrechtlichen Interessenabwägung unter die bisherigen „allseitigen Sammelmappen“ nicht paßt.$[$FN 126$]$ Solche Vorgänge sind üblicherweise Bestandteil multilateralistischer Systeme, solange der Anwendungsbereich für diese besonderen Normen autonom bestimmt wird. Sonderanknüpfungen markieren dagegen \textsl{den methodischen Wechsel zur unilateralistischen Ausgangsposition,} d.h. zu globalen Kollisionsgrundnormen, die das auf jeweils eigene besondere Sachnormen bezogene Kollisionsrecht aller Staaten ungezielt aktivieren.$[$FN 127$]$
----

$[$FN 120$]$ Cf. MEESSEN, Zu den Grundlagen des internationalen Wirtschaftsrechts, AöR 110 (1985), S. 398-418, 407-416; Deutsch, Wettbewerbstatbestände mit Auslandsbeziehung, Stuttgart 1962, S. 21, 42f. und passim.

$[$FN 121$]$ Cf. DÄUBLIER, Grundprobleme des internationalen Arbeitsrechts, AWD/RIW 1972, S. 1-12, 8-12; Gamillscheg, Ein Gesetz über das internationale Arbeitsrecht, ZfA 14 (1983), S. 307-373, 348f.; SCHLUNCK, Die Grenzen der Parteiautonomie im internationalen Arbeitsrecht, München 1990, S. 57, 60, 109f., 114f., 195; KREBBER, Internationales Privatrecht des Kündigungsschutzes bei Arbeitsverhältnissen, Baden-Baden 1997, S. 237-283.

$[$FN 122$]$ Cf. NEUMAYER, Betrachtungen zum internationalen Konzernrecht, ZVglRwiss 83 (1984), S. 129-177, 13lf.

$[$FN 123$]$ Cf. NEUHAUS, Empfiehlt sich eine Kodifizierung des IPR? RabelsZ 37 (1973), S. 453-465, 456.

$[$FN 124$]$ Cf. VON HOFFMANN, Über den Schutz des Schwächeren bei internationalen Schuldverträgen, RabelsZ 38 (1974), S. 396-420; ablehnend KROPHOLLER, Das kollisiosrechtliche $[$sic$]$ System des Schutzes der schwächeren Vertragspartei, RabelsZ 42 (1978), S. 634-661; der EVÜ hat deutlich die schwächeren Parteien (im Verbraucher- und Arbeitsverträgen) mit Sonderanknüpfungen geschützt; cf. auch Art. 7 EVÜ und Art. 18 des schweizerischen IPRG. 

$[$FN 125$]$ Cf. KROPHOLLER, ibid., S. 659; SCHURIG, Kollisionsnorm und Sachrecht, S. 323.

$[$FN 126$]$ Etwa im Sinne des $[$sic$]$ U.S.-amerikanischen \textsl{dépéçage} $[$sic$]$.

$[$FN 127$]$ Cf. SCHURIG, Kollisionsnorm und Sachrecht, S. 323.}
\end{fragmentpart}
\begin{fragmentpart}{Original \cite[S.~41,~323 Z.~5--13,~3--15]{Schurig-1981}}
\enquote{$[$S. 41$]$

Die \textquotedbl{}Sonderanknüpfung\textquotedbl{} entwickelte sich seitdem zu einem ... kollisionsrechtlichen Instrument... so besonders im internationalen Devisen- und Wirtschaftsrecht$[$FN 157$]$, auch etwa im Kartellrecht$[$FN 158$]$, Wettbewerbsrecht$[$159$]$, Arbeitsrecht $[$FN 160$]$, Konzernrecht $[$FN 161$]$, bei allen öffentlichrechtlichen Eingriffen in private Rechtsverhältnisse$[$162$]$ und auch überall dort, wo es um den \textquotedbl{}Schutz des Schwächeren\textquotedbl{} geht$[$FN 163$]$. 

$[$S. 323$]$

... Schritt von der „besonderen Anknüpfung“ zur „Sonderanknüpfung“. Diese beiden Begriffe bezeichnen — trotz ihres ähnlichen Klanges — etwas Grundverschiedenes$[$FN 228$]$. Unter der „besonderen Anknüpfung“ können wir eine Anknüpfung eigener oder fremder Sachnormen verstehen, die wegen der besonderen kollisionsrechtlichen Interessenkonstellation unter die (bisherigen) allseitigen Bündelungen nicht paßt. Solche Vorgänge sind normaler Bestandteil des „multilateralistischen“ Systems, solange der Anwendungsbereich für diese „besonderen“ Normen „autonom“ von uns bestimmt wird. „Sonderanknüpfung“ dagegen markiert den methodischen Wechsel zur unilateralistischen Ausgangsposition, zu — zumindest primär — ungezielt globalen Kollisionsgrundnormen, die das auf die jeweils „eigenen“ besonderen Sachnormen bezogene Kollisionsrecht zunächst aller Staaten berufen...
----
$[$FN S. 41$]$

$[$FN 156$]$ Neumayer, Auton.; ders. Notges.; ferner Rehbinder, Polit. 155 --- 158; von Hoffmann, Schutz 409-415.

$[$FN 157$]$ Neumayer, Notges.; Joerges, Int. Wirtssch. R. 34 --- 39, 56f (zumindest als \textquotedbl{}unvermeidbare Ausweichstrategie\textquotedbl{}).

$[$FN 158$]$ Vgl. Mertens, KartR.; Schwartz, Int. KartR. 221-225; Rehbinder, Extr. KartR. 281-292; Gamm, Rechtsw. 1554.

$[$FN 159$]$ Joerges, Klass. Konz ...

$[$FN 160$]$ Ablehnend jedoch Gamillscheg, Ged. 832 --- 837 ...

$[$FN 161$]$ Jedenfalls z. T., vgl. Westermann, Ges. R. 86 --- 89.

$[$FN 162$]$ Neuhaus, Kod. 456.

$[$FN 163$]$ von Hoffmann, Schutz; zweifelnd Neuhaus, Grdbegr. 37; ablehnend Kropholler, Schw. Vertr. Part ... 

$[$FN S. 323:$]$

$[$FN 228$]$ Das hat sich freilich noch nicht überall durchgesetzt; so wird z.B. auch \textsl{Schwander}, Lois d’appl. imm. 316, 323, 373 --- 376, beide Erscheinungen in einen Topf. Zutreffend unterscheidet indessen \textsl{Kropholler}, Schw. Vert. part. 659, von der \textquotedbl{}Sonderanknüpfung\textquotedbl{} die \textquotedbl{}getrennte Anknüpfung\textquotedbl{}.}
\end{fragmentpart}
\begin{fragmentpart}{Anmerkung}
Fortsetzung von \hyperlink{Lm-Fragment-206-03}{Fragment\_206\_03}.

Schurig wird zweimal in den Fußnoten genannt.

Die dépeçage erscheint in Fußnote 126 im männlichen Genus und mit einem Akzent zuviel, anders z.B. durchgehend auf S. 247.
\end{fragmentpart}
\end{fragment}
\phantomsection{}
\belowpdfbookmark{Fragment 208 14--22}{Lm-Fragment-208-14}
\hypertarget{Lm-Fragment-208-14}{}
\begin{fragment}
\begin{fragmentpart}{Dissertation S.~208 Z.~14--22 (BauernOpfer)}
\enquote{e. Sachnormen im IPR

Da die Anknüpfung eines internationalen Sachverhaltes zum entscheidenden Charakteristikum des internationalen Privatrechts geworden ist, hat auch die Grenze zwischen Kollisionsrecht und Auslandsbeziehung als Sachverhalt an Schärfe eingebüßt. Sachnormen können im IPR$[$FN 131$]$ überall vermutet werden, wo durch Berührung mit fremden Rechtsordnungen das Gesamtbild des auf den Fall anzuwendenden Sachrechts in einem anderen Licht erscheint, sei es bei Fremdenrecht, Auslandssachverhalt, oder Angleichung, sei es bei Qualifikation, Mehrfachanknüpfung,$[$FN 132$]$ Sonderanknüpfung oder ännlichem. $[$sic!$]$ $[$...$]$ $[$FN 133$]$

$[$FN 131$]$ Cf. STEINDORFF, Sachnormen im IPR, passim und supra, Kapitel VI 1 d.

$[$FN 132$]$ Cf. HEYN, a.a.O., passim und supra, Kapitel VI 1 i.

$[$FN 133$]$ Cf. SCHURIG, Kollisionsnorm und Sachrecht, S. 323.}
\end{fragmentpart}
\begin{fragmentpart}{Original \cite[S.~42,~45 Z.~42--1,~45--4]{Schurig-1981}}
\enquote{$[$S. 42$]$

\textsl{4. Bildung von „Sachnormen im IPR“}

$[$S. 45$]$

Da entscheidendes Charakteristikum des internationalen Privatrechts die internationale Verknüpfung des Sachverhalts geworden ist, schwindet auch die Grenze zwischen Kollisionsrecht und auf Auslandsbeziehungen Rücksicht nehmendem Sachenrecht. „Sachnormen im IPR“ können dann überall vermutet werden, wo durch Berührung mit fremden Rechtsordnungen das Gesamtbild des auf den Fall anzuwendenden Sachrechts in einem anderen Licht erscheint, sei es etwa bei Fremdenrecht, Auslandssachverhalt, Angleichung, Qualifikation, Mehrfachanknüpfung, „Sonderanknüpfung“ oder ähnlichem$[$FN 183$]$.

$[$183$]$ Vgl. insbes. \textsl{Steindorff}, Sachnormen, passim.}
\end{fragmentpart}
\begin{fragmentpart}{Anmerkung}
Schurig wird in Fußnote 133 genannt.
\end{fragmentpart}
\end{fragment}
\phantomsection{}
\belowpdfbookmark{Fragment 210 1--2,~6--12}{Lm-Fragment-210-01}
\hypertarget{Lm-Fragment-210-01}{}
\begin{fragment}
\begin{fragmentpart}{Dissertation S.~210 Z.~1--2,~6--12 (Verschleierung)}
\enquote{X. Kapitel:

Der Qualifikationsprozeß

$[$...$]$ Wichtig erscheint nun, daß, wenn man über die Qualifikation als Prozeß nachdenkt, zunächst drei Dinge auffallen: Zum einen, daß viele Autoren ein System der Qualifikation nach verschiedenen Fragen, Stufen, Graden oder Schritten entwickelt haben.$[$1$]$ Zweitens, daß mehrere Autoren ein „Aneinandervorbeiirren“ beklagen.$[$2$]$ Drittens die Feststellung, daß man im englischsprachigen Raum für die Qualifikation nicht einmal einen einheitlichen Terminus akzeptiert hat.$[$3$]$

$[$1$]$ Vgl. supra, Kapitel IX 2 d aa. (1) --- (3). $[$...$]$

$[$2$]$ Cf. Weber, a.a.O., S. 215 m.w.N..

$[$3$]$ Vgl. supra, Kapitel II 2.}
\end{fragmentpart}
\begin{fragmentpart}{Original \cite[S.~215 Z.~21--30]{Weber-1986}}
\enquote{A. Qualifikation als Prozeß

Geht man von dieser Fragestellung aus und überblickt man noch einmal die in den vorhergehenden Kapiteln ausführlich dargestellte Diskussion, so fallen zunächst drei Dinge auf: Zum einen, daß so viele Autoren ein System der Qualifikation nach verschiedenen Fragen, Stufen, Graden, Schritten oder Ordnungen entwickelt haben. Zweitens, daß mehrere Autoren ein Aneinandervorbeireden beklagen$[$2$]$. Das dritte ist die Feststellung, daß man im englischsprachigen Bereich für die Qualifikation nicht einmal ein einheitliches Wort gefunden hat und teils von characterization, teils von classification spricht.

$[$2$]$ Z. B. Raape, 1. Aufl., S. 74; Ferid, IPR, S. 96.}
\end{fragmentpart}
\begin{fragmentpart}{Anmerkung}
Im analytischen Teil seiner Arbeit übernimmt Lm drei Grundfragen für ein (Teil-)Kapitel von Weber, der nur für einen dieser drei Fragen in einer Fußnote erwähnt wird. Das scheinbar direkte Zitat --- \textquotedbl{}Aneinandervorbeiirren\textquotedbl{} --- ist nicht korrekt. Weber spricht von \textquotedbl{}Aneinandervorbeireden\textquotedbl{} (dort und S. 214, 220).
\end{fragmentpart}
\end{fragment}
\phantomsection{}
\belowpdfbookmark{Fragment 217 5--11}{Lm-Fragment-217-05}
\hypertarget{Lm-Fragment-217-05}{}
\begin{fragment}
\begin{fragmentpart}{Dissertation S.~217 Z.~5--11 (BauernOpfer)}
\enquote{Wer sich streng am Wortsinn orientiert, wird leicht eine Antwort geben können: nur bei der Subsumtion. Wer weiter blickt, wird eher dazu neigen, alle möglicherweise erforderlichen Auslegungs- und Subsumtionsakte dazuzurechnen. Es ist ferner denkbar, daß man Auslegung und Subsumtion nur auf der Tatbestandsseite der Kollisionsnorm oder nur auf der Rechtsfolgenseite als Qualifikation bezeichnet, oder daß man nur die Auslegung der kollisionsrechtlichen Begriffe dazu rechnet.$[$FN 22$]$
----
$[$FN 22$]$ Cf. MÜNCHKOMM-SONNENBERGER, Einl., Rn. 343.}
\end{fragmentpart}
\begin{fragmentpart}{Original \cite[S.~145 Z.~343]{Sonnenberger-1990}}
\enquote{Wer sich an den strengsten Wortsinn hält, wird die Antwort geben: nur bei den Subsumtionen. Wer vom weiteren Wortsinn ausgeht, wird dazu neigen, die möglicherweise erforderlichen Auslegungs- und Subsumtionsakte dazuzurechnen. Es ist ferner denkbar, daß man nur Auslegung und Subsumtion auf der Tatbestandsseite der Kollisionsnorm oder auf ihrer Rechtsfolgenseite als Qualifikation bezeichnet, oder daß man nur die Auslegung der kollisionsrechtlichen Begriffe dazu rechnet.}
\end{fragmentpart}
\begin{fragmentpart}{Anmerkung}
Fußnote 22 verweist mit »Cf.« auf Sonnenberger. Die Art der Übernahme – wörtlich mit minimalen Eingriffen – ist so nicht erkennbar.
\end{fragmentpart}
\end{fragment}
\phantomsection{}
\belowpdfbookmark{Fragment 232 4--6}{Lm-Fragment-232-04}
\hypertarget{Lm-Fragment-232-04}{}
\begin{fragment}
\begin{fragmentpart}{Dissertation S.~232 Z.~4--6 (Verschleierung)}
\enquote{Will man sich mit der Analyse ANCELS auseinandersetzen, gilt es, folgendes zu beachten: der \textsl{lien}, von dem er spricht, mag zwar in beiden Stufen richterlichen Qualifizierens das Subsumtionsobjekt bilden.}
\end{fragmentpart}
\begin{fragmentpart}{Original \cite[S.~29 Z.~28--30]{Heyn-1986}}
\enquote{Will man sich mit der Analyse Ancels auseinandersetzen, gilt es folgendes zu beachten: der „lien“ , von dem Ancel spricht, mag zwar in beiden Stufen das Subsumtionsobjekt bilden.}
\end{fragmentpart}
\begin{fragmentpart}{Anmerkung}
Heyn wird in diesem Zusammenhang nicht genannt. Wiederholung eines Teils der Übernahme in \hyperlink{Lm-Fragment-024-07}{Fragment\_024\_07}.
\end{fragmentpart}
\end{fragment}
\phantomsection{}
\belowpdfbookmark{Fragment 247 6--10}{Lm-Fragment-247-06}
\hypertarget{Lm-Fragment-247-06}{}
\begin{fragment}
\begin{fragmentpart}{Dissertation S.~247 Z.~6--10 (BauernOpfer)}
\enquote{Bei der \textsl{dépeçage} handelt es sich nach h.M. um eine Zerstückelung umfassender Kollisionsnormen. Sie liegt vor, wenn an die Stelle einer einzigen Kollisionsnorm
(z.B. fur das gesamte Deliktsrecht) unterschiedliche Anknüpfungsregeln für Einzelfragen treten sollen, wenn also z.B. für das Schmerzensgeld, die Produkthaftung, etc. je eine Sonderanknüpfung befürwortet wird.$[$FN 9$]$

$[$FN 99$]$ Cf. VON BAR, IPR I, Rn. 28; JAYME, a.a.O. (Fn. 96), S. 267f.; KEGEL, IPR7, § 2 II 3 b) a.E.; KROPHOLLER, IPR3, § 18 I vor 1.}
\end{fragmentpart}
\begin{fragmentpart}{Original \cite[S.~21 Z.~17--22]{Von-Bar-1987}}
\enquote{\textsl{Dépeçage:} \textquotedbl{}Zerstückelung\textquotedbl{} umfassender Kollisionsnormen. Sie liegt vor, wenn an die Stelle einer einzigen Kollisionsnorm (z.B. für das gesamte außervertragliche Haftungsrecht) unterschiedliche Anknüpfungsregeln für Einzelfragen treten sollen (wenn also z.B für das Schmerzensgeld, für die Produzentehnaftung, für den Verkehrsunfall, für die schuldunabhängige Haftung etc. je eine \textquotedbl{}Sonderanknüpfung\textquotedbl{} befürwortet wird).}
\end{fragmentpart}
\begin{fragmentpart}{Anmerkung}
von Bar wird in Fußnote 99 als einer von vier Belegen für eine dahingehende herrschende Meinung angegeben. Kropholler definiert an der angegebenen Stelle dépeçage anders (\textquotedbl{}Meist denkt man dabei an den Fall, daß die Beurteilung eines Vertrages, der Berührungspunkte zu mehreren Rechtsordnungen aufweist, in Einzelfragen aufgespalten und auf diese Weise mehreren Rechtsordnungen zugewiesen wird\textquotedbl{}), Kegel definiert an der gegebenen Stelle die dépeçage gar nicht. Die fehlende Kennzeichnung als Zitat spiegelt eine Einigkeit vor, die tatsächlich nicht besteht.
\end{fragmentpart}
\end{fragment}
\bibliographystyle{dinat-custom}
\renewcommand{\refname}{Quellenverzeichnis}
\bibliography{ab}
\end{document}
